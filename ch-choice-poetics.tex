\chapter{Choice Poetics}

\label{ch:choice-poetics}

The aim of this chapter is to lay the foundations of a theory of choice poetics.
%
In other words: to establish a theoretical framework for understanding the effects of choices in narrative contexts.
%
For example, what is it about a particular choice that makes some players feel regret after picking a particular option?
%
This chapter does not present a fully-developed theory, but rather attempts to establish terminology and formal perspectives that could be used as the basis of such a theory.
%
Some parts of the theory are developed (those that supported and were influenced by the development of \dunyazad/), but the ideas here are just the beginning of an in-depth study of narrative choices.



\section{Inspirations}

\citep{Carr2009}
\citep{Bizzocchi2011}


\section{Modes of Engagement}


In developing a theory of choice poetics, one must respect the lessons of the development of traditional poetics.
%
In particular, modern scholars have largely rejected ``objective'' narrative analysis and concluded that the experience of the individual reader must be taken into account.
%
Unlike Aristotle, who stated authoritatively what was good and bad about this drama or that, modern narrative scholars will analyse a work from a particular perspective, without claiming that this perspective is universal, perhaps performing a feminist or Marxist reading of a novel to see what insights a particular lens has to offer about the work.


Choice poetics must also respect the perspective of the reader, but it is in a slightly different position, as its readers are also players.
%
Insofar as the narrative that contains choices is experienced through the fabric of a game, the reader/players (henceforth players) are stepping into the magic circle of the game \citep{Huizinga1949} and intentionally taking on certain attitudes.
%
A narrative experienced within the framework of a game thus implicitly biases the attitudes of its readers, for example by setting up a score counter which players can try to maximize.
%
Counter-play is of course possible and important, but even counter-play is influenced by the rules of the game in virtue of being set against them.
%
The formal structures of the game rules that accompany a narrative-with-choices thus allow the theoretician to make an educated guess about the attitudes of players interacting with a piece.


There are still a range of attitudes that players can take on, however, and some understanding of this range is important before any analysis of choices in a narrative.
%
This range of attitudes has to some degree been studied by others interested in player types, although such studies are mostly focused on games without regard to narrative.
%
From simple binary ``casual/hardcore'' distinctions to more complex typologies such as Bartle's ``achievers,'' ``explorers,'' ``socializers,'' and ``killers,'' \citep{Bartle1996} player type classifications to some degree incorporate notions of player intent or attitude.
%
However, player type classifications also focus on player actions or approaches within the game, for example a distinction between those who enjoy stealth or direct combat more.
%
These player content preferences are as important as any other player preferences when it comes to how players approach choices, but they're less important than player motivations, which establish entire contexts within which preferences can be expressed.
%
For example, if one's motivation to play is a desire to act out a role, one's preferences might be expressed in terms of the kind of role one is trying to act out.
%
If one's motivation to play is to score points, on the other hand, preferences might be expressed in terms of different strategies for doing so.


The idea of ``modes of engagement'' captures these different player motivations.
%
Modes of engagement are different ways in which a player can approach a game.
%
At this point it is important to note that modes of engagement are neither exclusive nor permanent: players often engage in several modes at once (to varying degrees) and may change their modes of engagement during the course of play.
%
The modes of engagement presented here are also not intended to be comprehensive: these are some of the most common modes, but others may be possible and even the norm for certain games.
%
All of that notwithstanding, consideration of the mode(s) of engagement that players bring to a work is the first step in analyzing the choice poetics of a work.
%
This can take the form of assuming a particular mode: much as one can perform a feminist reading of a novel, one could perform a power-playing playthrough of a game.
%
This could also take the form of analyzing the game itself to determine which mode(s) of engagement it encourages and rewards.
%
It could also take the form of a qualitative study of actual players, to determine which mode(s) of engagement they are employing.
%
In any case, an analysis of choice poetics that does not consider modes of engagement is incomplete.

\Cref{tab:modes-of-engagement} lists some of the most common modes of engagement, and gives examples of decisions that employ them.
%
The last column references Nick Yee's work on motivations for play in massively-multiplayer online role-playing games as a comparison \citep{Yee2006}.
%
The grouping of Yee's motivation components into these modes of engagement has to do with a difference in focus: Yee is focused primarily on aspects of play, including a strong focus on online social interaction, whereas choice poetics is interested primarily in single-player offline engagement with an emphasis on narrative.
%
For example, Yee makes fine distinctions between different motivations grouped into the ``power play'' category here, but from a choice poetics perspective, whether someone is trying to maximize score or compete with an opponent doesn't matter, because both motivations are equally orthogonal to the narrative, and thus they have a similar effect on the player's experience of the narrative.


\begin{table}[h]
\begingroup
\renewcommand*{\arraystretch}{1.5}
\begin{tabular}{p{5em}p{10em}p{10em}p{7em}}
\textbf{Mode} & \textbf{Decision Process} & \textbf{Example} & \textbf{\citep{Yee2006}} \\
\hline
\textbf{Avatar Play} & Decide as if you were in the character's situation. & When picking a pet, pick the cat because you like cats. & Role-Playing, Customization, Escapism \\
\textbf{Role Play} & Decide in order to act out a persona. & Choose the wizard character class because you want to play a shy, bookish person. & Role-Playing, Customization, Escapism \\
\textbf{Power Play} & Choose options that advance game metrics like score, beating other players, or quick completion. & Sacrifice an ally to obtain a powerful item because it helps you beat the game more quickly. & Advancement, Mechanics, Competition, Teamwork \\
\textbf{Explora-tory Play} & Choose options to see what will happen. & Turn away from the path of your quest to explore the world. & Discovery \\
\textbf{Social Play} & In a multiplayer situation, choose options because of social considerations. & Turn down a high-level quest in order to accompany your friend on a lower-level quest. & Socializing, \newline Relationship \\
\textbf{Analyti-cal Play} & Make decisions in order to analyze the work. & Repeatedly load a saved game to explore every possibility at a choice. & \emph{none} \\
\textbf{Critical Play} & Make decisions as performance to deconstruct or criticize a work. & Drive your character into poverty in order to demonstrate a game's biased depiction of poor people. & \emph{none} \\
\end{tabular}
\endgroup
\caption{Some common modes of engagement.}
\label{tab:modes-of-engagement}
\end{table}

TODO: HERE

\section{Dimensions of Player Experience}


\section{Choice Idioms}


\section{TODO: Some deeper analysis}
