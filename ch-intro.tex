\chapter{Introduction}

When I began the work described here, I was a computer scientist by training, and having recently discovered the existing work on computer-generated narrative, I had two initial impressions.
%
First, I was intrigued at the idea of applying artificial intelligence techniques to such an artistic domain (I had previously done research in AI for robots and game-playing).
%
Second, I was firmly convinced that there was great potential for improvement in this domain.
%
In particular, it seemed to me then (as it apparently has to a variety of computer scientists since at least the 1980s) that by capturing the underlying rules of what makes a story interesting, a computer ought to be able to produce interesting, albeit formulaic, stories.
%
Certain pieces of writing advice and narrative theory reinforced this second impression, and I saw one of the main goals of my dissertation as the creation of a story generation system that would outperform existing work.


As things turned out, a new and improved story generator is not one of the main contributions of my work.
%
Working first to reconstruct Scott Turner's \minstrel/ \cite{Turner1993} and then on my own \dunyazad/, I have begun to understand a little about what makes automatic narrative generation such a difficult problem, and in fact that is one of the contributions of this dissertation (although some of my lessons are perhaps those that every student in the area will learn through experience).
%
But the main contribution of this document is actually an example of a hybrid research method which advances both artificial intelligence and narrative theory at once by combining the two.


The idea of using AI (and AI research) as a means to explore other subjects goes back to Phil Agre's idea of critical technical practice \cite{Agre19XX}: engaging in technical AI research while addressing and respecting the critical implications of that work.
%
In my case, I am using a formal model of narrative choices to test and explore what I call ``choice poetics:'' a narrative theory of audience response to choice structures.
%
Both the process of constructing such a formal model and experiments that can be performed with it can help develop the theory of choice poetics.
%
At the same time, the AI system relies on the theory, and the result of the research is thus both an improved theory of choice poetics and a novel AI system.


\section{Contributions}

As implied earlier, one of the contributions of my work is some general knowledge about the strengths and limitations of certain approaches to story generation.
%
Working with Brandon Tearse on the Skald system, which is a rational reconstruction of Turner's \minstrel/, we learned not only the limitations of that system, but also some broadly applicable rules regarding case-based story generation.
%
In particular, the ability of any case-based story generation system to recombine content effectively is limited by its commonsense reasoning capabilities.

\section{Minstrel and Skald}

\section{Choice Poetics}

\section{\dunyazad/}

\section{Outline}
