\chapter{Related Work}

\label{ch:related-work}

The work described in this document derives mainly from the tradition in computer science of building programs that generate narratives.
%
While one aim of this project is to build a program which can intentionally generate narrative choices (a computer science concern: ``How can we build a system that does X?'') another is to use such a program to learn more about the nature of such choices (a media theory concern: ``How does the construction of narrative choices shape player reactions?'').
%
Because of this, it also owes much to narratological theories which have nothing to do with computing, and to ideas related to the psychology of reading, decision-making, and persuasion.
%
The remainder of this chapter discusses these ideas that \dunyazad/ builds on in depth, starting with theories of narrative and moving on to computer systems that generate narrative.

\section{Narrative Theory}
%===============================================================================

\subsection{The Foundations of Narrative Theory}
%===============================================================================

Aristotle is one of the earliest scholars to discuss narrative theory, and his \work{Poetics} is a treatise on the inner workings of Greek drama \citep{Aristotle1917}.
%
Aristotle discussed which plot constructions were more effective than others, and identified basic types of dramatic plot, including the comedy and the tragedy, attempting to explain using examples what properties were important in these forms.
%
Although the idea that audience response is a simple function of the content of a narrative work has been recognized as na\"ive, the ideas of Aristotle sowed the seeds of modern narrative theory.
%
Along with other ancient critiques of drama, such as Horace's \work{Ars Poetica} \citep{Horace1783}, \work{Poetics} helped establish drama, and by extension narrative, as cites of critical study.


An important next step in for narrative theory happened in 1894, when Gustav Freytag published \work{Technique of the Drama} \citep{Freytag1894}.
%
Freytag's ideas about the emotional arc of traditional dramas have become standard fare in today's classrooms when students first begin learning how to unpack meaning in narrative.
%
His work on the dramatic structure of the traditional 5-act drama represents a further step from critique or craft advice towards more formal theories of drama (and more generally, narrative).


\subsection{Formalism and Structuralism}
%===============================================================================

In the early 20th century, this formalization of narrative theory was taken even farther by a group of Russian literary critics known as the Russian formalists.
%
These scholars were exemplified by Vladimir Propp, whose 1928 \work{Morphology of the Folk Tale} \citep{Propp1971} dissected common characters and scenes within a group of Russian folk-tales and come up with an abstracted representation which revealed fundamental similarities between the stories.
%
Propp observed that a simple grammar of a set of thirty-one basic `functions' could account for the plots of all of the stories that he was analysing, thus creating an abstract and formalist description of their common structure.
%
Although this formal structure has proved an attractive theory for computer scientists interested in creating stories, some contemporary literary theorists disagreed with Propp, asserting that Propp's analysis neglected many important aspects of a story, such as tone and mood.
%
In Prop's (and the formalists') attempts to understand similarities common to diverse narratives, he (and they) are forced to ignore many of the details that make individual stories stand out.
%
Although these details may be safely abstracted while still understanding the \emph{sequence of events} embedded in a particular story, they are often crucial to the \emph{audience's experience}, and thus to poetics.


The Russian formalists began a movement within literary criticism that gave rise to the discipline of narratology.
%
Rather than explicating the details of an individual work or the works of an author or movement, scholars such as Greimas and Barthes sought structural theories that could explain broad phenomenon in many narrative contexts \citep{Barthes1975,Greimas1988}.
%
Central to narratology is the idea that to some degree, narratives can be separated into an abstract series of events (the \textit{fabula}, or story) and a particular telling of those events (the \textit{syuzhet}, or discourse).
%
While work like Propp's focuses almost myopically on story at the expense of discourse, some later narratologists focused on discourse-related effects such as focalization.
%
Narratologists created theories and methods of analysis that could be applied to a broad range of narrative contexts.
%
Particularly because of the abstract nature of these theories and methods, computer scientists interested in creating artificial story generators have drawn on them as inspiration for their systems.


\subsection{Cognitive and Psychological Narrative Theories}
%===============================================================================

Although narratology grew out of literary criticism, in the late 20th century researchers in fields such as psychology and cognitive science began to approach narrative from a new perspective.
%
Psychologists were interested in questions such as ``Why do humans enjoy stories?'' or ``How do humans comprehend stories?''
%
They put forth models of narrative comprehension \citep{Kintsch1980,Brewer1982}, and asked questions about why people liked stories \citep{Iran-Nejad1987}.


In the early 1990's, this work accelerated, with work on suspense \citep{Gerrig1994}, narrative inferences \citep{Graesser1994}, and emotion \citep{Oatley1995}.
%
Models of comprehension continued to develop, including the event-indexing model of Zwaan, Langston, and Graesser \citep{Zwaan1995}, which has influenced several recent works in the field of narrative intelligence (see below).
%
This interest in narrative consumption as a psychological phenomenon was largely separate from literary criticism and theory, in large part because it dealt with much lower-level questions.
%
However, in the early 2000's, works like Palmer's \work{Fictional Minds} \citep{Palmer2004} and Zunshine's \work{Why We Read Fiction} \citep{Zunshine2006} began to connect literary theory with psychology.
%
Zunshine in particular draws on the work of psychologists like Oatley (e.g., \citep{Oatley1999}) to talk about how novels exercise our theory of mind when we read them.
%
Her work is at heart a work of literary criticism, however, and she uses critical analyses of several well-regarded novels as the basis of her argument.
%
\work{Why We Read Fiction} is an example of cross-disciplinary analysis: new theories from psychology inspired her to understand well-theorized works from a new perspective.
%
Zunshine has begun the work of combining psychological and critical perspectives on narrative, but of course there is still much unexplored overlapping territory between these two disciplines.


\subsection{Craft and Literary Criticism}
%===============================================================================

Although narratologists and psychologists have attempted to find a more `rigorous' and `objective' understanding of narrative, more traditional literary criticism and practical advice provides another source of knowledge about narrative.
%
Novelists, writing teachers, and critics (who often wear more than one of these hats) have write down their own views on narrative for both pedagogical and critical purposes.
%
Works like E. M. Forster's \work{Aspects of the Novel} \citep{Forster1927} offer insight into how producers and critics view narrative.
%
As someone who is acknowledged as skilled at the craft of narrative composition, Forster is in a unique place to observe and reveal underlying aspects common across different narrative contexts.


\work{Aspects of the Novel} and other works like it critically provide practical advice aimed at prospective authors attempting to \emph{create} effects such as foreshadowing or suspense.
%
It should not be surprising therefore that these resources are sometimes more useful than psychological analyses of the experience of these effects for researchers interested in automatically generating narrative.
%
Of course, craft and critical advice also exist for interactive works, as outlined below.


\section{Theories of Interaction}
%===============================================================================

Johan Huizinga's \work{Homo Ludens} is a seminal work on the history and culture of play which laid the foundation for the study of games as a medium \citep{Huizinga1949}.
%
Although Huizinga focused broadly on \emph{play} as a social phenomenon, his major philosophical points apply equally to games as a cultural medium.
%
Ideas like the transformation of norms afforded by the magic circle are just as applicable to the culture of modern videogames as they were to the historical play contexts that Huizinga studied.


Although interactive narratives are by no means a modern invention, the rise of available leisure time in combination with the advent of the computer drastically transformed them over the course of the 20th century.
%
The rise of media like tabletop role-playing games and wargames (documented in \citep{Peterson2012}) coincided with the development of gamebooks (inspired by the work of Borges \citep{Borges1956} and developed by writers like Queneau at the Oulipo, see e.g., \citep{Queneau1967}).
%
Both of these interactive media gained popularity in the 1960's and '70's which was also when the first electronic games began to reach mainstream popularity (the original arcade \work{Pong} appeared in 1972).
%
By the late '70's and into the early '80's, a commodity market for computer-enabled interactive fiction began to appear with titles like \work{Zork} ultimately selling hundreds of thousands of copies \citep{Zork}.


A fourth interrelated development during this period was the advent of hypertext, which also drew inspiration from Borges' \work{The Garden of Forking Paths}.
%
Developed for research rather than entertainment, hypertext nonetheless eventually attracted the interest of writers.
%
Compared to the authors of gamebooks, these writers were interested in the medium more as a new experimental form than as an extension of existing forms like tabletop roleplaying and adventure novels.
%
Because of this, authors like Michael Joyce and Shelley Jackson, who were sometimes also developers of hypertext systems, experimented extensively with their work \citep{Joyce1990, Jackson1995}.


\subsection{Theories of Interactive Fiction}
%===============================================================================

Brenda Laurel's 1986 thesis titled \work{Toward the Design of a Computer-Based Interactive Fantasy System} is one of the earliest theoretical works that discusses computer-enabled interactive fiction \citep{Laurel1986}.
%
Laurel's thesis was a visionary description of a future medium in which computers would enable a drastically more interactive form of theater.
%
Along with other early work like Smith and Bates' 1989 \work{Towards a Theory of Narrative for Interactive Fiction} \citep{Smith1989} Laurel's thesis was inspired by interactive fiction software and looked forward to potential future forms of interactive narrative.


Marie-Laure Ryan's \work{Possible Worlds, Artificial Intelligence, and Narrative Theory}, published in 1991, makes a slightly different connection between narrative and computers: it brings ideas from artificial intelligence (and philosophy) to bear on narratology \citep{Ryan1991}.
%
Ryan's work represents a link between traditional narratology and the study of interactive narratives.
%
Notably, she cites early computer story-generation systems such as \work{Tale-Spin} and \work{Universe} (discussed below).


As computer games became more main-stream, work that theorized about interactive narrative as a medium proliferated. 
%
In 1996, Richard Bartle wrote about different player types in multi-user dungeons (MUDs) \citep{Bartle1996}.
%
Bartle's work represents a vein of inquiry that focuses on players as active agents within interactive fiction, and is analogous to theories of reader response in literary criticism.
%
The following year, Espen Aarseth published \work{Cybertext: Perspectives on Ergodic Literature} and Janet Murray published \work{Hamlet on the Holodeck: The Future of Narrative in Cyberspace} \citep{Aarseth1997,Murray1997}.
%
These two books are foundational to the study of electronic interactive fiction, and they both include significant theoretical components based on analysis of the interactive fiction of the preceding decades.


Aarseth's Cybertext connects the dots between gamebooks, hypertext, and interactive fiction.
%
In all of these forms, he sees ``ergodic literature:'' narratives which are not merely present to be observed, but which require work to traversed.
%
The theory of choice poetics presented here is concerned with one manifestation of this phenomenon: choices as a site of player work within the context of interactive stories.
%
By focusing on choices as a specific form of interaction, choice poetics hopes to understand how authors set up interactive fiction so that the work that players do contributes to the thematic qualities of the narrative players experience.
%
In this sense, choice poetics fits into the broader theory of ergodic literature as an in-depth study of a particular form of player-work in interactive narratives.


Murray's \work{Hamlet on the Holodeck} explores the concept of computer-driven interactive theater envisioned a decade prior by Laurel, but in the context of a vastly different ecosystem of computer games.
%
Between 1986, when Laurel published her thesis, and 1997, when Murray published her book, electronic entertainment had undergone radical changes, including the advent of three-dimensional graphics in games and mass-market personal gaming (in America, the Nintendo Entertainment System was released in 1985; the Nintendo 64 was released in 1995).
%
Murray devotes an entire section of her book to aesthetics, with chapters on immersion, agency, and transformation.
%
These concepts, especially agency, have become central in the study of electronic game aesthetics, and choice poetics fits within this framework as well as a study of one microcosm of interactive experience.
%
While Murray talks broadly about the pleasures and annoyances of interactive media, choice poetics is interested in investigating the detailed mechanisms by which choices as a unit of interaction give rise to specific feelings.
%
In a sense, choice poetics as a study of detailed mechanisms is only possible in a world in which authors like Aarseth and Murray have first explored the larger shape of the space of interactive narrative.


Drawing on the work of Murray and Laurel, Michael Mateas published \work{A Preliminary Poetics for Interactive Drama and Games} in 2001 \citep{Mateas2001}, uniting ideas from Aristotle's \work{Poetics} with Murray's discussion of agency to propose the idea of agency as a balance between formal and material affordances.
%
By this, Mateas meant that the player's sense of agency in an interactive experience depends on a balance between actions that the system allows them to do and agendas that the system encourages them to pursue.
%
According to this analysis, a game like \work{Quake} \citep{Quake} is a high-agency game because the actions available to the player (killing enemies and progressing through levels) are aligned with the agendas that the fictional world suggests (killing the otherworldly invaders and finding a way to end the invasion of Earth).
%



%\citep{Flanagan2009} // critical play
%\citep{Mateas2009} // operational logics
%\citep{Ryan2009} // poetics of intnar
%\citep{Bizzocchi2011} // close reading of games
%\citep{Mitchell2012} // rereading for closure
%\citep{Mason2013} // games & links
%\citep{Treanor2013} // procedural rhetoric

\subsection{Hypertext Theory}
%===============================================================================

Parallel to the

\citep{Bernstein1998}
\citep{Morgan1999}
\citep{Tosca1999}
\citep{Tosca2000}
% TODO: Other hypertext theory authors...

\subsection{Craft and Criticism of Interactive Narratives}
%===============================================================================

\subsection{The Psychology of Decisions and Persuasion}
%===============================================================================


\section{The Foundations of Computational Narrative}
%===============================================================================

\cite{Klein1971, Klein1973}
\cite{Meehan1976}
\cite{Lebowitz1984}


\section{Modern Computational Narrative}
%===============================================================================


\subsection{Interactive Narrative}
%-------------------------------------------------------------------------------

\cite{Laurel1986}
% TODO: Other IntNar theory authors...


\subsection{Constraint-Based Systems}
%-------------------------------------------------------------------------------


\subsection{Other Related Systems}
%-------------------------------------------------------------------------------


% TODO: Use this?
%\section{Related Work}
%
%This study is part of a recent trend focusing on choices in narrative, and several groups have published interesting results.
%%
%In 2011, Thue, Bulitko, Spetch, and Romanuik demonstrated \work{PaSSAGE}'s ability to increase player perceptions of agency by selecting content that players liked more \citep{Thue2011}.
%%
%This study demonstrated a link between desirable content and perceptions of agency.
%%
%The \work{PaSSAGE} system itself does directly construct choices, however, nor does it consider choice poetics in its operation.
%Instead, it uses online player modelling to determine what kinds of content a player prefers, and selects from pre-written quest alternatives based on that.
%
%
%Fendt, Harrison, Ware, Cardona-Rivera, and Roberts showed that direct feedback after players make a choice can create an illusion of agency, albeit in an extremely simple interactive narrative \citep{Fendt2012}.
%%
%In this study, the choices were part of a hand-written adventure game, whereas \dunyazad/ generates choices on its own.
%%
%This study concentrated on how different choice structures (mainly in terms of outcome text) affected players' perceptions of agency.
%
%
%In a similar study, Cardona-Rivera, Robertson, Ware, Harrison, Roberts, and Young found that choices where the options lead to significantly divergent outcomes increase feelings of agency relative to choices with similar outcomes \citep{Cardona-Rivera2014}.
%%
%Again this study used hand-written content as the test material; it focused on the connections between agency and divergence of outcomes.
%
%
%In another study using the Choose-Your-Own-Adventure format, Yu and Riedl were able to predict and influence players' choices using collaborative filtering \citep{Yu2013}.
%%
%Like the Fendt et al. and Cardona-Rivera et al. studies, this study used hand-written text, but this study chose to adapt two existing Choose-Your-Own-Adventure books for their study, rather than creating their own stories.
%%
%This resulted in much more complex and sophisticated stories, which more closely mimics the actual conditions of entertainment consumption.
%%
%Yu and Riedl's system did not generate options, however, it simply selected which options to present at a choice out of a number of hand-written alternatives that conveyed different motives for choosing a particular path.
%%
%This is a form of choice construction (since the exact options are dynamic) but the framing and outcomes are not dynamic and the options themselves are not constructed by the system but merely selected.
%
%Nevertheless, Yu and Riedl succeeded in using collaborative filtering to predict which of several motivations for taking the same action would be most appealing to players, and they were able to use a drama manager to guide players' choices.
%%
%Although their system only focused on guiding each player towards content that the collaborative filtering predicted that player would enjoy, one could imagine a more complex version that might try to influence players' perceptions of the choices themselves.
%%
%In contrast to Yu and Riedl's system, \dunyazad/ uses a logical model of player expectations that includes authorial input. 
%%
%While collaborative filtering allows a system to make user-specific adjustments and potentially allows fine-grained discrimination of player preferences, it is also quite opaque.
%%
%\dunyazad/'s logical model, while not adaptive, is more useful for informing a theory of choice poetics, and as our results demonstrate, it gets the job done.
%%
%That said, the use of online player data is something highly desirable for a system like \dunyazad/, and implementing something like Yu and Riedl's collaborative filtering is a tempting direction for future work.
%
%
%These studies are mostly focused on a single particular aspect of choice poetics (such as agency) whereas \dunyazad/ attempts to address constructing poetic choices more broadly.
%%
%Additionally, none of the studies mentioned thus far involved narrative generation systems that construct stories from raw material.
%%
%Considering constructive systems that are similar to \dunyazad/, Szilas' \work{IDtension} and El-Nasr's \work{Mirage} both stand out as interactive narrative systems rooted in theories about how narrative works \citep{Szilas2003,El-Nasr2007}.
%%
%\work{IDtension} is based on traditional dramatic theory (i.e., normal poetics) whereas \work{Mirage} incorporates theories of drama in the theatrical sense.
%%
%\dunyazad/ is also guided by a narrative theory: it is based on choice poetics--the theory of how choices are perceived by an audience \citep{Mawhorter2014}.
%%
%In this paper, we focus on \dunyazad/'s ability to generate individual choices that manipulate player expectations--obvious choices, relaxed choices, and dilemmas.
%%
%In some ways, this is reminiscent of recent narrative generation systems that have focused on specific traditional poetic effects, such as \work{Suspenser} (focused on suspense) and \work{Prevoyant} (focused on foreshadowing) \citep{Cheong2006,Bae2008}.


\section{}
