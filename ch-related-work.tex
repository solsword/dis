\chapter{Related Work}

\label{ch:related-work}

The work described in this document derives mainly from the tradition in computer science of building programs that generate narratives.
%
While one aim of this project is to build a program which can intentionally generate narrative choices (a computer science concern: ``How can we build a system that does X?'') another is to use such a program to learn more about the nature of such choices (a media theory concern: ``How does the construction of narrative choices shape player reactions?'').
%
Because of this, it also owes much to narratological theories which have nothing to do with computing, and to ideas related to the psychology of reading, decision-making, and persuasion.
%
The remainder of this chapter discusses these ideas that \dunyazad/ builds on in depth, starting with theories of narrative and moving on to computer systems that generate narrative.

\section{Narrative Theory}
%===============================================================================

\subsection{The Foundations of Narrative Theory}
%===============================================================================

Aristotle is one of the earliest scholars to discuss narrative theory, and his \work{Poetics} is a treatise on the inner workings of Greek drama \citep{Aristotle1917}.
%
Aristotle discussed which plot constructions were more effective than others, and identified basic types of dramatic plot, including the comedy and the tragedy, attempting to explain using examples what properties were important in these forms.
%
Although the idea that audience response is a simple function of the content of a narrative work has been recognized as na\"ive, the ideas of Aristotle sowed the seeds of modern narrative theory.
%
Along with other ancient critiques of drama, such as Horace's \work{Ars Poetica} \citep{Horace1783}, \work{Poetics} helped establish drama, and by extension narrative, as cites of critical study.


An important next step in for narrative theory happened in 1894, when Gustav Freytag published \work{Technique of the Drama} \citep{Freytag1894}.
%
Freytag's ideas about the emotional arc of traditional dramas have become standard fare in today's classrooms when students first begin learning how to unpack meaning in narrative.
%
His work on the dramatic structure of the traditional 5-act drama represents a further step from critique or craft advice towards more formal theories of drama (and more generally, narrative).


\subsection{Formalism and Structuralism}
%===============================================================================

In the early 20th century, this formalization of narrative theory was taken even farther by a group of Russian literary critics known as the Russian formalists.
%
These scholars were exemplified by Vladimir Propp, whose 1928 \work{Morphology of the Folk Tale} \citep{Propp1971} dissected common characters and scenes within a group of Russian folk-tales and come up with an abstracted representation which revealed fundamental similarities between the stories.
%
Propp observed that a simple grammar of a set of thirty-one basic `functions' could account for the plots of all of the stories that he was analysing, thus creating an abstract and formalist description of their common structure.
%
Although this formal structure has proved an attractive theory for computer scientists interested in creating stories, some contemporary literary theorists disagreed with Propp, asserting that Propp's analysis neglected many important aspects of a story, such as tone and mood.
%
In Prop's (and the formalists') attempts to understand similarities common to diverse narratives, he (and they) are forced to ignore many of the details that make individual stories stand out.
%
Although these details may be safely abstracted while still understanding the \emph{sequence of events} embedded in a particular story, they are often crucial to the \emph{audience's experience}, and thus to poetics.


The Russian formalists began a movement within literary criticism that gave rise to the discipline of narratology.
%
Rather than explicating the details of an individual work or the works of an author or movement, scholars such as Greimas and Barthes sought structural theories that could explain broad phenomenon in many narrative contexts \citep{Barthes1975,Greimas1988}.
%
Central to narratology is the idea that to some degree, narratives can be separated into an abstract series of events (the \textit{fabula}, or story) and a particular telling of those events (the \textit{syuzhet}, or discourse).
%
While work like Propp's focuses almost myopically on story at the expense of discourse, some later narratologists focused on discourse-related effects such as focalization.
%
Narratologists created theories and methods of analysis that could be applied to a broad range of narrative contexts.
%
Particularly because of the abstract nature of these theories and methods, computer scientists interested in creating artificial story generators have drawn on them as inspiration for their systems.


\subsection{Cognitive and Psychological Narrative Theories}
%===============================================================================

Although narratology grew out of literary criticism, in the late 20th century researchers in fields such as psychology and cognitive science began to approach narrative from a new perspective.
%
Psychologists were interested in questions such as ``Why do humans enjoy stories?'' or ``How do humans comprehend stories?''
%
They put forth models of narrative comprehension \citep{Kintsch1980,Brewer1982}, and asked questions about why people liked stories \citep{Iran-Nejad1987}.


In the early 1990's, this work accelerated, with work on suspense \citep{Gerrig1994}, narrative inferences \citep{Graesser1994}, and emotion \citep{Oatley1995}.
%
Models of comprehension continued to develop, including the event-indexing model of Zwaan, Langston, and Graesser \citep{Zwaan1995}, which has influenced several recent works in the field of narrative intelligence (see below).
%
This interest in narrative consumption as a psychological phenomenon was largely separate from literary criticism and theory, in large part because it dealt with much lower-level questions.
%
However, in the early 2000's, works like Palmer's \work{Fictional Minds} \citep{Palmer2004} and Zunshine's \work{Why We Read Fiction} \citep{Zunshine2006} began to connect literary theory with psychology.
%
Zunshine in particular draws on the work of psychologists like Oatley (e.g., \citep{Oatley1999}) to talk about how novels exercise our theory of mind when we read them.
%
Her work is at heart a work of literary criticism, however, and she uses critical analyses of several well-regarded novels as the basis of her argument.
%
\work{Why We Read Fiction} is an example of cross-disciplinary analysis: new theories from psychology inspired her to understand well-theorized works from a new perspective.
%
Zunshine has begun the work of combining psychological and critical perspectives on narrative, but of course there is still much unexplored overlapping territory between these two disciplines.


\subsection{Craft and Literary Criticism}
%===============================================================================

Although narratologists and psychologists have attempted to find a more `rigorous' and `objective' understanding of narrative, more traditional literary criticism and practical advice provides another source of knowledge about narrative.
%
Novelists, writing teachers, and critics (who often wear more than one of these hats) have write down their own views on narrative for both pedagogical and critical purposes.
%
Works like E. M. Forster's \work{Aspects of the Novel} \citep{Forster1927} offer insight into how producers and critics view narrative.
%
As someone who is acknowledged as skilled at the craft of narrative composition, Forster is in a unique place to observe and reveal underlying aspects common across different narrative contexts.


\work{Aspects of the Novel} and other works like it critically provide practical advice aimed at prospective authors attempting to \emph{create} effects such as foreshadowing or suspense.
%
It should not be surprising therefore that these resources are sometimes more useful than psychological analyses of the experience of these effects for researchers interested in automatically generating narrative.
%
Of course, craft and critical advice also exist for interactive works, as outlined below.


\section{Theories of Interaction}
%===============================================================================

Johan Huizinga's \work{Homo Ludens} is a seminal work on the history and culture of play which laid the foundation for the study of games as a medium \citep{Huizinga1949}.
%
Although Huizinga focused broadly on \emph{play} as a social phenomenon, his major philosophical points apply equally to games as a cultural medium.
%
Ideas like the transformation of norms afforded by the magic circle are just as applicable to the culture of modern videogames as they were to the historical play contexts that Huizinga studied.


Although interactive narratives are by no means a modern invention, the rise of available leisure time in combination with the advent of the computer drastically transformed them over the course of the 20th century.
%
The rise of media like tabletop role-playing games and wargames (documented in \citep{Peterson2012}) coincided with the development of gamebooks (inspired by the work of Borges \citep{Borges1956} and developed by writers like Queneau at the Oulipo, see e.g., \citep{Queneau1967}).
%
Both of these interactive media gained popularity in the 1960's and '70's which was also when the first electronic games began to reach mainstream popularity (the original arcade \work{Pong} appeared in 1972).
%
By the late '70's and into the early '80's, a commodity market for computer-enabled interactive fiction began to appear with titles like \work{Zork} ultimately selling hundreds of thousands of copies \citep{Zork}.


A fourth interrelated development during this period was the advent of hypertext, which also drew inspiration from Borges' \work{The Garden of Forking Paths}.
%
Developed for research rather than entertainment, hypertext nonetheless eventually attracted the interest of writers.
%
Compared to the authors of gamebooks, these writers were interested in the medium more as a new experimental form than as an extension of existing forms like tabletop roleplaying and adventure novels.
%
Because of this, authors like Michael Joyce and Shelley Jackson, who were sometimes also developers of hypertext systems, experimented extensively with their work \citep{Joyce1990, Jackson1995}.


\subsection{Theories of Interactive Fiction}
%===============================================================================

\subsubsection{Early Work}
%-------------------------------------------------------------------------------

Brenda Laurel's 1986 thesis titled \work{Toward the Design of a Computer-Based Interactive Fantasy System} is one of the earliest theoretical works that discusses computer-enabled interactive fiction \citep{Laurel1986}.
%
Laurel's thesis was a visionary description of a future medium in which computers would enable a drastically more interactive form of theater.
%
Along with other early work like Sean Smith and Joseph Bates' 1989 \work{Towards a Theory of Narrative for Interactive Fiction} \citep{Smith1989} Laurel's thesis was inspired by interactive fiction software and looked forward to potential future forms of interactive narrative.


Marie-Laure Ryan's \work{Possible Worlds, Artificial Intelligence, and Narrative Theory}, published in 1991, makes a slightly different connection between narrative and computers: it brings ideas from artificial intelligence (and philosophy) to bear on narratology \citep{Ryan1991}.
%
Ryan's work represents a link between traditional narratology and the study of interactive narratives.
%
Notably, she cites early computer story-generation systems such as \work{Tale-Spin} and \work{Universe} (discussed below).


\subsubsection{Foundations}
%-------------------------------------------------------------------------------

As computer games became more main-stream, work that theorized about interactive narrative as a medium proliferated. 
%
In 1997, Espen Aarseth published \work{Cybertext: Perspectives on Ergodic Literature} and Janet Murray published \work{Hamlet on the Holodeck: The Future of Narrative in Cyberspace} \citep{Aarseth1997,Murray1997}.
%
These two books are foundational to the study of electronic interactive fiction, and they both include significant theoretical components based on analysis of the interactive fiction of the preceding decades.


Aarseth's Cybertext connects the dots between gamebooks, hypertext, and interactive fiction.
%
In all of these forms, he sees ``ergodic literature:'' narratives which are not merely present to be observed, but which require work to traversed.
%
The theory of choice poetics presented here is concerned with one manifestation of this phenomenon: choices as a site of player work within the context of interactive stories.
%
By focusing on choices as a specific form of interaction, choice poetics hopes to understand how authors set up interactive fiction so that the work that players do contributes to the thematic qualities of the narrative players experience.
%
In this sense, choice poetics fits into the broader theory of ergodic literature as an in-depth study of a particular form of player-work in interactive narratives.


Murray's \work{Hamlet on the Holodeck} explores the concept of computer-driven interactive theater envisioned a decade prior by Laurel, but in the context of a vastly different ecosystem of computer games.
%
Between 1986, when Laurel published her thesis, and 1997, when Murray published her book, electronic entertainment had undergone radical changes, including the advent of three-dimensional graphics in games and mass-market personal gaming (in America, the Nintendo Entertainment System was released in 1985; the Nintendo 64 was released in 1995).
%
Murray devotes an entire section of her book to aesthetics, with chapters on immersion, agency, and transformation.
%
These concepts, especially agency, have become central in the study of electronic game aesthetics, and choice poetics fits within this framework as well as a study of one microcosm of interactive experience.
%
While Murray talks broadly about the pleasures and annoyances of interactive media, choice poetics is interested in investigating the detailed mechanisms by which choices as a unit of interaction give rise to specific feelings.
%
In a sense, choice poetics as a study of detailed mechanisms is only possible in a world in which authors like Aarseth and Murray have first explored the larger shape of the space of interactive narrative.


\subsubsection{Agency}
%-------------------------------------------------------------------------------

Drawing on the work of Murray and Laurel, Michael Mateas published \work{A Preliminary Poetics for Interactive Drama and Games} in 2001 \citep{Mateas2001}, uniting ideas from Aristotle's \work{Poetics} with Murray's discussion of agency to propose the idea of agency as a balance between formal and material affordances.
%
By this, Mateas meant that the player's sense of agency in an interactive experience depends on a balance between actions that the system allows them to do and agendas that the system encourages them to pursue.
%
According to this analysis, a game like \work{Quake} \citep{Quake} is a high-agency game because the actions available to the player (killing enemies and progressing through levels) are aligned with the agendas that the fictional world suggests (killing the otherworldly invaders and finding a way to end the invasion of Earth).
%
Although \work{Quake} may do a good job of providing agency, it only does so for players interested in pursuing a violent agenda, and few games in 2001 provided agency for players who were interested in social interactions.
%
Mateas suggests an architecture for an interactive drama which would provide agency in social interaction, and this project was eventually published as \facade/ \citep{Mateas2002b}.


This concept of agency was further developed in \citep{WardripFruin2009} and later extended by Stacey Mason in \citep{Mason2013} to distinguish between diegetic and extra-diegetic agency.
%
Seen as one of the qualities of interactive media that distinguishes them from books and movies, agency in various conceptions is an important topic in the theory of interactive narrative.
%
In a context where discrete choices are the only game mechanic, as in a Choose-Your-Own-Adventure book, any agency present can be ascribed to the choice structures in a work, so in some cases, agency is purely a product of choice poetics.
%
However, because agency as a phenomenon has been so thoroughly considered by other authors, the theory presented in \cref{ch:choice-poetics} does not focus on agency as the primary goal of choice construction, but instead sees it as one of many poetic effects.


\subsubsection{Operational Logics}
%-------------------------------------------------------------------------------

Another important concept in the theory of interactive media is that of operational logics, first proposed by Noah Wardrip-Fruin in 2005 \citep{WardripFruin2005} and described in detail in \citep{Mateas2009}.
%
An operational logic is a system of processes which work together to present a coherent domain of action or perception to a player.
%
For example, a resource logic might include mechanisms for buying, selling, trading, mining, and consuming resources within a game.
%
Each individual operation is backed by a distinct set of software routines which each enforce their own logic, but taken together, the player perceives resources as a coherent game element that can be manipulated in various ways.
%
Another example would be the two-dimensional graphical logics of platform games: on-screen representations move in certain ways, affected by gravity but stopping when in contact with a surface below.
%
A host of internal logic supports these behaviors, and together, they are \emph{perceived} as expressing a virtual environment which a character can traverse by running, jumping and falling.


As a theory focused on how system rules at a low level give rise to player perceptions and conceptions, the idea of operational logics is similar to the idea of choice poetics, which focuses on how choice structures encourage player emotions and reactions.
%
Although choice poetics is focused on the emotional rather than conceptual implications of choices, choice structures within a work can together form an operational logic, and the language of choice poetics can help understand this.
%
For example, if the player is consistently given options to travel north, south, east, or west in certain circumstances, that can be understood from the perspective of choice poetics as a recurring choice structure which presents neutral (and presumably usually mysterious) options, encouraging an exploratory mode of engagement (see \cref{ch:choice-poetics}).
%
At the same time, such regular choices together form an operational logic which presents to the player the concept of a navigable space, something that the same choices would not do if they only appeared once or twice throughout a work.
%
An analysis of choice structures from the perspective of operational logics can thus be helpful for an analysis of their poetics, and vice versa.


\subsubsection{Procedural Rhetoric}
%-------------------------------------------------------------------------------

Related to the concept of operational logics (how low-level system logics combine to form coherent representations) is the concept of procedural rhetoric: how interactive representations can further rhetorical goals.
%
Ian Bogost writes about procedural rhetoric in his 2008 article \work{The Rhetoric of Video Games} \citep{Bogost2008}, and Mike Treanor expands on the concept in his thesis \citep{Treanor2013}.
%
If procedural rhetoric is the study of how interactive artifacts can advance rhetorical goals through patterns of interaction, then choice poetics can be thought of as a microcosm of procedural \emph{poetics}.
%
Of course, rhetoric and poetics inevitably intermingle with each other: a rhetorical strategy such as an appeal to a shared ethical principle falls flat without poetics which reinforce the sense of community between the author and the reader.
%
If procedural rhetoric is concerned with how interactive experiences make meaning, procedural poetics would be concerned with how interactive experiences give rise to feelings, and choice poetics is a subset of that.


\subsubsection{Motives in Play}
%-------------------------------------------------------------------------------

In 1996, Richard Bartle wrote about different player types in multi-user dungeons (MUDs) \citep{Bartle1996}.
%
Bartle's work represents a vein of inquiry that focuses on players as active agents within interactive fiction, and has this in common with theories of reader response in literary criticism.
%
By attempting to categorize players according to their approach to a game, critics can better understand why different players might respond differently to a particular design element or in-game situation.
%
Building on this work in the context of massively multiplayer online role-playing games, Nick Yee published \work{Motivations for Play in Online Games} in 2006, analyzing play motives in a more recent multiplayer context \citep{Yee2006}.
%
Although these studies focused on games as social contexts, players of single-player games also exhibit a range of motivations, as shown in e.g., \citep{Kallio2011}.
%
Juho Hamari and Janne Tuunanen have published a broad meta-analysis of work on player types \citep{Hamari2014}, which identified five key dimensions of player motivation: achievement, exploration, sociability, domination, and immersion.
%
There has even been some work that explores alternative modes of engagement, such as Mary Flanagan's book \work{Critical Play}, which explores the notion of play as critique of both games and larger social structures \citep{Flanagan2009}.


Studies of player motivations are important for understanding choice poetics because players with different goals will respond to choice structures differently.
%
Choice poetics is interested in what particular aspects of play each player finds rewarding, in order to understand which outcomes of a choice a player will evaluate as positive or negative.
%
Any attempt to discuss choice poetics must acknowledge that each player's experience is ultimately subjective, and must therefore account for differences in option and outcome evaluations between players.

% TODO: THIS!
\subsubsection{The Player Experience}
%-------------------------------------------------------------------------------

Like work on motives for play, theories about the player experience center player behaviors rather than games.
%
These theories often develop out of media psychology or communication research instead of literary or film criticism.
%
Studies on specific effects like immersion \citep{Douglas2001,Ermi2005} and identification \citep{Klimmt2009} are relevant because players of choice-based narratives experience these effects, and because they may arise from both the gameplay and the narrative.
%
Additionally, studies of reasons for engaging with games (as distinct from reasons for specific behaviors \emph{within} gameplay) such as \citep{Ryan2006,Olson2008} provide insight into some of the unique pleasures that games have to offer, such as autonomy.
%
These studies often focus on differences between games and traditional media such as literature and film, and so choice-based narratives are an edge case in their analyses.
%
However, they also draw parallels between effects experienced by readers and players, and in a choice-based narrative, an effect like immersion could be supported by both elements.


One study that stands out in this category is Alex Mitchell's dissertation on re-reading in interactive fiction \citep{Mitchell2012}.
%
Focused on player experiences of re-reading in interactive narratives, Mitchell's work draws largely from player experiences with choice-based narratives, and helps untangle the complicated topic of re-reading when choice is present.
%
Although \dunyazad/ does not yet enable or anticipate re-reading, it is an important part of the average player's experience of modern interactive fiction (in the most obvious case through saving and loading in digital games), and thus the topic is an important one to choice poetics.

%\citep{Ryan2009} // poetics of intnar
%\citep{Bizzocchi2011} // close reading of games

\subsection{Hypertext Theory}
%===============================================================================

Parallel to the development of early theories about interactive narrative focused on electronic games, researchers and critics interested in hypertext developed their own theories about narratives with links.
%
This particular subset of interactive narrative developed its own strong genre conventions, but among digital narratives of the '80's and '90's, hypertext stories most resemble gamebooks in the sense that they consist of connected blocks of text.
%
If one views each virtual page of a hypertext as presenting the reader with a single choice, where each link on the page is an option, hypertexts can be analyzed in terms of choice poetics (although such an analysis is distorted, as this view of hypertext is not how readers usually approach it).
%
By the same token, hypertext can be used to implement a purely choice-based narrative such as a gamebook, and in fact most of \dunyazad/'s output formats work within a hypertext engine.


At the end of the '90's, several scholars published on the aesthetics and poetics of hypertext, including Mark Bernstein, Wendy Morgan, and Susana Tosca \citep{Bernstein1998,Morgan1999,Tosca1999,Tosca2000}.
%
Of particular interest to choice poetics are discussions of the suggestive nature of links, and of course, of the links-as-choices pattern.
%
In the context of hypertext literature, links are often ambiguous, connecting a word or phrase with an obtusely-related scene.
%
In fact, figuring out the significance of the relationship between linked nodes is often integral to reaching a full understanding of a work.
%
Because of this, hypertext scholars have talked at length about the potentials suggested by a link, and the process of following a link and resolving what one finds on the other side with one's original expectations (see e.g., \citep{Tosca2000}).
%
The same process happens at a discrete choice, although some aspects are magnified by the context: the player generally considers the implications of each option, compares them against one another, and then upon making a decision and seeing an outcome, has to reconcile that outcome with the option chosen and the other options available.


\subsection{Craft and Criticism of Interactive Narratives}
%===============================================================================

Absent a vibrant body of theoretical work, much current knowledge about choice poetics is contained in craft advice and criticism.
%
Although there may be no existing theory of how, for example, regret is created and functions in interactive narratives, authors of interactive fiction certainly know how to evoke regret and use it within plot structures.
%
Through practice and experience, a trial-and-error authoring process is honed into authorial instinct, and so when authors talk about their craft, theorists should listen.
%
Likewise, critics develop a sense for why certain stories are better than others at achieving poetic goals, and can describe in detail which elements of an interactive narrative contribute to or detract from specific effects.


Regarding interactive narrative, resources for the game masters of tabletop roleplaying games provide a wealth of information about how players respond to choices as well as how to manipulate them; see e.g., \citep{Laws2001}.
%
Even more relevant to \dunyazad/ as a project are blog posts on the Choice of Games website that talk about game design \citep{ChoiceOfGamesGameDesignCategory}.
%
Choice of Games is a company that publishes online gamebooks, and their authors and editors understandably have good advice to give about constructing choices.
%
Articles like ``5 Rules for Writing Interesting Choices in Multiple Choice Games'' give concrete advice about how to construct choices and in particular, highlight some constructions that are likely to sustain players' interest \citep{ChoiceOfGamesChoiceRules}.
%
Of course sources like these are tailored to specific writing styles and reflect authorial biases, but in some sense they represent known facts about choice poetics in their particular contexts, and are thus invaluable to a project attempting to lay the groundwork for a theory of choice poetics.


\subsection{The Psychology of Decisions}
%===============================================================================

Perhaps because digital games offer so many new interaction paradigms and are in fact constantly inventing new ways to tell stories, there are few scholarly resources dedicated to the study of choice-based narratives exclusively.
%
However, the subject of decisions in real-life behavior is of great interest to economists and psychologists, and in these contexts decision-making has received a great deal of attention.
%
A full review of the literature on the psychology of choices is beyond the scope of this work, but several studies that are key to choice poetics are worth mentioning.


First, Amos Tversky has studied framing and context with several co-authors including Daniel Kahneman \citep{Tversky1981} and Itamar Simonson \citep{Tversky1993}, famously showing that even when two options are identical from an economic perspective (the opportunity to prevent a loss of \$20 vs\@. the opportunity to gain \$20, for example) they can elicit very different reactions.
%
Because framing has such important effects on preference, it is necessarily important in the construction of choices designed to achieve poetic effects.
%
Tversky's research also demonstrated that the idea of independence from irrelevant alternatives doesn't hold up in many situations, meaning that the presence or absence of a non-chosen option can affect which option a person will choose in various ways.
%
This result is important for choice poetics because it implies that the poetics of a choice depend on the entire set of options presented, and suggests that analysis of a single option or a subset of options is insufficient to understand the full cognitive impact of a decision.
%
Because of this, systems which generate choices passively by independently generating several possible continuations of a plot and then letting a player choose between them will not be able to reason about the poetics of the choices they create: alternatives and the option text that introduces them must be considered simultaneously.


Context is important not only for options but also for outcomes.
%
Barry Schwartz, Andrew Ward, John Monterosso, Sonja Lyubomirsky, Katherine White, and Darrin Lehman have proposed a theory of `satisficing' and `maximizing' personalities, showing that some people feel regret unless they are confident that they selected the best option at a choice (maximizers), while others are happy as long as the outcome they got was good (satisficers).
%
This again implies that considering the full set of options at a choice is necessary for assessing it's poetic impact, a finding that is reinforced by the data presented in \cref{ch:option-results,ch:outcome-results}.
%
It also stresses that individual players will react differently to each choice, and authors must be aware of this.


Another psychological theory relevant to the results presented here is decision affect theory \citep{Mellers1997,Mellers1999}.
%
Decision affect theory broadly states that when faced with risky choices, humans make decisions that maximize their expected emotional reaction to outcomes (accounting for various cognitive biases) rather than decisions that maximize their expected overall payoff.
%
One of decision affect theory's specific predictions is that unexpected results will be perceived as more positive than expected results when they're positive overall, but will be perceived as more negative when they're negative (unexpectedness effectively amplifying the perception of goodness or badness) \citep{Shepperd2002}.
%
The results presented in this dissertation largely confirm this hypothesis, and the broad implication for choice poetics is again that the perception of outcomes will be colored by the structure of the choices that lead to them.


Although these various theories of decision making help inform choice poetics, they should not simply be taken at face value in the context of choice-based narratives.
%
Because player motivations are often different from people's motives in real-life situations, and because norms can be substantially altered in the context of playing a game, players may make in-game decisions differently than real-life decisions and may have different responses to their outcomes.
%
Studies of the psychology of everyday choices are thus only a starting point for the study of choices in interactive narratives.



\section{Computational Narrative Systems}
%===============================================================================

\subsection{The Foundations of Computational Narrative}
%-------------------------------------------------------------------------------

While the theories of interactive (and traditional) narrative just discussed mainly relate to the development of choice poetics, \dunyazad/ also includes a system-building component.
%
As a system designed to generate poetic choices, \dunyazad/ is part of a long history of computer systems designed to create (or help create) stories.
%
The earliest example of such a system appeared in 1971, in a technical report by Sheldon Klein, John Oakley, David Suurballe, and Robert Ziesemer titled \work{A Program for Generating Reports on the Status and History of Stochastically Modifiable Semantic Models of Arbitrary Universes} \citep{Klein1971}.
%
Two years later another publication about the project was simply titled \work{Automatic Novel Writing: A Status Report}, which made it a bit more clear what the aim was \citep{Klein1973}.
%
Klein and his collaborators created a computer representation of a mystery world that included some random elements, and thus generated different output each time it was run.
%
Although presented as an ``automatic novel writer'' most of the plot structure in their program's output was hard-coded in the form of events or actions scheduled to take place at particular times.
%
Still, as a first foray into narrative generation, their system was impressive, especially given the limitations of computer systems at that time.


It did not take long for the idea of an artificial author to receive more attention.
%
In 1976, James Meehan published his thesis titled \work{The Metanovel: Writing Stories by Computer} \citep{Meehan1976}.
%
Influenced by Roger Schank (his advisor at Yale), Meehan built a program that allowed autonomous characters to build and carry out plans.
%
Based on the idea that fables were stories about problem-solving, his program would place characters in problematic situations and then narrate their execution of a plan to resolve their problems.
%
Much more dynamic than the system build by Klein, \work{Tale-Spin} (as Meehan dubbed it) became an inspiration for future work on narrative generation.
%
Although \dunyazad/ is neither plan-based nor character-driven, the echoes of these ideas are still present in the character motivation system, and in its reliance on canned dramatic situations which drive dynamic character reactions.


Following Meehan at Yale, Natalie Dehn proposed a different model of story generation in 1981 \citep{Dehn1981}.
%
Her proposed system \work{Author} would simulate not the actions of characters, but instead the goals and plans of an author.
%
Dehn realized that for many plots, an explanation purely at the level of character motivations fails to account for many serendipitous events that make the work interesting.
%
However, if we appeal to an author who plans for things like suspense or foreshadowing, it is much easier to explain why events occur as they do.
%
Dehn accordingly planned a system that would simulate an author, thus inspiring an author-simulation-based model of narrative intelligence systems that remains popular today.


Following Dehn, Michael Lebowitz published \work{Creating a Story-Telling Universe} in 1983, proposing a story-generation system for generating soap-opera episodes called \work{Universe} \citep{Lebowitz1983}.
%
Like \work{Author}, \work{Universe} worked at the author level, but it focused more on character creation, and it used hierarchical planning to assemble plots \citep{Lebowitz1984,Lebowitz1985}.
%
Most of \work{Universe}'s generation activity consisted of picking plot fragments that elaborated on higher-level plot fragments.
%
For example, \work{Universe} might start with a high-level plot fragment such as ``Mary breaks up with John,'' and then decide to instantiate this using ``Mary and John have a fight and Mary leaves John.''
%
Both ``Mary and John have a fight,'' and ``Mary leaves John,'' might themselves be abstract plot fragments with multiple possible instantiations, and the choice of instantiations at any level might depend on the character attributes of Mary and John.
%
Lebowitz' \work{Universe} approach is thus radically different from the forward-planning approach of Meehan and Dehn, and in some ways can be compared to a grammar-based or template-based system.


As mentioned above, Brenda Laurel's dissertation on interactive theater was published in 1986 \citep{Laurel1986}, so around this time the idea of computer-generated narratives was supplemented by the idea of computer-mediate narratives.
%
\dunyazad/ ultimately sits between these two traditions, as it includes interactivity in the form of choices for the player to make, but focuses more on offline generation than on-line mediation.
%
Researchers have continued to develop non-interactive story generators, and two more landmark systems debuted in the 1990's: Scott Turner's \minstrel/ in 1993 and Rafael P\'{e}rez~y~P\'{e}rez' \work{Mexica} in 1999 \citep{Turner1993,PerezyPerez1999}.


\minstrel/ was in fact the impetus for the creation of \dunyazad/ , as explained in \cref{ch:skald}.
%
It used the author-centric approach of Dehn and Lebowitz but relied heavily on a mechanism called imaginative recall which worked to re-use bits of remembered stories from a story library as parts of a new story.
%
Although Turner does not use the term, it was in large part a case-based reasoning system for story generation.


\work{Mexica} took the author-centric approach and pushed it further, modelling not just authorial planning but also reflection \citep{PerezyPerez2001}.
%
After \work{Mexica} creates an episode, it reflects on it, revising in much the same way an author revises their work to create a finished product.
%
Although \dunyazad/ does not have a second-pass refinement step, nor does it explicitly simulate an author, it does devote much of its processing to reasoning about player perceptions.
%
In this regard it is more similar to a system like \work{Mexica} than it is to systems like \work{Tale-Spin} or \minstrel/ which do not explicitly reason about audience reception.


\subsection{Modern Computational Narrative}
%-------------------------------------------------------------------------------


\subsubsection{Story Generators}
%-------------------------------------------------------------------------------

Following early efforts in computer-enabled story generation, a wide range of methods have been applied to this challenge.
%
Methods including case-based reasoning (beyond \minstrel/) \citep{Gervas2005}, computational analogy \citep{Zhu2010}, evolutionary algorithms \citep{Bui2010}, and crowd-sourcing \citep{Li2013} have been explored.
%
One particularly prolific vein of research has been planning-based story generators, exemplified by the work of R. Michael Young and Mark Riedl, among others \citep{Young1999,Riedl2004}.
%
These systems work at the author level, but nonetheless use planning to decide which characters will carry out what actions in service of author goals.
%
This line of research has been able to accommodate formal models of various narrative effects as additional constraints at the planning level, including things like suspense \citep{Cheong2006}, foreshadowing \citep{Bae2008}, and character conflict \citep{Ware2014}.
%
These planning-based story generators share a predicate representation of the world with \dunyazad/, and to some degree exhibit a similar reasoning process, which is able to integrate relatively arbitrary declarative constraints into the creation of stories.
%
As is evident, this approach has provided fertile ground for experimentation with particular psychological and critical theories of narrative because these models can be integrated as declarative constraints on the output without worrying about designing a process that can reliably produce specific effects.
%
Of course, \dunyazad/ at the moment is focused on constructing individual choices, and so its capabilities for reasoning across multiple actions are much more limited than those of a planner.
%
Because of this, one promising direction for future work would be to integrate \dunyazad/ with a narrative planner, letting the planner specify choice structures at each choice point and later relying on \dunyazad/ to construct the specifics of each choice.
%
Such a setup would involve solving some hard problems, however, such as how \dunyazad/ and the planner would communicate about the actions and outcomes that correspond to each option at a choice.


\subsubsection{Interactive Narrative}
%-------------------------------------------------------------------------------

One of the earliest and most successful projects in interactive narrative was the Oz Project at Carnegie Mellon University.
%
Spearheaded by Joseph Bates (already mentioned in regards to theories of interactive narrative above, which also came out of this project), the Oz Project was a decade-long investigation into interactive fiction focused on stories supported by believable characters that could exhibit emotional reactions to a player's actions \citep{Bates1992}.
%
Using reactive planning architectures, the Oz project created virtual characters and then used these characters to tell stories.
%
The most successful system that resulted was Michael Mateas and Andrew Stern's \facade/ \citep{Mateas2002b}.
%
\facade/ was a true interactive drama that incorporated free-form text recognition input, procedural character animation, and a beat-based drama management system, putting players in the shoes of a visitor in the house of a pair of old friends as their marriage falls apart (or doesn't, depending on the player's actions).


\facade/ combines multiple approaches including character simulation and drama management into a single experience, but these ideas have been explored by other projects as well.
%
The idea of drama management as an AI technique for improving player experience in interactive fiction was first proposed by Peter Weyhrauch in 1997 \citep{Weyhrauch1997}.
%
\facade/ incorporated some of the ideas of drama management, and then in 2006 two projects proposed new approaches.
%
Mark Nelson, David Roberts, and Charles Isbell worked with Michael Mateas on declarative optimization-based drama management, a perspective that saw the drama-manager's intervention in an interactive experience as an optimization problem \citep{Nelson2006}.
%
At the same time, Bradford Mott and James Lester saw drama management as a decision problem, which they addressed using dynamic decision networks in \work{Crystal Island}, an educational interactive narrative \citep{Mott2006}.
%
Mott and Lester's work is notable because they used a dynamic model of the player to inform their system's decisions.
%
Later approaches to drama management have included case-based drama management \citep{Sharma2010}, and collaborative filtering for personalized drama management \citep{Yu2013}.
%
Even when not the focus of research, many interactive narrative projects include some system for managing and directing player actions.


Drama management usually takes the form of subtle manipulations of the world state that make certain actions more- or less-attractive (or simply unavailable) to the player in order to guide them towards a system-preferred path without taking away their agency entirely.
%
In this context \dunyazad/ could be a useful tool for drama management because it knows how to construct choices that not only evoke particular feelings but which also encourage the player to choose particular options.
%
The work of Hong Yu and Mark Riedl \citep{Yu2013} deserves further mention here because Yu and Riedl used Choose-Your-Own Adventure stories as the basis for their drama manager.
%
In order to combat player frustration, Yu and Riedl gathered data from many playthroughs of a digital version of a classic CYOA book and used player ratings to determine which decision paths led to more satisfying experiences.
%
At the same time, they developed alternate text for the options at each choice, and learned a model for which alternate option text was most enticing.
%
Then, when a new players interacted with their system, they could determine, based on the sequence of choices a player had made so far, which option similar players had found most rewarding in the path.
%
Using this desirability metric, they were further able to figure out which of their variant option texts would be most enticing for the desirable option (and least-enticing for the non-desirable options) and substitute those option texts, thus subtly influencing player choices.
%
In this manner, without ever changing the choices available to players, they were able to show that players influenced by their system reported more positive experiences overall.
%
Results like those of Yu and Riedl strongly motivate my work on \dunyazad/: If this kind of choice manipulation can be used to increase player satisfaction, a deeper understanding of choice poetics should be able to explain how they successfully influenced players, and a choice-generating system has an obvious use-case within interactive narrative.


Another important influence on \dunyazad/ comes from Nicolas Szilas' \work{IDTension} system \citep{Szilas2003,Szilas2007}.
%
Like Mott and Lester's \work{U-Director}, \work{IDTension} incorporates a formal model of the player in making judgements about story construction.
%
\work{IDTension} also uses predicates to represent its story states, and reasons over them using logical rules.

Its player model is used to score possible actions, and includes components for ethical consistency, motivation, conflict, and relevance.
%
Note that \work{IDTension}'s player model, like \dunyazad/'s, is static: it is a projection by the author about generic users, rather than an on-line measurement of an actual user.


The idea of player modelling has been taken further by other systems, including  David Thue, Vadim Bulitko, and Marcia Spetch's \work{PaSSAGE} \citep{Thue2008}, which tries to identify player preferences and select appropriate content in response to these.
%
Other work has built on this approach, but although \dunyazad/ could potentially make good use of on-line player modelling, it's current player model is static.
%
The problem of recognizing and categorizing player behavior is a difficult one, but \dunyazad/ could also enable a new approach: because \dunyazad/ deliberately constructs choices, once can imagine a setup where it creates some choices designed explicitly to discriminate between player categorizations, and uses these strategically to gain information about a player unobtrusively.
%
This potential synergy between online player-modelling techniques and deliberate choice construction is a promising direction of future work enabled by \dunyazad/.


One final interactive system with significant similarities to \dunyazad/ is Heather Barber and Daniel Kudenko's dilemma-based interactive fiction system \citep{Barber2007a,Barber2007b}.
%
Unlike most interactive narrative systems, Barber and Kudenko's project reasons explicitly about the choices it presents to the user: its main process for event generation is a planning system that continually tries to force the player's character into dilemmas.
%
Although Barber and Kudenko focus on only a single choice structure, they identify several specific variants of the dilemma (such as ``betrayal,'' which pits the character's needs against those of a friend, and ``favor,'' which positions the player as king-maker between two other characters).
%
The five subspecies of dilemma that Barber and Kudenko identify are used in their system as templates for story situations, but their analysis is also a micro-theory within choice poetics.
%
\dunyazad/ reasons at a slightly higher level, using `dilemma' as a unified label for one kind of choice that it can produce, but extending it to distinguish Barber and Kudenko's sub-categories would be possible.
%
Ultimately, the strategy of using dilemmas to drive narrative engagement is one that \dunyazad/ should enable, along with strategies involving other choice structures.


%\citep{Cavazza2002} // character-based
%\citep{Theune2003} // agent-based
%\citep{Aylett2005} // agent-based; emergent

%\citep{Hartmann2005} // interactive w/ Propp

%\citep{El-Nasr2007} // drama-theory-based

%\citep{McCoy2010,McCoy2012} // prom week

%\citep{Harrell2007} // poetry; conceptual blending

\subsubsection{Studies of Narrative Choices}
%-------------------------------------------------------------------------------

In the past several years there has been a movement within computational narrative circles towards attempts to reproduce specific affects within generated and/or interactive stories.
%
Several of these have focused on topics that bear on choice poetics, including the topic of agency.
%
In 2012, Matthew Fendt, Brent Harrison, Stephen Ware, Rogelio Cardona-Rivera, and David Roberts conducted a study where participants played through a simple CYOA-style game (albeit with only a few sentences per `page') and answered questions about their perceptions of agency afterwards \citep{Fendt2012}.
%
They found that techniques like explicitly acknowledging a player's choice were quite effective at increasing the perception of agency within their story, which reinforces the notion that the poetics of an interactive narrative can depend on textual details as much as abstract events.
%
Although the applicability of their results to full-scale interactive fiction is not clear, their research bears some resemblance to the experiments presented in \cref{ch:option-results,ch:outcome-results}.


Another similar study by Rogelio Cardona-Rivera, Justus Robertson, Stephen Ware, Brent Harrison, David Roberts, and R. Michael Young explored player perceptions of the divergence of outcomes \citep{Cardona-Rivera2014}.
%
Again using a simplified CYOA format, they found that when choices had clearly diverging outcomes, they were perceived as more influential than choices whose options indicated similar outcomes.
%
Of course, this might be the case because those choices were in fact more meaningful, but regardless, a link between divergence of outcomes and agency is established.
%
From the perspective of choice poetics, these studies can be seen as investigations into individual facets of specific poetic effects.
%
While agency is a fascinating and important effect, however, there are other effects, even some unique to interactive contexts (like regret) that are deserving of study, and results like this also need to be placed within a broader theoretical context.
%
Accordingly, one of the goals of \dunyazad/ as a project is to establish a broad context for choice poetics and lay down theoretical foundations within which specific effects like agency or regret can be talked about as parts of a larger space.


%\subsection{Other Related Systems}
%-------------------------------------------------------------------------------

%\citep{Montfort2009} // curveship

%\citep{Mitchell2012} // rereading

%\citep{Evans2014} // versu


\section{\dunyazad/'s Position}
%===============================================================================

As both a theoretical and practical project, \dunyazad/ inherits much from existing systems and theories.
%
It can been seen as a specialization or extension of several existing theories, including procedural rhetoric, operational logics, formalist narratology, link poetics, and theories of agency.
%
It also owes much to several related fields, such as the psychology of real-world decisions, the cognitive psychology of reading, and of course theories of interactive narrative.
%
As a technical effort, it grows out of rich traditions in both static narrative generation and interactive narrative, and it further draws inspiration from craft advice and critical resources directed at authors of both traditional and interactive narratives.


What distinguishes \dunyazad/ from related projects is first a focus on choices as a poetic form, and second, a commitment to simultaneous and mutually-informative theory and system development.
%
By privileging the explicit discrete choice as the object of study, a variety of interesting effects and interactions are brought into focus that do not necessarily seem important in the broader contexts such as `interactive narrative' or `hypertext.'
%
This focus also allows an investigation of specific mechanisms, taking a question like ``How can interactive narratives bias players' decisions?'' and reformulating it as ``What specific arrangements of framings, options, and outcomes can cause a player to feel that a choice is obvious?''
%
Furthermore, the development of a generative system which operationalizes the theoretical answer to produce obvious choices allows the theory to be ``field-tested,'' and experimental results can be used to directly inform the theory.
%
Although there are many systems and theories that resemble individual aspects of \dunyazad/, as the following chapters demonstrate, \dunyazad/ as a project is much more than the sum of its parts.
