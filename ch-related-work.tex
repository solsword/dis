\chapter{Related Work}

\label{ch:related-work}

\section{Classical Narrative Theory}
%===============================================================================

\cite{Aristotle1917}
\cite{Freytag1894}
\cite{Forster1927}
\cite{Propp1973}


\section{The Foundations of Computational Narrative}
%===============================================================================

\cite{Klein1971, Klein1973}
\cite{Meehan1976}
\cite{Lebowitz1984}


\section{Modern Narrative Theories}
%===============================================================================

\cite{Iran-Nejad1987}
\cite{Oatley1995, Oatley1999}
\cite{Mar2004, Mar2008}
\cite{Zunshine2006}


\section{Modern Computational Narrative}
%===============================================================================


\subsection{Interactive Narrative}
%-------------------------------------------------------------------------------

\cite{Laurel1986}
% TODO: Other IntNar theory authors...


\subsection{Constraint-Based Systems}
%-------------------------------------------------------------------------------


\subsection{Other Related Systems}
%-------------------------------------------------------------------------------


\section{Theories of Interaction}
%===============================================================================

\cite{Huizinga1949}
\cite{Bernstein1998}
% TODO: Other hypertext theory authors...


% TODO: Use this?
%\section{Related Work}
%
%This study is part of a recent trend focusing on choices in narrative, and several groups have published interesting results.
%%
%In 2011, Thue, Bulitko, Spetch, and Romanuik demonstrated \work{PaSSAGE}'s ability to increase player perceptions of agency by selecting content that players liked more \citep{Thue2011}.
%%
%This study demonstrated a link between desirable content and perceptions of agency.
%%
%The \work{PaSSAGE} system itself does directly construct choices, however, nor does it consider choice poetics in its operation.
%Instead, it uses online player modelling to determine what kinds of content a player prefers, and selects from pre-written quest alternatives based on that.
%
%
%Fendt, Harrison, Ware, Cardona-Rivera, and Roberts showed that direct feedback after players make a choice can create an illusion of agency, albeit in an extremely simple interactive narrative \citep{Fendt2012}.
%%
%In this study, the choices were part of a hand-written adventure game, whereas \dunyazad/ generates choices on its own.
%%
%This study concentrated on how different choice structures (mainly in terms of outcome text) affected players' perceptions of agency.
%
%
%In a similar study, Cardona-Rivera, Robertson, Ware, Harrison, Roberts, and Young found that choices where the options lead to significantly divergent outcomes increase feelings of agency relative to choices with similar outcomes \citep{Cardona-Rivera2014}.
%%
%Again this study used hand-written content as the test material; it focused on the connections between agency and divergence of outcomes.
%
%
%In another study using the Choose-Your-Own-Adventure format, Yu and Riedl were able to predict and influence players' choices using collaborative filtering \citep{Yu2013}.
%%
%Like the Fendt et al. and Cardona-Rivera et al. studies, this study used hand-written text, but this study chose to adapt two existing Choose-Your-Own-Adventure books for their study, rather than creating their own stories.
%%
%This resulted in much more complex and sophisticated stories, which more closely mimics the actual conditions of entertainment consumption.
%%
%Yu and Riedl's system did not generate options, however, it simply selected which options to present at a choice out of a number of hand-written alternatives that conveyed different motives for choosing a particular path.
%%
%This is a form of choice construction (since the exact options are dynamic) but the framing and outcomes are not dynamic and the options themselves are not constructed by the system but merely selected.
%
%Nevertheless, Yu and Riedl succeeded in using collaborative filtering to predict which of several motivations for taking the same action would be most appealing to players, and they were able to use a drama manager to guide players' choices.
%%
%Although their system only focused on guiding each player towards content that the collaborative filtering predicted that player would enjoy, one could imagine a more complex version that might try to influence players' perceptions of the choices themselves.
%%
%In contrast to Yu and Riedl's system, \dunyazad/ uses a logical model of player expectations that includes authorial input. 
%%
%While collaborative filtering allows a system to make user-specific adjustments and potentially allows fine-grained discrimination of player preferences, it is also quite opaque.
%%
%\dunyazad/'s logical model, while not adaptive, is more useful for informing a theory of choice poetics, and as our results demonstrate, it gets the job done.
%%
%That said, the use of online player data is something highly desirable for a system like \dunyazad/, and implementing something like Yu and Riedl's collaborative filtering is a tempting direction for future work.
%
%
%These studies are mostly focused on a single particular aspect of choice poetics (such as agency) whereas \dunyazad/ attempts to address constructing poetic choices more broadly.
%%
%Additionally, none of the studies mentioned thus far involved narrative generation systems that construct stories from raw material.
%%
%Considering constructive systems that are similar to \dunyazad/, Szilas' \work{IDtension} and El-Nasr's \work{Mirage} both stand out as interactive narrative systems rooted in theories about how narrative works \citep{Szilas2003,El-Nasr2007}.
%%
%\work{IDtension} is based on traditional dramatic theory (i.e., normal poetics) whereas \work{Mirage} incorporates theories of drama in the theatrical sense.
%%
%\dunyazad/ is also guided by a narrative theory: it is based on choice poetics--the theory of how choices are perceived by an audience \citep{Mawhorter2014}.
%%
%In this paper, we focus on \dunyazad/'s ability to generate individual choices that manipulate player expectations--obvious choices, relaxed choices, and dilemmas.
%%
%In some ways, this is reminiscent of recent narrative generation systems that have focused on specific traditional poetic effects, such as \work{Suspenser} (focused on suspense) and \work{Prevoyant} (focused on foreshadowing) \citep{Cheong2006,Bae2008}.


\section{}
