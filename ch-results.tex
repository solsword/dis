\chapter{Experimental Results}

\label{ch:results}

\Footnotetext{*}{Note: parts of this chapter have appeared in abridged form in \citep{Mawhorter2015b}.}

The goal of this chapter is to demonstrate \dunyazad/'s ability to manage player reactions by generating distinct choice structures.
%
Because it uses answer set programming, \dunyazad/'s choice generation system not only generates choices, but itself constitutes a theory of choice poetics.
%
Survey data presented here thus not only validate that players' perceptions match \dunyazad/'s intent, but also inform the theory of choice poetics.
%
Statistical analysis of the data indicates that \dunyazad/ is largely successful in its goals, but also reveals some places where either the code, the theory, or both can be improved. 
%
These unexpected results are in fact one of the larger goals of \dunyazad/ as a project: by operationalizing choice poetics, \dunyazad/ enables experiments which can reveal details that wouldn't be obvious from simply observing human-authored choices, because as a computer program, \dunyazad/ makes \emph{inhuman} mistakes.


The results presented here are of course limited by \dunyazad/'s specificity: \dunyazad/ has a particular approach for creating e.g., `obvious' choices and information about properties of it's `obvious' choices doesn't necessarily generalize to all obvious choices.
%
Largely, however, the extra considerations that this data suggests should be taken into account \emph{do} generalize, because they apply to any analysis or generation scheme which uses a particular subset of \dunyazad/'s constraints.
%
For example, one result that will be discussed suggests that when trying to figure out how players will evaluate different options, analysis in terms of absolute values is insufficient.
%
This result (which is unsurprising, as it echoes psychological research on real-life choices) clearly isn't something that's likely to be limited to just the particular choices that \dunyazad/ generates (in terms of genre or any other factor).
%
There might be some cases where humans do use only absolute value judgements, but until specific conditions under which this is true are found, it's safer to assume that both absolute and relative value judgements between options should be accounted for.
%
In any case, the discovery prompted by \dunyazad/ that absolute value judgements are insufficient for understanding choice poetics generalizes beyond the specifics of \dunyazad/'s choices.


The two experiments presented here were set up in order to test \dunyazad/'s functionality and thereby also inform the theory of choice poetics that it is based on.
%
The first is mainly concerned with prospective impressions, which are the result of the ``relative option analysis'' step in the goal-based choice analysis method described in \cref{sec:goal-based-choice-analysis}.
%
In \dunyazad/, these are represented using \prq{option\_feel}{} predicates, as described in \cref{sec:dunyazad-poetic-constraints}.
%
After analyzing the results of the first experiment, I conducted a second experiment focused on retrospective impressions, which are the results of retrospective analysis and are represented by \prq{outcome\_feel}{} predicates.
%
\Cref{sec:exp-prospective} describes the details of the first experiment, along with much of the details of the experimental setup that are common to both experiments, while \cref{sec:exp-retrospective} describes the second experiment.


\section{Experiment I: Prospective Impressions}

\label{sec:exp-prospective}

To exercise \dunyazad/'s prospective impressions system, I set up \dunyazad/ to construct three different types of choice:
%
\begin{itemize}
  \item \emph{Relaxed} choices, where the stakes were low and there were no bad options.
  \item \emph{Obvious} choices, where there was a single option that stood out as more advantageous than the rest.
  \item \emph{Dilemmas}, where every option was about equally undesirable.
\end{itemize}
%
These three prospective impressions cover positive and negative option impressions, low- and high-stakes choices, and contrasting and similar option sets, so they exercise every dimension of player expectations that \dunyazad/ attempts to model.
%
Of course, the full set of \dunyazad/'s prospective impressions (shown in \cref{tab:prospective-impressions}) is broader, and there are plenty of prospective impressions that \dunyazad/ does not include definitions for, but relaxed, obvious, and dilemma choices were chosen as representative for this experiment.
%
Note that factors which give rise to ``obviousness'' or ``being a dilemma'' are quite a bit simpler than, say, factors that make a player feel regret.
%
These experiments and indeed work on \dunyazad/ so far has focused on these simple effects because if they can't be produced reliably, more complex effects are unlikely to work either.
%
The results presented here are thus only the first step in \dunyazad/'s use as a tool for exploring choice poetics.


After constructing choices, I ran a survey that asked participants to read a single choice generated by the system and answer some questions about their perception of the choice.
%
I analyzed the responses across choice categories and compared them against a uniform distribution to determine if players' perceptions match what the system intended.
%
My data show that \dunyazad/ was mostly able to produce the desired prospective impressions, but in a few specific cases there were surprising results.
%
Because \dunyazad/ is a transparent operationalization of choice poetic theory, both the expected and surprising results can usefully inform not only the system's construction but also the theory of choice poetics.


\subsection{Method}

\label{sec:e1-method}

The primary goal of this experiment was to assess \dunyazad/'s ability to manage and predict player's prospective impressions, i.e., perceptions of a choice before making a decision.
%
To that end, the experiment focuses on players' perceptions of options at a choice, and does not even present outcomes to the participants at all.
%
The three choice types that were generated were chosen because they are each distinct in terms of the player expectations they engender, and because together they exercise \dunyazad/'s capacity to reason about stakes, positive and negative indicators, and both contrasting and similar options.


To gather data on player expectations, I generated choices using \dunyazad/, showed them to study participants, and asked participants a series of questions about specific qualities of the choice they just read.
%
Because I used Amazon Mechanical Turk to gather data, my participants were each paid a small amount and presumably approached the survey as a means to earn money rather than as a voluntary undertaking.
%
Because of this, questions were asked in a hypothetical manner (e.g. ``If you were reading this story, which option would you choose?'') rather than directly (e.g. by having participants pick an option) to imply that the task at hand was asking them to judge the choice as someone reading it for entertainment might.
%
Of course, this framing (and being asked specific questions in general) might encourage an analytical mode of engagement, which is not what \dunyazad/ is designed to support, but that limitation is to some degree inevitable when survey responses are solicited.


To control for participants paying little attention, being unfamiliar with English, or simply filling in random responses, two check questions were asked.
%
Responses from participants who failed to answer these questions satisfactorily were excluded from the analysis, as were responses where one or more questions were left blank (about 15\% of all participants).

\subsubsection{Treatments}

For this experiment, there were three experimental treatments, each corresponding to a different set of rules used by \dunyazad/ to generate the choice experienced by a subject.
%
These are the ``obvious,'' ``relaxed,'' and ``dilemma'' choice types described above (see also \cref{tab:prospective-impressions} for formal definitions of these as \dunyazad/ perceives them).
%
The system definitions given in that section represent extra constraints placed on the choices generated by \dunyazad/ beyond its common core rules.
%
Each participant thus saw a choice shaped by one of three different constraint sets.


Of course, each constraint set can generate a potentially large range of specific choices, but this study was interested in perceptions common across choices generated using the same constraints.
%
One possibility would be to show each participant a unique choice from the space of choices possible given one of the treatment conditions.
%
However, this setup would mean that no single choice would be seen by more than one participant, and so there would be no way to analyze the contribution of individual choices to the perception of the different treatments.
%
Instead, I generated three different choices per treatment, and showed each choice to ten participants, for a total of 30 participants per treatment pre-attrition.

\begin{figure}[!h]
\quotebox{
  \slshape
You come to a tavern and decide to rest for a while. 
%
A merchant is bored and a noble is bored and an innkeeper seems knowledgeable. 
%
What do you do?
\begin{enumerate}[itemsep=0pt,topsep=4pt,parsep=0pt,partopsep=0pt]
\item You play a song for the noble \\
  (You have skill: musician. You have no tool for music).
\item You gossip with the innkeeper \\
  (You are missing skill: negotiation).
\item You play a song for the merchant \\
  (You have skill: musician. You have no tool for music).
\end{enumerate}
}
  \caption{An example choice.}
  \label{fig:results-exchoice}
\end{figure}

\subsubsection{Setup}

To set up the experiment, I used \dunyazad/ in its ``experiment'' mode (which causes it to generate only a single choice and to use a special framing) to generate three choices for each of the experimental conditions.
%
As an additional constraint, each choice was required to have exactly three options, so that the number of options wasn't a confounding factor in the data.
%
These nine choices were generated sequentially by the system, so there was no opportunity to cherry-pick ``good'' examples of each treatment category.
%
The choice shown in \cref{fig:results-exchoice} is the first choice that was generated; it is in the ``dilemma'' treatment.
%
The framing for each choice presented the skills that the system had assigned the player character for that choice, and established a basic context for the choice (see \cref{fig:exframing}).
%
The framing for each choice differed only in the skills presented and the fictional destination name, which \dunyazad/ chooses randomly from a fixed list of made-up names.


\begin{figure}[!h]
\quotebox{
  \slshape
You are about to set out on an epic journey. You are are heading towards towards the distant country of Jyv\"asky, hoping to earn fame and fortune.
%
You have some perfume and a book of legends, and you have skill: literacy, you have skill: musician, and you have skill: healing. 
%
Eager to be on your way, you set off on the road towards Jyv\"asky.
}
  \caption{Example framing.}
  \label{fig:exframing}
\end{figure}


Once the choices were generated, their text was broken into parts and put into a comma-separated values file for upload to Amazon Mechanical Turk where the parts would be substituted into a template.
%
Given a survey template, Mechanical Turk generated individual survey pages for each question, and ten tasks were posted per question (a total of 90 tasks), which workers on Mechanical Turk were able to preview, accept, and fill out for payment (50 cents per task\footnote{This was chosen based on online advice to pay about minimum wage for tasks on Mechanical Turk. Given the median response time, the hourly pay rate was \$7.50, but in retrospect, the inconvenience of survey tasks (where a worker cannot complete the same task many times and thus work more efficiently) suggests that a higher pay rate would be appropriate. For this reason (and because the second survey included more questions) I used a higher pay rate for the second experiment. The total cost (about \$50 in this case counting Amazon's 10\% fee) was quite cheap.}).
%
Myle Ott's ``uniqueturker'' script (\url{https://uniqueturker.myleott.com/}) was used to ensure that no individual worker filled out the survey more than once.
%
To avoid being targeted by bots, the tasks required workers with a 97\% acceptance rate across at least 1000 accepted tasks.

\subsubsection{Survey Content}

Each survey was divided into three sections.
%
The first section ``Preliminaries'' began with a prompt that read:
\begin{quote}
  \slshape
To provide useful data for this survey, you must be at least 18 years old and able to read English. To confirm this (and to confirm that you aren't a bot), please answer the following question.
\end{quote}
%
This section just contained the following question designed to ensure that subjects were at least 18 years old and had basic English proficiency:
%
\begin{quote}
  \slshape
If you're at least 18 years old, please don't write ``Age of years eighteen least at am I that confirm I,'' as the answer here, instead write that sentence backwards, ended with an exclamation point. If not, please do a different HIT, as I cannot use your data in my results, and thus I will not accept your response.
\end{quote}


The second section of the survey was titled ``The Choice'' and began with a prompt:
%
\begin{quote}
  \slshape
Please read the following short story which presents you with a choice, then answer the questions about that choice below.
\end{quote}
%
In this section, each survey displayed one of the nine generated choices, followed by a single multiple-choice question:
%
\begin{quote}
  \slshape
If you were reading this story, which option would you pick?
\end{quote}
%
This question had three options: ``Option 1,'' ``Option 2,'' and ``Option 3.''


The final section of the survey was titled ``Opinion Questions'' and began with a prompt:
%
\begin{quote}
  \slshape
Please rate your agreement with the following statements, from 1 (strongly disagree) to 5 (strongly agree).
\end{quote}
%
This section contained the following 8 Likert items in this fixed order\footnote{These were individual Likert items which did not compose a Likert scale, as the goal of the survey was to directly measure opinions, and there were no underlying psychological variables presumed to be giving rise to behavior.} (the quotes were part of the survey):
%
\begin{enumerate}
    \slshape
    \setstretch{1.2}
  \item ``There are no bad options at this choice.''
  \item ``There is a clear best option at this choice.''
  \item ``The stakes for this choice are low.''
  \item ``There are no good options at this choice.''
  \item ``All of the options at this choice are about equally promising.''
  \item ``There are options at this choice.'' (This is a trick question to test whether you're paying attention. Please simply indicate that you are in complete disagreement.)
  \item ``This is a difficult choice to make.''
  \item ``This choice feels like it will have important consequences.''
\end{enumerate}
%
Each question in this section was followed by the same five numbered options presented vertically:
%
\begin{enumerate}
    \slshape
    \setstretch{1.2}
  \item strongly disagree
  \item somewhat disagree
  \item neutral
  \item somewhat agree
  \item strongly agree
\end{enumerate}
%
For these questions (and the multiple-choice question in the previous section) participants selected an option by clicking a radio button next to that option. There were no default responses, so someone who didn't click any of the radio buttons would submit a blank response for that question.
%
Responses were treated as ordinal data, and labeled with the numbers 1 through 5 in the same order they appeared here (i.e. 1 $\rightarrow$ strongly disagree; 5 $\rightarrow$ strongly agree).

\subsection{Hypotheses}

Before conducting the survey, I came up with a set of initial hypotheses about how participants would answer these questions based on the treatment conditions.
%
There were three types of hypothesis: single-treatment hypotheses, between-treatment hypotheses, and stakes hypotheses.
%
Each singe-treatment hypothesis posited that under a particular treatment, respondents would generally agree with or disagree with a particular question.
%
The single-treatment hypotheses used are listed in \cref{tab:e1-single-treatment-hypotheses}.
%
Agreement with a question was determined by a one-sided Mann-Whitney-Wilcoxon U test \citep{Mann1947,Wilcoxon1945} against a uniform distribution%
\footnote{%
Another possible standard for comparison would be an all-neutral distribution.
%
Comparison against a uniform distribution helps demonstrate that the answers aren't random, however, and the unsupervised nature of the study meant that participants might have answered some questions at random in order to complete it quickly (although there were some other guards against this).
%
To verify that a uniform distribution was an acceptable standard I ran the same hypothesis checks against an all-neutral distribution and found that support for each was equal to or greater than support when comparing to the uniform distribution used.
}%
\ of responses with the alternate hypothesis being ``The median of the survey responses is significantly higher than the median of the uniform distribution.''
%
Disagreement likewise used a Mann-Whitney-Wilcoxon U test with the alternate hypothesis that the median of the survey responses was smaller than that of a uniform distribution.
%
In both cases, a confirmation of a hypothesis was taken to be significant for $p < 0.05$.


\begin{table}[!h]
\begin{center}
\bgroup
\def\arraystretch{1.3}
\begin{tabular}{l|c c c}
Question & Obvious & Relaxed & Dilemma \\
\hline
\eIqIshort/    &     -    &  agree   & disagree \\
\eIqIIshort/   &  agree   &     -    & disagree \\
\eIqIIIshort/  &     -    &  agree   &     -    \\
\eIqIVshort/   & disagree & disagree &  agree   \\
\eIqVshort/    & disagree &     -    &  agree   \\
\eIqVIshort/   &     -    &     -    &     -    \\
\eIqVIIshort/  & disagree &     -    &  agree   \\
\eIqVIIIshort/ &     -    &     -    &  agree   \\
\end{tabular}
\egroup
\end{center}
  \caption{Prospective single-treatment hypotheses by treatment}
  \label{tab:e1-single-treatment-hypotheses}
\end{table}


For the between-treatment hypotheses, survey data from two different treatments were compared using a Mann-Whitney-Wilcoxon U test to test whether one was statistically more-agreed-with than another (with the threshold for significance again being set at $p < 0.05$).
%
The statistics are the same for the converse cases, so only one test was performed per hypothesis (i.e., if responses to question 3 showed significantly more agreement for the ``obvious'' treatment than for the ``relaxed'' treatment, it follows automatically that they show significantly less agreement for the ``relaxed'' treatment than for the ``obvious'' treatment).
%
\Cref{tab:e1-between-treatment-hypotheses} lists the between-treatment hypotheses.
%
Across all 3 treatments, there were a total of 13 single-treatment and 9 between-treatment hypotheses.


%\begin{table}
%\bgroup
%\def\arraystretch{1.1}
%\begin{tabular}{l | c c c}
%Question & Obvious & Relaxed & Dilemma \\
%\hline
%\eIqIshort/    &      -     & $>$dil. &      -     \\
%\eIqIIshort/   & $>$dil. &      -     &      -     \\
%\eIqIIIshort/  &      -     &      -     &      -     \\
%\eIqIVshort/   &      -     &      -     & $>$both \\
%\eIqVshort/    &      -     &      -     & $>$obv. \\
%\eIqVIshort/   &      -     &      -     &      -     \\
%\eIqVIIshort/  &      -     &      -     & $>$both \\
%\eIqVIIIshort/ &      -     &      -     & $>$both \\
%\end{tabular}
%\egroup
%  \caption[Prospective between-treatment hypotheses by treatment]{Prospective between-treatment hypotheses by treatment The ``$>$'' signs indicate more agreement relative to the alternate treatment indicated by abbreviation. ``$>$both'' indicates that that treatment is hypothesized to be more-agreed-with than both other treatments individually (two separate hypotheses).}
%  \label{tab:between-treatments}
%\end{table}

\begin{table}
\bgroup
\def\arraystretch{1.3}
\begin{tabular}{l | c}
Question & Hypotheses \\
\hline
\eIqIshort/    & relaxed $>$ dilemma \\
\eIqIIshort/   & obvious $>$ dilemma \\
\eIqIIIshort/  & - \\
\eIqIVshort/   & dilemma $>$ obvious \& relaxed \\
\eIqVshort/    & dilemma $>$ obvious \\
\eIqVIshort/   & - \\
\eIqVIIshort/  & dilemma $>$ obvious \& relaxed \\
\eIqVIIIshort/ & dilemma $>$ obvious \& relaxed \\
\end{tabular}
\egroup
\caption[Prospective between-treatment hypotheses by treatment]{Prospective between-treatment hypotheses by treatment. The ``$>$'' signs indicate more agreement relative to an alternate treatment. ``$>$ obvious \& relaxed'' indicates that that treatment is hypothesized to be more-agreed-with than both of those treatments individually (two separate hypotheses).}
  \label{tab:e1-between-treatment-hypotheses}
\end{table}


The stakes hypotheses were simpler: across all treatments, participants who were shown a choice identified by the system as low-stakes should agree with the statement ``The stakes for this choice are low.''
%
Additionally, participants shown high-stakes choices were expected to disagree with that statement.
%
Both of these hypotheses were validated using Mann-Whitney-Wilcoxon U tests against uniform distributions as above.
%
A fall-back hypothesis for stakes was that participants who saw low-stakes choices would agree more strongly with that statement than participants who saw high-stakes choices (a between-cases hypothesis).


\subsection{Results}

\label{sec:e1-results}

Before processing the data from Amazon Mechanical Turk, responses that showed signs of inattentiveness, non-proficiency with English, or excess haste were filtered out.
%
In fact, as responses were being submitted, responses where either the age question was left blank or where the trick question (question 6) had an answer other than ``1 - strongly disagree'' or ``5 - strongly agree'' were rejected, meaning that Mechanical Turk would not pay the responder and the task would be reposted.
%
Including 6 rejected responses, a total of 96 responses were gathered, with 30 non-rejected responses to each treatment (10 per question).
%
Of the responses which remained, further filtering was performed:
%
\begin{itemize}
  \item Responses where the answer to question 6 wasn't ``1 - strongly disagree'' were dropped. These indicate a responder who isn't paying close attention to the survey.
  \item Responses where the total time spent on the task was less than or equal to 90 seconds were dropped. Answering all 8 questions in just 90 seconds is probably not possible if each question is given reasonable consideration. The median response time was 240 seconds (4:00) before filtering and 280 seconds (4:40) after filtering, although these times are simply the time between a participant accepting and submitting a task; they don't necessarily work on the task exclusively during that time, and in general accepting several tasks before doing them is a common pattern of behavior.
  \item Responses where any question was left blank were dropped. Given the ``neutral'' response option and the content of the survey, leaving a question blank is more likely a sign of lack of attention than of hesitation to answer.
\end{itemize}
%
After these filtering steps, a total of 79 valid responses remained, with 25 responses to the ``obvious'' treatment and 27 responses each to the ``relaxed'' and ``dilemma'' treatments.
%
Given these final response counts, I used a uniform distribution of 25 data points to test my single-treatment hypotheses, as that was the closest multiple of 5 (the number of response options) to the sizes of my data sets.
%
The low- and high-stakes conditions had 44 and 35 responses respectively; I used the same 25-point uniform distribution to test my stakes hypotheses.


\begin{figure}[!p]
  \centering
  \includegraphics[width=\textwidth]{fig/combined-report-cropped.pdf}
  \caption[Prospective data summary]{A summary of the data by question plotted as percentages of respondents per treatment who gave each possible answer following \citep{Robbins2011}. Responses range from 1 (strongly disagree) to 5 (strongly agree) with 3 being ``neutral.'' Disagreeing responses are plotted to the left, and agreeing responses are plotted to the right.}
  \label{fig:e1-report}
\end{figure}

%\begin{table}[!p]
%  \centering
%\bgroup
%\def\arraystretch{1.3}
%%\setlength{\tabcolsep}{0.7em}
%%\begin{tabular}{l | p{0.5em} | p{2.2em} | p{1.8em} | p{0.5em} | p{2.2em} | p{1.8em} | p{0.5em} | p{2.2em} | p{1.8em}}
%\begin{tabular}{l | c|c|c | c|c|c | c|c|c}
%Question      & \multicolumn{3}{|c|}{Obvious} & \multicolumn{3}{|c|}{Relaxed} & \multicolumn{3}{|c}{Dilemma} \\
%\hline
%\eIqIabbr/    & \tenp & \tensig{A}{0.176} & \tesig{D}{0.009}{66\%} \\
%\hline
%\eIqIIabbr/   & \tesig{A}{0.023}{64\%}  & \tenp & \tesig{D}{0.047}{62\%} \\
%\hline
%\eIqIIIabbr/  & \tenp & \tesig{A}{0.015}{65\%}  & \tenp \\
%\hline           
%\eIqIVabbr/   & \tesig{D}{0.006}{68\%}  & \tesig{D}{0.005}{68\%}  & \tesig{A}{0.007}{67\%} \\
%\hline
%\eIqVabbr/    & \tensig{D}{0.069} & \tenp & \tensig{A}{0.322} \\
%\hline
%\hline
%\eIqVIIabbr/  & \tensig{D}{0.073} & \tenp & \tesig{A}{0.009}{66\%} \\
%\hline
%\eIqVIIIabbr/ & \tenp & \tenp & \tesig{A}{0.001}{71\%} \\
%\end{tabular}
%\egroup
%  \caption[Prospective experiment single-treatment results]{%
%Single-treatment results.
%%
%Each entry has a letter indicating the hypothesis (agree (A) or disagree (D)) followed by the $p$-value for that test.
%%
%Significant entries ($p < 0.05$) are listed in bold, and indicate a common-language effect size (the percentage of comparisons supporting the hypothesis).
%%
%Non-significant entries are highlighted in \nsighcolor/.
%%
%Note that question 6 (the trick question) is omitted here.
%}
%  \label{tab:e1-single-results}
%\end{table}

\begin{table}[!p]
\centering
{
\def\arraystretch{1.3}
\setlength{\tabcolsep}{0.4em}
\begin{tabular}{l  c  c c c  c  c c c  c  c c c}
\toprule
Question &%
 & \multicolumn{3}{c}{\eIobviousabbr/} &%
 & \multicolumn{3}{c}{\eIrelaxedabbr/} &%
 & \multicolumn{3}{c}{\eIdilemmaabbr/} \\
\midrule
\eInobadabbr/ &%
 & \tenp  &%
 & \tensig{A}{0.176} &%
 & \tesig{D}{0.009}{69\%} \\
\eIclearbestabbr/ &%
 & \tesig{A}{0.023}{66\%} &%
 & \tenp  &%
 & \tesig{D}{0.047}{63\%} \\
\eIlowstakesabbr/ &%
 & \tenp  &%
 & \tesig{A}{0.015}{67\%} &%
 & \tenp  \\
\eInogoodabbr/ &%
 & \tesig{D}{0.006}{70\%} &%
 & \tesig{D}{0.005}{70\%} &%
 & \tesig{A}{0.007}{69\%} \\
\eIbalancedabbr/ &%
 & \tensig{D}{0.069} &%
 & \tenp  &%
 & \tensig{A}{0.322} \\
\eIdifficultabbr/ &%
 & \tensig{D}{0.073} &%
 & \tenp  &%
 & \tesig{A}{0.009}{69\%} \\
\eIconsequencesabbr/ &%
 & \tenp  &%
 & \tenp  &%
 & \tesig{A}{0.001}{73\%} \\
\bottomrule
\end{tabular}

}
  \caption[Prospective experiment single-treatment results]{%
Single-treatment results.
%
Each entry has a letter indicating the hypothesis (agree (A) or disagree (D)) followed by the $p$-value for that test.
%
Significant entries ($p < 0.05$) are listed in bold, and indicate a common-language effect size (the percentage of comparisons supporting the hypothesis).
%
Non-significant entries are highlighted in \nsighcolor/.
%
Note that question 6 (the trick question) is omitted here.
}
  \label{tab:e1-single-results}
\end{table}


%\begin{table}[!p]
%\centering
%\bgroup
%\def\arraystretch{1.3}
%\setlength{\tabcolsep}{0.7em}
%\begin{tabular}{l | c | c | c}
%Question & Hypothesis & \hspace{1.2em}$p$ & effect \\
%\hline
%\eIqIabbr/    & \tesig{relaxed$>$dilemma}{0.0003}{73\%} \\
%\hline
%\eIqIIabbr/   & \tesig{obvious$>$dilemma}{0.0001}{75\%} \\
%\hline
%\eIqIVabbr/   & \tesig{dilemma$>$obvious}{$<$0.0001}{83\%} \\
%\cline{2-4}
%              & \tesig{dilemma$>$relaxed}{$<$0.0001}{83\%} \\
%\hline
%\eIqVabbr/    & \tesig{dilemma$>$obvious}{0.036}{63\%} \\
%\hline
%\eIqVIIabbr/  & \tesig{dilemma$>$obvious}{$<$0.0001}{77\%} \\
%\cline{2-4}
%              & \tesig{dilemma$>$relaxed}{0.001}{71\%} \\
%\hline
%\eIqVIIIabbr/ & \tensig{dilemma$>$obvious}{0.141} \\
%\cline{2-4}
%              & \tesig{dilemma$>$relaxed}{0.0004}{73\%} \\
%\end{tabular}
%\egroup
%\caption[Prospective between-treatment results]{%
%Between-treatments results.
%%
%Each row indicates a hypothesis, the corresponding $p$-value, and the effect size if the result is significant ($p < 0.05$). Significant results are shown in bold; non-significant results are in \nsighcolor/.
%}
%  \label{tab:e1-between-results}
%\end{table}


\begin{table}[!p]
\centering
\bgroup
\def\arraystretch{1.3}
\setlength{\tabcolsep}{0.7em}
\begin{tabular}{l c c c c}
\toprule
Question & Hypothesis & $p$-value & Effect \\
\toprule
\eInobadabbr/ &\tesig{relaxed$>$dilemma}{$\bm{3.1\sqtimes 10^{-4}}$}{76\%} \\
\midrule
\eIclearbestabbr/ &\tesig{obvious$>$dilemma}{$\bm{1.1\sqtimes 10^{-4}}$}{78\%} \\
\midrule
\multirow{2}{10em}{\eInogoodabbr/} &\tesig{dilemma$>$obvious}{$\bm{1.9\sqtimes 10^{-7}}$}{88\%} \\
 &\tesig{dilemma$>$relaxed}{$\bm{1.2\sqtimes 10^{-7}}$}{87\%} \\
\midrule
\eIbalancedabbr/ &\tesig{dilemma$>$obvious}{0.036}{64\%} \\
\midrule
\multirow{2}{10em}{\eIdifficultabbr/} &\tesig{dilemma$>$obvious}{$\bm{2.7\sqtimes 10^{-5}}$}{80\%} \\
 &\tesig{dilemma$>$relaxed}{0.001}{73\%} \\
\midrule
\multirow{2}{10em}{\eIconsequencesabbr/} &\tensig{dilemma$>$obvious}{0.140} \\
 &\tesig{dilemma$>$relaxed}{$\bm{3.8\sqtimes 10^{-4}}$}{75\%} \\
\bottomrule
\end{tabular}

\egroup
\caption[Prospective between-treatment results]{%
Between-treatments results.
%
Each row indicates a hypothesis, the corresponding $p$-value, and the effect size if the result is significant ($p < 0.05$). Significant results are shown in bold; non-significant results are in \nsighcolor/.
}
  \label{tab:e1-between-results}
\end{table}


A summary of the data is shown in \cref{fig:e1-report}.
%
The percentages shown are total percent of participants who disagreed, were neutral, or agreed with the given question (from left to right) and the percentages for each separate response are plotted as colored bars.
%
For each question, data from each of the three treatments is plotted separately, and in some cases divergence is immediately apparent.
%
Of course, differences that seem apparent on this summary graph may or may not be statistically significant.


I tested my hypotheses as described above, and the results of those tests are show in \cref{tab:e1-single-results,tab:e1-between-results}.
%
\Cref{tab:e1-single-results} shows the single-treatment hypothesis results: 9 of my 13 hypotheses were confirmed by my data, while 4 were not.
%
Each entry in this table indicates the hypothesis (agree $\rightarrow$ ``A'' and disagree $\rightarrow$ ``D''), the $p$-value for that hypothesis (the hypothesis is confirmed if the $p$-value is below 0.05), and if confirmed, the common-language effect size for that test.
%
\Cref{tab:e1-between-results} shows the between-treatment hypothesis results: 8 of my 9 hypotheses were confirmed by my data and 1 was not.


Note that the common-language effect size is just the percentage of comparisons between the cases being tested that support the alternate hypothesis.
%
This means that, for example, if all responses are neutral, the common-language effect size when asserting that the responses are greater than a uniform distribution will be exactly 50\% (the theoretical minimum common-language effect size).
%
This is because 50\% of comparisons between an all-neutral data set and a uniform data set will support the alternate hypothesis (that the neutral data's median is higher) and 50\% will refute it (counting tied comparisons as half-supporting and half-refuting).
%
By the same logic, if responses were all ``somewhat agree,'' the effect size would be 70\%, and if responses were all ``strongly agree,'' the effect size would be 90\% (the theoretical maximum effect size in the studies presented here).
%
The effect sizes listed in \cref{tab:e1-single-results} range from 62\% to 71\%, which are moderate to strong effects.
%
The effect sizes in \cref{tab:e1-between-results} range from 63\% to 83\%, with most being strong effects at $>$70\% effect size.


Besides the single-treatment and between-treatment hypotheses, all three of the stakes hypotheses were confirmed.
%
For the first (low-stakes choices would elicit agreement that their stakes were low) the $p$-value was 0.0047 and the effect size was 65\%.
%
The second stakes hypothesis (that high-stakes choices would elicit disagreement with the same statement) had a $p$-value of 0.0011 and an effect size of 70\%.
%
Finally, the backup hypothesis (that agreement would be higher in the low-stakes case than the high-stakes case) was a given as the first two were confirmed; it had $p = 9.7\times10^{-11}$ and an effect size of 83\%.


\subsection{Discussion}

Out of the 25 specific hypotheses, 20 were supported by the data.
%
This indicates that most of the perceptual qualities I expected given the constraints used to generate choices were in fact identified by most of the survey participants.
%
In particular, the fact that all of the stakes hypotheses were confirmed indicates that \dunyazad/'s author-based estimation of which player goals are more and less important is working.
%
On a treatment-by-treatment basis, the observed properties were:
%
\begin{itemize}
  \item ``Obvious'' choices -- Participants tended to agree that obvious choices had a clear best option (question 2) and they tended to disagree with the statement that they had no good options (question 4). I expected that participants would disagree that all of the options were equally promising (question 5) and that these choices were difficult (question 7) but in both cases the data did not confirm these expectations.
  \item ``Relaxed'' choices -- Participants tended to agree that the stakes for these choices were low (question 3) and they tended to disagree with the statement that these choices had no good options (question 4). I expected participants to agree that these choices had no bad options (question 1) but the data did not support this hypothesis.
  \item ``Dilemma'' choices -- Participants tended to disagree that there were ``no bad options'' at these choices (question 1) and agree that there were ``no good options'' at these choices (question 4). Furthermore, they disagreed with the statement that these choices had a clear best option (question 2), and agreed that these choices were difficult and had important consequences (questions 7 and 8). However, I expected participants to agree that all options at these choices were about equally promising, but the data did not support this hypothesis.
\end{itemize}
%
Overall, the data support \dunyazad/'s ability to control stakes and outcome valences, but they show that it has a bit of trouble making outcomes seem similar.
%
The areas where its choices were able to produce the desired poetic effects are important successes for automatic reasoning about choice poetics, and they also imply that the goals my survey participants considered when judging the choices align at least somewhat with those the system assumes players will have.


Places were the data did not support my hypotheses are opportunities for further scrutiny.
%
To start with, I wanted to investigate whether the failed hypotheses were the result of general trends across all choices generated under a treatment condition or whether any single choice contributed disproportionately to an unexpected result.
%
To do this I broke down the results by individual questions within a treatment and plotted them to see if there was any indication of per-question differences.


\subsubsection{Option Relativity}


\begin{figure}[!h]
  \includegraphics[width=\textwidth]{fig/relaxed-q1.pdf}
  \caption[``No bad options'' prospective responses]{A graph of responses to question 1 under the ``relaxed'' treatment. The three numbers are the seeds used to generate the three different choices for this treatment. The graph setup is the same as in \cref{fig:e1-report}.}
  \label{fig:e1-relaxedq1}
\end{figure}


The first of my failed hypotheses involved the ``relaxed'' treatment.
%
I expected the ``relaxed'' treatment to elicit agreement with the statement ``There are no bad options at this choice,'' but it didn't do so definitively (the statistical test failed to reject the null hypothesis that the answers to this question were consistent with a uniform distribution of underlying responses).
%
A per-choice breakdown of the data for the relaxed condition shown in \cref{fig:e1-relaxedq1} gives a strong indication that the question with seed 4897 elicited qualitatively different responses than the two other questions in this treatment.
%
That particular question is shown in \cref{fig:e1-seed-4897}, and reveals one possible reason for what I observed: unlike the other two questions in the ``relaxed'' case, option 3 of this question lists ``no relevant skills'' rather than giving a relevant skill possessed by the player.


\begin{figure}[!h]
\centering
\fbox{
\parbox{0.95\columnwidth}{
  \slshape
You come to a tavern and decide to rest for a while.
%
A noble is bored and a peasant is bored and a merchant is selling a book of herbal lore.
%
What do you do?
\begin{enumerate}[itemsep=0pt,topsep=4pt,parsep=0pt,partopsep=0pt]
\item You tell the peasant a story \\
  (You have skill: storytelling).
\item You tell the noble a story \\
  (You have skill: storytelling).
\item You offer to trade the merchant your dragon scale for the merchant's book of herbal lore \\
  (no relevant skills).
\end{enumerate}
}
}
\caption[``Relaxed'' choice 4897]{The ``relaxed'' choice with seed 4897 (minus the framing, which is largely the same as that shown in \cref{fig:exframing}).}
  \label{fig:e1-seed-4897}
\end{figure}


The fact that the player doesn't have any skills relevant to that action does not mean that the action will fail, but it might make that option seem less desirable than the others at that choice.
%
None of the options at the other two choices in the ``relaxed'' treatment listed ``no relevant skills,'' they all listed some skill that the player had as relevant, which explains why there might be a difference in responses.
%
If that wording caused the shift, it would be consistent with Schwartz et al.'s theory of satisficing versus maximizing personalities \citep{Schwartz2002} for real-world choices.
%
Schwartz et al. have found that while some people are happy as long as their choices lead to satisfactory results, others are unhappy if their choices lead to good but nevertheless suboptimal results.
%
The strong split in responses for this specific case (including both significant ``strongly disagree'' and significant ``strongly agree'' contingents) indicates that some people may be interpreting the phrase ``bad option'' as meaning options that are absolutely bad, while others may be comparing the options against each other.
%
It would take more data to discern whether this distinction is what is at work here, but it is clear that it is an important distinction for choice poetics, and it is not yet something that \dunyazad/ reasons about.


Although \dunyazad/ does not reason about this, it is to some degree aware of the distinction between the question with seed 4897 and the other two questions in that treatment.
%
The constraints for the ``relaxed'' condition were that each option either ``enables'' or ``achieves'' a goal (in the technical senses; see page \pageref{page:choicetypes}), and in this case, the system generated two options that ``achieved'' a goal and one that merely ``enabled'' a goal, thus creating a distinction even on its own terms.
%
The other two questions in the relaxed category each included three options which ``achieved'' a goal.
%
In light of the survey results, it is clear that to construct choices that unambiguously have ``no bad options'' the system should not only require that each option works towards a player goal, but that each option is balanced against all others.

% TODO: HERE!

\subsubsection{Balancing Failures}

Another unsupported hypothesis was that in the ``dilemma'' treatment participants would agree with the statement ``All of the options at this choice are about equally promising.''
%
I expected this because one of the constraints of the ``dilemma'' treatment was that all of the threatened goals should have the same priority.
%
However, even for the individual choice in the ``dilemma'' treatment where participants reported the most agreement, 30\% of participants answered ``somewhat disagree.'' 


\begin{figure}[!h]
  \centering
  \includegraphics[width=0.8\textwidth,page=3]{fig/choices-cropped.pdf}
  \caption[Histogram of option preferences for ``dilemma'' choices]{A histogram of options selected by participants at the three different ``dilemma'' choices (each labeled by seed).}
  \label{fig:e1-dilchoices}
\end{figure}


\Cref{fig:e1-dilchoices} shows the options that participants said they would choose for the three dilemma choices.
%
For the first two choices, option 2 is a clear loser, and looking at the choices, it's easy to see why.
%
Both of those choices (which are nearly identical) involve being attacked by a dragon.
%
In both choices, option 2 is an option to attack the dragon yourself, but of course you have neither the ``fighting'' skill nor a weapon, and the dragon has both.
%
Although you also lack relevant skills for the other options, making a desperate attempt to flee from or pacify a dragon seems like a better choice than fighting it (at the very least, it did to all of my participants).


There are three factors contributing to the system's divergence from players' analysis of the options.
%
The first has to do with the granularity of expectations, the second with the granularity of goal priorities, and the third with stacking goals.


First, there are some clear arguments available to players as to why fighting might be a worse option than fleeing (for example) in an unfavorable situation.
%
One is that fighting presents a \emph{greater risk} of a bad result, and another is that fighting is directly related to a \emph{categorically worse} result.
%
The first argument has to do with the granularity of expectations: \dunyazad/ just recognizes events as \prq{unlikely}{,} \prq{neutral}{,} or \prq{likely}{.}
%
In this case, it reasons that since both fleeing and fighting are \prq{likely}{} to \prq{fail}{} the goal of avoiding the present threat, both options fail a high-priority goal.
%
However, given that all options are marked negatively, the prospect of fleeing evokes more hope for unwarranted success than the prospect of fighting.
%
This could be expressed in \dunyazad/ by introducing evaluations like \prq{somewhat\_likely}{} and \prq{very\_likely}{} that would distinguish these cases.
%
The \prq{skill\_link}{} mechanism would have to be updated to give estimates of these levels of likelihood for different actions in different circumstances, of course.


Another argument a player could make as to why fighting is worse than fleeing in this situation has to do with the valence of the results.
%
Internally, \dunyazad/ recognizes that the \prq{attack}{} option is likely to \prq{fail}{} both the \prq{avoid\_threats\_to}{} and \prq{preserve\_health}{} goals, while the \prq{flee}{} option is only expected to \prq{fail}{} the \prq{avoid\_threats\_to}{} goal.
%
However, as \dunyazad/ is written now, it treats an option that indicates one high-priority goal failure no differently form one that indicates multiple failures.
%
Particularly for choice structures which are supposed to have balanced options, some kind of counting logic would be useful to determine when one option is better or worse than another, even when they're in the same general category.


In the same vein, one could argue that even without counting how many goals fail, the \prq{preserve\_health}{} goal should be higher-priority than the \prq{avoid\_threats\_to}{} goal.
%
In part because of complexity concerns (although I have not tested this extensively) I made a decision to limit \dunyazad/ to two priority levels.
%
One could imagine instead a partially directed ``more-important-than'' graph between all player goals at each timepoint, which would be another way for \dunyazad/ to realize that fighting is worse than fleeing in this case.


The problem here is that the system's representation of player expectations is not fine-grained enough.
%
To the system, all of the options at these choices are expected to ``fail'' the player's goal of keeping their character alive and uninjured, but the system makes no distinction beyond that.
%
How certain does such failure seem to the player?
%
Exactly how badly does the player expect to fare when their goal is not met?
%
In this case, even when told that the situation is hopeless (or perhaps especially then), fleeing seemed a better option than attacking the dragon head-on, but the system doesn't distinguish those cases.
%
Based on this data, the system should be improved by adding more detail to its assessments of goal failure and success.


Any one of these corrections would help this particular case, but especially given that \dunyazad/ also has difficulty producing balanced positive options as mentioned in the previous sub-section, I suspect a more focused approach would be better.
%
After all, the likelihood reasoning \emph{is} doing a good job of predicting which options players will view positively or negatively; it just has a hard time figuring out whether options are equally positive or equally negative.
%
Rather than further overload the current reasoning, an additional subsystem dedicated to direct relative analysis of options should be added.
%
This system could separately maintain more-nuanced representations of goal priorities and outcome likelihoods, and then build arguments as to why each option is more- or less-desirable than each other option.
%
In many cases, this nuanced relative analysis could be short-circuited when the old low-fidelity analysis detects a difference.


In terms of the choice analysis method presented in \cref{ch:choice-poetics}, this exact problem was already mentioned in \cref{sec:cp-relative-option-analysis} (see the second paragraph where it talks about the balance of impacts).
%
I hypothesized here that \dunyazad/ would be able to fake a more detailed analysis by just insuring that the threatened goals had equal stakes, but that turned out not to be true.
%
In this case, what was learned from \dunyazad/'s failure backs up the theory: dilemmas are a case which needs closer inspection.
%
Getting this experimental result is still useful, because it provides a concrete example of why this is the case.


\subsubsection{Outcome Clarity}


\begin{figure}[!p]
  \includegraphics[width=\textwidth]{fig/obvious-q5-q7.pdf}
  \caption[Prospective balance and difficulty responses for ``obvious'' choices]{A graph of responses to questions 5 and 7 under the ``obvious'' treatment by seed. For question 5, the choice with seed 64487 is clearly different from the other two, and for question 7, the choice with seed 21105 is the one that stands out. \Cref{fig:e1-seed-64487,fig:e1-seed-21105} show the text of the divergent choices.}
  \label{fig:e1-obviousq57}
\end{figure}

\begin{figure}[!p]
\centering
\fbox{
\parbox{0.95\columnwidth}{
  \slshape
You come to a tavern and decide to rest for a while.
%
A merchant is bored and a peasant seems knowledgeable and an innkeeper seems knowledgeable.
%
What do you do?
\begin{enumerate}[itemsep=0pt,topsep=4pt,parsep=0pt,partopsep=0pt]
\item You gossip with the innkeeper \\
  (You are missing skill: negotiation).
\item You tell the merchant a story \\
  (You have skill: storytelling).
\item You gossip with the peasant \\
  (You are missing skill: negotiation)
\end{enumerate}
}
}
\caption[``Obvious'' choice 64487]{The ``obvious'' choice with seed 64487.}
  \label{fig:e1-seed-64487}
\end{figure}

\begin{figure}[!p]
\centering
\fbox{
\parbox{0.95\columnwidth}{
  \slshape
You come across some bandits attacking a merchant.
%
The bandits are threatening the merchant.
%
What do you do?
\begin{enumerate}[itemsep=0pt,topsep=4pt,parsep=0pt,partopsep=0pt]
\item You attack the bandits \\
  (They have skill: fighting. You are missing skill: fighting. They have no tool for fighting).
\item You travel onwards \\
  (no relevant skills).
\item You talk the bandits down \\
  (no relevant skills).
\end{enumerate}
}
}
\caption[``Obvious'' choice 21105]{The ``obvious'' choice with seed 21105.}
  \label{fig:e1-seed-21105}
\end{figure}


Not only did I expect participants to agree that options were balanced for ``dilemma'' choices, but I also expected them to disagree for ``obvious'' choices.
%
Here again my hypothesis was not supported by the data.
%
A per-choice analysis of question 5 in the ``obvious'' condition (the top half of \cref{fig:e1-obviousq57}) shows that again, one choice is divergent from the other two.
%
\Cref{fig:e1-seed-64487} shows that middle choice, which compared to the other two ``obvious'' choices is low-stakes (one of the other choices starts with a dragon attack, the other with bandits attacking some merchants).



Not only are the stakes for this choice low, but they are unclear.
%
What exactly does the player hope to gain by gossiping or by telling a story?
%
Perhaps friendship or some useful information, but those potential rewards seem both uncertain (even given a ``successful'' action) and questionably useful.
%
In contrast (albeit a contrast that participants did not see directly) the utility of fleeing from an attacking monster is clear, even if it is uncertain whether you will succeed.
%
Furthermore, there aren't any obvious risks associated with options 1 and 3, so even if the player is missing a relevant skill, they might still be worth trying.
%
Given this combination of low stakes, a dubious reward for the most-successful-seeming option, and seemingly consequence-free options all around, it is not hard to see how some might find these options ``about equally promising.''


On the other hand, the system's attempt to construct an obvious choice in this case was still somewhat successful.
%
7 of 9 participants who saw this choice ``strongly agreed'' with the statement ``There is a clear best option at this choice'' and 8 of those 9 picked option \#2 as the option they would choose.
%
While it might seem like a contradiction to agree (as 3 participants did) with both the statement that a choice has equally promising options and the statement that it has a clear best option, this highlights the difference between outcomes-focused evaluation of individual options and choice-oriented option comparison.
%
The phrasing of ``All of the options at this choice are about equally promising,'' suggests evaluating each option independently and comparing those values roughly.
%
In contrast, ``There is a clear best option at this choice,'' suggests comparing the options against each other to find one that is better than the others.
%
The fact that people often make decisions inconsistent with simple utility calculation is well-known (see e.g., \citep{Tversky1993}), so it should not be surprising that a context in which someone is asked to make a decision might elicit a different response than a context in which someone is asked to rate responses.


The implications for choice poetics are interesting, because choice poetics is concerned with both mindsets.
%
At least in terms of impact on the player, there's clearly a difference between a choice where all of the options seem ``about equally promising'' and where that's definitely not the case, even if both choices have ``a clear best option.''
%
Choices where options seem equally promising but most players choose a particular route regardless might even be good candidates for the focus of regret.
%
For example, if it turns out that the ``clear best option'' leads to failure and players are forced to revisit the original decision, having seen some potential in the alternatives they're more likely to feel that the choice was fair, and thus blame their own decision for the consequences.
%
At the same time, if the designer can be confident that most players will go down the ``obvious'' route first even without making the contrast with the other options huge, they can assume that most players will experience the regretful path as opposed to choosing the ultimately correct option the first time.


Despite the interesting revelations prompted by this case, \dunyazad/ does actually have a problem here: it is analyzing outcomes according to the game mechanics that it knows about, but the players in this case aren't playing a game: they're experiencing a single choice with very little context.
%
If someone had played a few of \dunyazad/'s output stories already, they'd have a view of the possibility space much more similar to \dunyazad/'s, especially in terms of the outcomes of actions like \prq{gossip}{} and \prq{tell\_story}{.}
%
If the players knew what to expect from both success and failure at these actions, and had a stronger sense of the role skills play in determining outcomes, they wouldn't get their hopes up, and the choice \emph{would} be ``obvious.''


That is the deeper issue here as well: the simple notions of ``success'' and ``failure'' when performing an action often don't correspond to overall positive an negative results when an action isn't clearly oriented towards some pressing player goal.
%
For example, an attempt to ``gossip'' despite lacking the relevant skill seems like it might still yield interesting results, and it isn't likely to be disastrous.
%
The same is not true of attacking an enemy: when they're more skilled or better-armed, there's a clear sense of danger (and of what is at stake).
%
In other words, \dunyazad/ is doing a good job of manipulating player expectations when it is heavy-handed, threatening high-priority player goals to make options appear hopeful or doomed.
%
However, when \dunyazad/ attempts to use low-priority goals in the same fashion, players don't follow its lead: ``My gossiping probably won't go `as intended?' Well, it might still be awesome!''


The simplest response to this is to just tread carefully when creating using low-stakes choices and assume that some of them aren't going to get the result you want, while making sure to use high-stakes choices if a particular property like obviousness is critical.
%
In the longer term, \dunyazad/ should recognize things like the desire to explore rather than always travel the beaten path (especially when the stakes are low).
%
At the same time, one area for future work is goal-probing questions, which would allow \dunyazad/ to ask the player (either directly or indirectly) to indicate their goals.
%
These questions would serve two purposes: first, they would allow \dunyazad/ to dynamically determine which goals (including low-priority goals) the player thinks are important.
%
Second, these questions when successful subtly influence players by committing them to a goal: one a player affirms a goal, they are more likely to behave consistent with that goal in the future.


\subsubsection{Conflicting Goals}


Our hypothesis that respondents would not find ``obvious'' choices difficult highlights a different choice in the ``obvious'' category.
%
Our data did not support this hypothesis, and as shown in \cref{fig:e1-obviousq57}, the choice with seed 21105 accounts for the majority of all responses that contradict it.
%
That choice is shown in \cref{fig:e1-seed-21105}, and from the system's perspective, it satisfies the definition of an obvious choice (see page \pageref{page:choicetypes}) because the second option ``achieves'' a player goal, while neither the first nor the third do, and both the first and third ``threaten'' a player goal.


The perceived difficulty of this decision probably stems from the fact that it pits two player goals against one another: the goal of self-preservation is best served by option 2, but the goal of helping others in need is best served by one of the other options.
%
This is similar to the lack of distinction between threatening one and two goals encountered above.
%
Even when one option at a choice clearly has the most-positive outcome for the player considering all goals to be equal, when that choice pits multiple goals against one another, it may be very difficult indeed.
%
In fact, choices with these structures may give rise to an entirely different set of poetic concerns that involves moral judgements (this is exactly how some morality thought experiments are constructed, for example).
%
A difficult decision between two desirable or undesirable outcomes feels completely different from a decision between two outcomes each justified by competing moral principles, and it can resonate powerfully to the broader narrative structures of a story.
%
In its current state, \dunyazad/ doesn't reason about morality at all: it makes no distinction between goals in terms of their root motivations, and it has no compunction about how a result is achieved as long as no player goals are threatened in the process.
%
While \dunyazad/ gets perhaps surprisingly far without these capabilities, they're clearly important and can come up by accident even when \dunyazad/ isn't trying to make use of them.


That said, a detailed inspection of the answer set that resulted in this choice reveals another problem with the system: ignoring the lack of moral reasoning, it's simply not working as intended.
%
In this case, \dunyazad/ actually thinks that travelling onwards serves the goal of preventing the threat to the merchant (because if the player travels onwards, that entire situation is left behind and therefore the threat no longer exists) while it has no conception that this serves a goal of self-preservation (although it understands that interfering by either means threatens the player's safety).
%
There are thus two more changes to the system suggested by this data: first a bug-fix related to the consequences of travelling onwards, and second, the addition of relative goal relevance across options: the idea that if all but one option threatens an important goal, then the remaining option can be seen as indirectly supporting that goal even if none of its outcomes directly further that goal.
%
Without running this experiment, I would eventually have found and fixed the ``travel onwards'' bug, but I may not have thought to make the second change.
%
In this case, the data served to help find and diagnose an anomaly in my system, which turned out to involve an error in the code, but at the same time also pointed to two future goals for the choice structure model: adding explicit moral reasoning, and a notion of relative goal relevance when all but one option relates to a goal in the same way.


\subsubsection{Stakes and Consequences}

The final unsupported hypothesis was that participants would feel more strongly that ``dilemma'' choices had important consequences than that ``obvious'' choices did.
%
This hypothesis was based on the idea that consequences might seem more relevant (and thus important) when a decision was more difficult.
%
Given that one of my ``obvious'' choices seemed difficult to many participants as discussed above, it is unsurprising that this hypothesis was not supported.


\begin{figure}[!h]
  \includegraphics[width=\textwidth]{fig/stakes-q8.pdf}
  \caption[Prospective consequences results by stakes]{A graph of responses to question 8 across all treatments by stakes.}
  \label{fig:e1-stakesq8}
\end{figure}


What is interesting is the effect of the choice stakes on this question.
%
Our similar hypothesis that participants would agree more in the ``dilemma'' case than the ``relaxed'' case \emph{was} supported by the data, and one big difference between those two treatments is the stakes associated with them.\footnote{A statistical analysis of the assertion that participants agree more with the statement that a choice's stakes are low in the ``relaxed'' case than the ``dilemma'' case (not one of my original hypotheses) reveals a strong significant effect ($p = 2.88\times10^{-5}$; effect size 76\%).}
%
In fact, when analyzing the data for question 8 according to stakes rather than the three treatments, high-stakes choices overwhelmingly seem as if they will have important consequences, while low-stakes choices are mixed (see \cref{fig:e1-stakesq8}).


Statistics confirm the obvious: not only did the high-stakes condition elicits significantly more agreement on question 8 than the low-stakes condition ($p = 2.14\times10^{-8}$; effect size 79\%), it was also significantly above a uniform distribution ($p = 4.9\times10^{-6}$; effect size 78\%).
%
Because all of the ``relaxed'' choices were low-stakes by design, there is of course correlation between the stakes and the treatments, but the 35 high-stakes cases were about evenly distributed between the ``obvious'' and ``dilemma'' treatments (16 in ``obvious'' and 19 in ``dilemma'').
%
Such an overwhelming effect (none of the 35 respondents who saw a high-stakes choices thought it would \emph{not} have important consequences) further indicates that the system is successful in predicting high-stakes player goals: for 94\% of choices where the system thought that an important player goal was affected, players agreed at least somewhat that the consequences seemed important.


\subsection{Conclusions}

Overall, this study confirmed \dunyazad/'s ability to construct choices based on player expectations when certain criteria are met.
%
Notably, many of the failed hypotheses involved situations where several options together affected how a choice was perceived:
%
\begin{itemize}
  \item For question 1 (``There are no bad options at this choice.'') relative rather than absolute judgements of what is a ``bad'' option may have come into play.
  \item For question 5 (``All of the options at this choice are about equally promising.'') it seems \dunyazad/ may need to make finer distinctions between different modes of goal failure, as several options expected to lead to failure may still seem to offer a range of possibilities when no better options are present.
  \item Question 7 (``This is a difficult choice to make.'') showed that a choice can be difficult when multiple goals conflict, even when expectations for one goal are much better than for another.
\end{itemize}
%
These results point to several possible improvements for \dunyazad/, and collectively reinforce the importance of considering choices holistically when evaluating choice poetics.


As already discussed, there are a number of changes that could be made to \dunyazad/ based on these results:
%
\begin{itemize}
  \item Implement separate ``satisfaction'' and ``maximization'' player decision modes so that the system can reason about how difficult a decision might seem whichever decision modality a player is using.
  \item Upgrade \dunyazad/'s system for estimating how individual options affect goals by adding more detail so that it can further distinguish different magnitudes of risk and reward.
  \item Have \dunyazad/ represent players' uncertainty about the possible outcomes of actions like ``gossip'' and ``tell story'' so that it can have a clearer picture of which options seem promising in low-stakes situations.
  \item Implement goal-probing questions (either direct or indirect) that allow the system to gain direct knowledge about player goals, especially for low-stakes options. This would also allow \dunyazad/ to further support role-playing by allowing different players to choose options which imply different goals or goal priorities.
  \item Implement a model of goal conflicts and more detailed relative goal priorities including moral reasoning.
  \item Implement the idea of relative goal relevance: if all but one option at a choice either threatens or enables a goal, then the remaining option implicitly does the opposite.
\end{itemize}
%
These changes could make \dunyazad/ even more successful at constructing choices which produce particular player expectations (and thus which can produce specific poetic effects).
%
They are also in line with my goal of getting \dunyazad/ to generate choices with more complex poetics, such as choices that elicit regret or prompt deliberation.


One important functionality of \dunyazad/ not tested in this first study was its ability to judge outcomes.
%
This study did largely verify that \dunyazad/'s player expectations about options are working, and it can generate outcomes that both support and betray those expectations.
%
\Cref{sec:exp-retrospective} uses this functionality to test how accurate \dunyazad/'s estimates of retrospective perceptions are.

% TODO: Reformat tables to match!!

\section{Experiment II: Retrospective Impressions}

\label{sec:exp-retrospective}

The results from testing \dunyazad/'s prospective impression subsystem were largely successful, so I ran another experiment to test the retrospective impression system.
%
In this experiment both prospective and retrospective impression constraints were used to create six kinds of choices:

\begin{itemize}
  \item \textbf{Expected Success}--These choices have three options which present themselves as leading to success, and each option has a successful outcome. In terms of \prq{outcome\_feel}{} values (cf. \cref{tab:dunyazad-outcome-feel-structures}) each option at these choices is an \prq{expected\_success}{} option. No matter which option a participant selects, they will be picking an option that suggests success, and they will get a successful outcome (as far as \dunyazad/ understands things).
  \item \textbf{Expected Failure}--The opposite of expected success: every option suggests and results in failure. Internally, each option is an \prq{expected\_failure}{.}
  \item \textbf{Unexpected Failure}--These choices consist of three \prq{unfair}{} options. In other words, each option seems as though it will be successful, but each ultimately results in failure. These choices should appear largely the same as \prq{expected\_success}{} choices before an option is chosen.
  \item \textbf{Unexpected Success}--The opposite of unexpected failures, these choices consist of three \prq{miracle}{} options--each indicates failure but results in success. These choices mimic ``expected failure'' options until the player selects an option and sees an outcome.
  \item \textbf{Obvious Success}--These choices have a single \prq{expected\_success}{} option and two \prq{expected\_failure}{} options, forming an \prq{obvious}{} \prq{option\_structure}{.} The outcomes of each choice align with what the option text suggests: the ``good'' option leads to success and the others lead to failure.
  \item \textbf{Obvious Failure}--These choices have a single \prq{unfair}{} option and two options which are either \prq{miracle}{}s or \prq{nice\_gamble}{}s. Just like ``obvious success'' choices, their \prq{option\_structure}{} is \prq{obvious}{.} Of course, instead of leading to the expected results, each option results in the opposite of what it suggests.
\end{itemize}

These six choice structures were designed to manipulate three variables: valence of outcomes, expectedness of outcomes, and the presence of viable alternatives.
%
Note that due to the study design, each participant only ever saw a single choice, they could not form judgements based on comparing choices across categories.
%
The bulk of the hypotheses were designed simply to verify that the participants agreed with \dunyazad/'s internal evaluation of these variables, but there were a few hypotheses designed to probe for possible relationships between these six conditions.
%
Overall the results show that \dunyazad/ did a good job producing positive outcomes, but struggled with negative outcomes, and (partially as a result of this) had some trouble producing unexpected results.
%
Despite these problems, the data reveal some interesting interactions between the three underlying variables, such as a difference in satisfaction based on the presence or absence of viable alternatives and a relationship between the valence of unexpected outcomes and their perceived fairness.


\subsection{Method}

The experimental setup for this experiment was largely the same as the setup for the prospective impressions experiment described in \cref{sec:exp-prospective}.
%
This section lists the major differences between the two setups but does not reiterate the details of the original setup (see \cref{sec:e1-method}) that were unchanged.
%
The first obvious difference was the sample size.
%
As already mentioned, there were six types of choice generated, rather than 3, for a total of 18 choices (three per condition, just as in the original experiment).
%
Instead of showing each choice to 10 subjects, each choice was shown to 15 subjects, which meant that there were nominally 270 participants instead of 90.
%
Luckily the use of Amazon Mechanical Turk means that there is very little effort per participant, so the logistics of the second study weren't drastically different from those of the first.


\subsubsection{Extra Constraints}

In an attempt to avoid some of the problems with low-stakes choices in the original experiment, every choice in this experiment was required to have high stakes.
%
Furthermore, rather than allowing \dunyazad/ to pick any setup it wanted, the three choices in each condition were generated using fixed setups (\prq{market}{,}, \prq{monster\_attack}{,} and \prq{threatened\_innocents}{}).
%
This helped insure that the setup text wasn't a significant variable between the cases, and that each case would include a variety of setups.
%
Setup parameters were still allowed to vary freely; the \prq{market}{} setup in particular has multiple possible configurations.


\subsubsection{Framing}

Unlike the first survey, where participants were asked to pick the option that they ``would have chosen,'' participants in this study were instructed:
%
``Please read the following short story which presents you with a choice, make a decision, and then answer the questions about that choice below.''
%
The choices were presented with a radio button next to each option, and hovering over the text of each option highlighted it.
%
When the participant clicked anywhere over the text of an option, the outcome text for that option would be displayed in the space directly below it (moving any following options down).
%
At the same time, the selected option text and the outcome text were highlighted in bold and the radio button for the selected option was filled in, while the non-selected options were faded to gray and their radio buttons were disabled.
%
Once an option was selected the highlight-on-hover and click-to-choose functionality were disabled.
%
Immediately before the option texts the phrase ``To choose, click on an option below. Once you make a decision you will not be able to change it,'' appeared to make sure that participants understood the format.
%
The option selected by the player was recorded as part of the study data; every single participant selected an option and thus saw an outcome.


Of course, although this decision mechanism mimicked the decision-making environment of modern browser-based narrative choice games, it meant that participants would see the outcome of their decision before answering questions about the options that they had.
%
This approach was chosen intentionally to maintain the flow of the setup into the choice; asking a set of survey questions after seeing the choice but before making a decision would have an impact on the poetics of the choice.
%
Of course, having seen an outcome inevitably changes one's perception of an option (especially if that outcome was unexpected).
%
The data in the first survey therefore is more informative about option-related poetics.
%
To counteract this somewhat, the following text appeared before the option questions:
%
``For these questions, consider just the options presented, and ignore the outcome of the option that you chose.''
%
Complying with such a request is not entirely possible, of course, but the option-related questions in this survey were only present to establish a point of comparison with the first study, and were not the focus of the results.


\subsubsection{Compensation}

Participants were still Amazon Mechanical Turk workers, but for this study, each participant was paid \$2.00 instead of \$0.50.
%
This was partially because this study involved more questions, and partially because of the desire to pay more fairly for participants' time.
%
Based on the quality of the responses that I received and on reviews of the tasks posted on \url{https://turkopticon.ucsd.edu/}, I believe that this level of pay was satisfactory both from my own perspective and from the perspective of the participants.
%
The total cost of the experiment including fees paid to Amazon and the costs of pilot tasks (4 pilot tasks were posted to debug the survey before posting the 270 main tasks) was \$765.20.


\subsubsection{Questions}

Of course, one major difference between the studies was the set of statements included.
%
This study had three sections: statements about the options, statements about the outcome, and statements about motivation.
%
The option statements were actually a subset of the statements from the original survey, with extra text to encourage participants to think only about the options and ignore the outcome as mentioned above.
%
The following five option statements (listed by their internal labels) were included: \\[0.3\baselineskip]
%
{
  \setstretch{1.0}
  \mydef{5em}{\raggedleft\textbf{\eIIoptobviousabbr/}}{\hangparas{1.5em}{1}\slshape 1. ``Considering just the options, there seems to be a clear best option at this choice.''} \\
  \mydef{5em}{\raggedleft\textbf{\eIIoptbalancedabbr/}}{\hangparas{1.5em}{1}\slshape 2. ``Ignoring outcomes, the options at this choice all seem about equally good (or bad).''} \\
  \mydef{5em}{\raggedleft\textbf{\eIIoptnobadabbr/}}{\hangparas{1.5em}{1}\slshape 3. ``Ignoring outcomes, there are no options that seem bad at this choice.''} \\
  \mydef{5em}{\raggedleft\textbf{\eIIoptnogoodabbr/}}{\hangparas{1.5em}{1}\slshape 4. ``Ignoring outcomes, none of the options at this choice seem good.''} \\
  \mydef{5em}{\raggedleft\textbf{\eIIoptstakesabbr/}}{\hangparas{1.5em}{1}\slshape 5. ``Considering just the options, the stakes for this choice seem low.''} \\
}


In addition to option statements, this survey included statements about the outcome that participants saw: \\[0.3em]
%
{
\setstretch{1.0}
\mydef{7em}{\raggedleft\textbf{\eIIoutfairabbr/}}{\hangparas{1.5em}{1}\slshape 1. ``Given the options available, the outcome I got is fair.''} \\
\mydef{7em}{\raggedleft\textbf{\eIIoutsenseabbr/}}{\hangparas{1.5em}{1}\slshape 2. ``The outcome that I got makes sense given the option that I selected.''} \\
\mydef{7em}{\raggedleft\textbf{\eIIoutbadabbr/}}{\hangparas{1.5em}{1}\slshape 3. ``I got a bad outcome.''} \\
\mydef{7em}{\raggedleft\textbf{\eIIouthappyabbr/}}{\hangparas{1.5em}{1}\slshape 4. ``I'm happy with the option that I chose.''} \\
\mydef{7em}{\raggedleft\textbf{\eIIoutunfairabbr/}}{\hangparas{1.5em}{1}\slshape 5. ``The outcome that I got is unfair, given the options available.''} \\
\mydef{7em}{\raggedleft\textbf{\eIIoutunexpectedabbr/}}{\hangparas{1.5em}{1}\slshape 6. ``The outcome that I got is completely unexpected.''} \\
\mydef{7em}{\raggedleft\textbf{\eIIouttrickabbr/}}{\hangparas{1.5em}{1}\slshape 7. ``There is an outcome.'' (This is a trick question to test whether you're paying attention. Please simply indicate that you are in complete \emph{disagreement}.)} \\
\mydef{7em}{\raggedleft\textbf{\eIIoutbrokenabbr/}}{\hangparas{1.5em}{1}\slshape 8. ``There might be a problem with this choice--the outcome I got does not make sense.''} \\
\mydef{7em}{\raggedleft\textbf{\eIIoutgoodabbr/}}{\hangparas{1.5em}{1}\slshape 9. ``The outcome that I got is a good outcome.''} \\
\mydef{7em}{\raggedleft\textbf{\eIIoutexpectedabbr/}}{\hangparas{1.5em}{1}\slshape 10. ``I pretty much expected the outcome that I got.''} \\
\mydef{7em}{\raggedleft\textbf{\eIIoutregretabbr/}}{\hangparas{1.5em}{1}\slshape 11. ``I wish I had chosen a different option.''} \\
}
%
Underlying these statements were 5 concepts which were each addressed by a pair of statements with counterbalanced answers.
%
These underlying concepts were: \textbf{fairness} (questions 1 and 5), \textbf{sense} (questions 2 and 8), \textbf{valence} (questions 3 and 9), \textbf{satisfaction} (questions 4 and 11), and \textbf{expectedness} (questions 6 and 10).
%
The order of the questions was scrambled and chosen such that no pair of complimentary statements would appear adjacent to each other in order to encourage subjects to approach each question individually.


Although the outcome questions were the main focus of this study, some questions asking participants to self-report their motivations were added at the end of the study to get direct feedback about modes of engagement.
%
These questions were prefaced with the prompt:
%
\begin{quote}
  \slshape
Please answer the following questions about your motivations for making the decision you made, as well as about your motivations and goals in general when playing/reading interactive stories.
\end{quote}
%
The first question asked directly about motivation:
%
\begin{quote}
  \slshape
  1. Which of the following motive(s) contributed to your decision? (pick one or more) \\
  %
  (Note: no submissions will be rejected based on this information; wanting to get this survey done quickly is a perfectly valid reason for making a decision.) \\
  \begin{itemize}
    \item[\ding{111}] I'm taking an online survey. I just chose an option quickly so that I could complete the survey.
    \item[\ding{111}] I was just curious to find out what would happen if I chose the option I did.
    \item[\ding{111}] I imagined a character in the story situation and chose what that character would do.
    \item[\ding{111}] I chose what I would have chosen were I in the situation described in the story.
    \item[\ding{111}] I chose the option that I thought would lead to the most interesting/satisfying result.
    \item[\ding{111}] I looked at the skill information and chose the option that I thought would be most successful.
    \item[\ding{111}] Other (please explain):
  \end{itemize}
\end{quote}
%
A free text area was provided below the ``Other'' option.
%
As suggested in the prompt, participants were free to select more than one option or leave them all blank.
%
Note that the options here mostly correspond to the modes of engagement proposed in \cref{ch:choice-poetics}.
%
Of course, by providing exactly these options players may be subconsciously coerced into aligning their motivations with the archetypes that I have defined, and so the results of this study shouldn't be taken to be a confirmation that player motivations \emph{do} fit into the theoretical framework discussed above.
%
However, the relative frequency of different responses still provides some information about whether players' motives are diverse or generally similar, for example.
%
The responses to this question were labelled respectively as \pr{speed}, \pr{curious}, \pr{role}, \pr{avatar}, \pr{interesting}, \pr{power}, and \pr{other}.

In a similar vein, the next question asked about how players evaluate outcomes:
%
\begin{quote}
  \slshape
  2. Which of the following judgement(s) contributes to how you generally define a ``good'' outcome in interactive experiences like the one you just played? (pick one or more) \\
  \begin{itemize}
    \item[\ding{111}] I feel an outcome is good when something good happens to my character in the story world.
    \item[\ding{111}] I feel an outcome is good if it is an interesting development in the story.
    \item[\ding{111}] I feel an outcome is good when it fits the role that I am building for my character.
    \item[\ding{111}] I feel an outcome is good when it makes my character more powerful.
    \item[\ding{111}] I feel an outcome is good when it makes progress towards beating a game.
    \item[\ding{111}] Other (please explain):
  \end{itemize}
\end{quote}
%
These responses were labelled respectively as \pr{avatar}, \pr{interesting}, \pr{role}, \pr{power}, \pr{progress}, and \pr{other}.
%
Directly following was effectively the same question but for ``bad'' outcomes:
%
\begin{quote}
  \slshape
  3. Which of the following judgement(s) contributes to how you generally define a ``bad'' outcome in interactive experiences like the one you just played? (pick one or more)
  \begin{itemize}
    \item[\ding{111}] I feel an outcome is bad when something bad happens to my character in the story world.
    \item[\ding{111}] I feel an outcome is bad if it doesn't develop the plot.
    \item[\ding{111}] I feel an outcome is bad when my character doesn't do what I expected them to do.
    \item[\ding{111}] I feel an outcome is bad when my character expresses a value or opinion that is different from what I wanted them to express.
    \item[\ding{111}] I feel an outcome is bad when it makes my character less powerful.
    \item[\ding{111}] I feel an outcome is bad when it prevents me from making progress towards beating a game.
    \item[\ding{111}] Other (please explain):
  \end{itemize}
\end{quote}
%
These responses were labelled as \pr{avatar}, \pr{interesting}, \pr{no.control}, \pr{value.clash}, \pr{power}, \pr{progress}, and \pr{other}.

The final question in the motives section asked about consistency of motives.
%
Unlike the other questions in this section, it forced participants to select a single response (although it still had an ``other'' response):
%
\begin{quote}
  \slshape
  4. Do you feel you approach all interactive experiences (e.g., Choose-Your-Own-Adventure novels, video games, tabletop role-playing games, etc.) with a consistent set of motivations and judgements, or do your motivations and judgements change from story to story?
  \begin{itemize}
    \item[\ding{109}] I feel that my motivations and judgements are the same no matter what interactive story I'm playing.
    \item[\ding{109}] I feel that my motivations and judgements change from story to story.
    \item[\ding{109}] I don't play interactive stories often enough to have a definite answer.
    \item[\ding{109}] Other (please explain):
  \end{itemize}
\end{quote}
%
The responses to this question were labelled \pr{consistent}, \pr{variable}, \pr{no.exp}, and \pr{other}.
%
Although the questions in the motivation section weren't subject to any statistical tests, they provide some informal information about how players approach choices.
%
A further experiment designed to seriously investigate player motivations would require a much tighter design, and would likely derive no benefit from using \dunyazad/ to generate its choices, as \dunyazad/ isn't designed to be aware of or work with modes of engagement.


Unlike the first survey, this survey included an optional text area at the end for any additional feedback participants wanted to share.
%
This additional feedback was not analyzed, but responses were used to help with approval decisions (several participants indicated honest confusion about some aspect of the age/fluency question; their responses were accepted).


Besides these differences, the details of this survey were the same as the first survey, including the format of response options and the framing of the various sections.
%
Of course, this survey included a different set of hypotheses.

\subsection{Hypotheses}

The hypotheses in this study were tested in largely the same way as those in the prospective study.
%
Each ``agree'' or ``disagree'' hypothesis was tested by comparing the results under a given condition with a uniform distribution of responses using a Mann-Whitney-Wilcoxon U test with an appropriate alternate hypothesis.
%
A uniform distribution with 35 elements was used for all comparisons, as this was the multiple of five (the number of response categories) closest to the number of samples in most of the relevant cases (32-36 for each condition, although as low as 8 in some sub-conditions; these low-$n$ comparisons are discussed where they appear).
%
For between-condition hypotheses, the two sample populations were compared directly using Mann-Whitney-Wilcoxon U tests.
%
In all cases, results are taken to be significant when their $p$-values are $<$0.05, and common-language effect sizes (see \cref{sec:e1-results} for an explanation) are indicated when results are significant.

\subsubsection{Option Hypotheses}

The three option structures used in this study were \prq{positive\_alternatives}{,} \prq{negative\_alternatives}{,} and \prq{obvious}{,} as described above.
%
Each option structure corresponded to two conditions with different outcomes.
%
These paired conditions are listed in the header of \cref{tab:e2-option-hypotheses}; each option structure was predicted to yield a different set of answers to the five option statements, as shown in that table.
%
These 30 hypotheses (5 statements $\times$ 3 option structures $\times$ 2 conditions per structure) were meant to check whether \dunyazad/ was producing the desired option structures successfully.


Given the results of the first experiment, several of these hypotheses were not expected to be supported by the data.
%
These tentative hypotheses are marked with `\lc/'s in \cref{tab:e2-option-hypotheses}.
%
In particular, hypotheses having to do with options being balanced or there being no clear best option were not expected to be well-supported by the data.

\begin{table}[!h]
\centering
\bgroup
\def\arraystretch{1.3}
\setlength{\tabcolsep}{0.6em}
\begin{tabular}{r c c c}
\toprule
\multirow{2}{5em}{\centering Question} &%
 \eIIexpectedsuccessabbr/ &%
 \eIIobvioussuccessabbr/ &%
 \eIIexpectedfailureabbr/ \\
&%
 \eIIunexpectedfailureabbr/ &%
 \eIIobviousfailureabbr/ &%
 \eIIunexpectedsuccessabbr/ \\
\midrule
\eIIoptobviousabbr/ &%
 disagree\lc/ &%
 agree &%
 disagree\lc/ \\
\eIIoptbalancedabbr/ &%
 agree\lc/ &%
 disagree &%
 agree\lc/ \\
\eIIoptnobadabbr/ &%
 agree\lc/ &%
 disagree &%
 disagree \\
\eIIoptnogoodabbr/ &%
 disagree &%
 disagree &%
 agree \\
\eIIoptstakesabbr/ &%
 disagree &%
 disagree &%
 disagree \\
\bottomrule
\end{tabular}

\egroup
\caption[Retrospective option hypotheses]{Hypotheses about option statements in the retrospective study. Each entry corresponds to two hypotheses: one for each condition listed in the header of that column. Because the two conditions in each column share option structures, they are predicted to elicit identical responses to the option-related statements. Low-confidence hypotheses are marked with a `\lc/'.}
  \label{tab:e2-option-hypotheses}
\end{table}


\subsubsection{Basic Outcome Hypotheses}

\begin{table}[!p]
\centering
\bgroup
\def\arraystretch{1.3}
\setlength{\tabcolsep}{0.6em}
\begin{tabular}{r  c c}
\toprule
\multirow{2}{5em}{\centering Question} &%
 \eIIexpectedsuccessabbr/ &%
 \eIIunexpectedsuccessabbr/ \\
&%
 \eIIobvioussuccessmainabbr/ &%
 \eIIobviousfailurealtabbr/ \\
\midrule
\eIIoutfairabbr/ &%
 agree &%
 agree \\
\eIIoutunfairabbr/ &%
 disagree &%
 disagree \\
\eIIoutsenseabbr/ &%
 agree &%
 agree \\
\eIIoutbrokenabbr/ &%
 disagree &%
 disagree \\
\eIIoutgoodabbr/ &%
 agree &%
 agree \\
\eIIoutbadabbr/ &%
 disagree &%
 disagree \\
\eIIouthappyabbr/ &%
 agree &%
 agree \\
\eIIoutregretabbr/ &%
 disagree &%
 disagree \\
\eIIoutexpectedabbr/ &%
 agree &%
 disagree \\
\eIIoutunexpectedabbr/ &%
 disagree &%
 agree \\
\midrule
\multirow{2}{5em}{Question} &%
\eIIexpectedfailureabbr/ &%
\eIIunexpectedfailureabbr/ \\
&%
\eIIobvioussuccessaltabbr/ &%
\eIIobviousfailuremainabbr/ \\
\midrule
\eIIoutfairabbr/ &%
 agree &%
 disagree \\
\eIIoutunfairabbr/ &%
 disagree &%
 agree \\
\eIIoutsenseabbr/ &%
 agree &%
 disagree \\
\eIIoutbrokenabbr/ &%
 disagree &%
 agree \\
\eIIoutgoodabbr/ &%
 disagree &%
 disagree \\
\eIIoutbadabbr/ &%
 agree &%
 agree \\
\eIIouthappyabbr/ &%
 - &%
 disagree \\
\eIIoutregretabbr/ &%
 - &%
 agree \\
\eIIoutexpectedabbr/ &%
 agree &%
 disagree \\
\eIIoutunexpectedabbr/ &%
 disagree &%
 agree \\
\bottomrule
\end{tabular}

\egroup
\caption[Retrospective outcome hypotheses]{Outcome-related hypotheses for the retrospective study. Each column lists two conditions in each half of the table; these conditions the same expected and actual outcome valences, and are thus predicted to elicit the same responses. Eight conditions are listed here because the ``obvious'' conditions have sub-cases: [main] for participants who chose the ``best'' option and [alt] for participants who chose otherwise. These sub-cases arise because participants experience outcomes with different valences depending on the option they choose.}
  \label{tab:e2-outcome-hypotheses}
\end{table}

Assuming that \dunyazad/ is able to construct options which indicate either success or failure as intended, and that players agree with \dunyazad/'s evaluations of outcomes, there are four different types of option in this study.
%
Each option indicates either failure or success, and each outcome is either largely successful or largely unsuccessful.
%
The \prq{expected\_success}{} condition, for example, consists of three options which both indicate and lead to a good outcome (or at least, \dunyazad/ has constructed them with that intent).
%
The \prq{obvious\_success}{} and \prq{obvious\_failure}{} conditions are more complicated than the other four, which each consist of three options of the same type.
%
\prq{obvious\_success}{} has one option which suggests and results in success, but two that suggest and lead to failure.
%
Likewise, \prq{obvious\_failure}{} has one option which suggests success but leads to failure, and two that suggest failure but lead to success.
%
This leads to the [main] and [alt] sub-cases, [main] indicating participants who were shown an \prq{obvious\_*}{} choice and chose the ``best'' option, and [alt] indicating participants who chose a different option at these choices.


\Cref{tab:e2-outcome-hypotheses} lists all of the outcome-related hypotheses.
%
As with \cref{tab:e2-option-hypotheses}, each cell corresponds to two hypotheses: one for each condition/case listed at the top of its column.
%
The top half lists hypotheses about ultimately positive outcomes, while the bottom half lists hypotheses about negative outcomes.
%
At the same time, the left column covers expected results (where the option indicates a result of the same valence as the outcome) and the right column covers unexpected results.
%
The third underlying variable in this study, namely, whether the options at a choice seemed to indicate similar or divergent outcomes, was not expected to cause any ``disagree'' predictions to become ``agree'' predictions or vie versa, which is why each cell in this table can list two hypotheses.
%
\cref{tab:e2-outcome-hypotheses} represents a total of 76 individual hypotheses, which makes a total of 106 flat ``agree'' or ``disagree'' hypotheses when combined with the 30 hypotheses in \cref{tab:e2-option-hypotheses}.
%
Taken together, these hypotheses are designed to ensure that \dunyazad/'s choices have the properties \dunyazad/ assumes they have in terms of both what is suggested by their options and in terms of how their outcomes are interpreted.

\subsubsection{Relative Outcome Hypotheses}

\begin{table}[!h]
\centering
\bgroup
\def\arraystretch{1.3}
\setlength{\tabcolsep}{0.6em}
\begin{tabular}{r c}
\toprule
Question & Hypothesis \\
\midrule
\eIIoutfairabbr/ & unexp. failure$>$obv. failure [main] \\
\eIIoutunfairabbr/ & unexp. failure$<$obv. failure [main] \\
\eIIoutsenseabbr/ & unexp. failure$>$obv. failure [main] \\
\eIIoutbrokenabbr/ & unexp. failure$<$obv. failure [main] \\
\eIIoutgoodabbr/ & unexp. failure$>$obv. failure [main]\lc/ \\
\eIIoutbadabbr/ & unexp. failure$<$obv. failure [main]\lc/ \\
\eIIouthappyabbr/ & unexp. failure$<$obv. failure [main]\lc/ \\
\eIIoutregretabbr/ & unexp. failure$>$obv. failure [main]\lc/ \\
\eIIoutexpectedabbr/ & unexp. failure$>$obv. failure [main]\lc/ \\
\eIIoutunexpectedabbr/ & unexp. failure$<$obv. failure [main]\lc/ \\
\bottomrule
\end{tabular}

\egroup
\caption[Retrospective free vs. forced failure hypotheses]{Relative hypotheses regarding free vs. forced failures. Hypotheses marked with a `\lc/' are low-confidence hypotheses.}
  \label{tab:e2-free-vs-forced-failure-hypotheses}
\end{table}

As with the first study, there are some hypotheses about the relative agreement between two different statements as opposed to just agreement with or disagreement with a single statement.
%
These hypotheses are probing for a variety of interaction effects which may or may not be present, as opposed to trying to verify that \dunyazad/ is working as intended.


The first such proposed effect is roughly state as: ``Unexpected failure is more acceptable when a choice seems free vs. forced.''
%
In other words, I suspected that when a promising outcome leads to failure, players will be more unhappy if that seemed to be the only promising outcome, rather than one of several promising outcomes.
%
The reasoning behind this is that when a choice seems to have both good and bad options, players may expect to be rewarded for successfully picking a good option.
%
On the other hand, if only good options are present, players may not see their choice as meritorious, and thus may have lower expectations of the result.
%
Additionally, after seeing an unexpected negative outcome, if there were other seemingly good options present players may reason that one of those options would have led to a better results, while no such reasoning is available if the other options seemed bad.
%
\Cref{tab:e2-free-vs-forced-failure-hypotheses} lists the individual hypotheses associated with this proposed effect, but they can be stated more simply in terms of the underlying variables.
%
This proposed effect predicts that compared to surprising failures at \prq{obvious}{} choices, surprising failures at \prq{positive\_alternatives}{} choices will be perceived as more fair, more sensible, better, more regretted (because changing options seems more promising), and more expected.
%
When I formulated these hypotheses I was most confident in the effects on perceptions of fairness and sense, so the other hypotheses are marked as low-confidence with a `\lc/'.


\begin{table}[!h]
\centering
\bgroup
\def\arraystretch{1.3}
\setlength{\tabcolsep}{0.6em}
\begin{tabular}{r c}
\toprule
Question & Hypothesis \\
\midrule
\eIIoutgoodabbr/ & exp. success$<$obv. success [main] \\
\eIIoutbadabbr/ & exp. success$>$obv. success [main] \\
\eIIouthappyabbr/ & exp. success$<$obv. success [main] \\
\eIIoutregretabbr/ & exp. success$>$obv. success [main] \\
\bottomrule
\end{tabular}

\egroup
\caption[Retrospective chosen vs. inevitable success hypotheses]{Relative hypotheses regarding chosen vs. inevitable successes.}
  \label{tab:e2-chosen-vs-inevitable-success-hypotheses}
\end{table}


Another proposed difference was between the ``expected success'' and ``obvious success [main]'' cases.
%
Both of these involve expected success outcomes, the only difference being whether other outcomes at the choice were promising or not.
%
One hypothesis was that when alternatives seemed dangerous, a positive outcome would seem slightly better (both cases were predicted to evaluate their outcomes as good overall) because of a greater perceived difference in outcomes (although participants didn't get to see the outcomes of non-chosen options).
%
This accounts for the first two rows of \cref{tab:e2-chosen-vs-inevitable-success-hypotheses}, while the other two indicate a prediction that participants will feel more satisfied with the option that they chose when the alternatives are negative.
%
The reasoning here is that curiosity about the results of non-chosen but still positive-seeming options would slightly change the degree of agreement with the statements about being satisfied and regret (both cases were predicted to elicit overall agreement with the satisfied statement and disagreement with the regret statement).
%
As with the free-vs-forced hypotheses above, these hypotheses were predicting a change in the degree of agreement/disagreement on some items, rather than an outright contrast between one category which is expected to agree with a statement while another disagrees.


\begin{table}[!h]
\centering
\bgroup
\def\arraystretch{1.3}
\setlength{\tabcolsep}{0.6em}
\begin{tabular}{r c}
\toprule
Question & Hypothesis \\
\midrule
\eIIoutfairabbr/ & unexp. failure$<$unexp. success \\
\eIIoutunfairabbr/ & unexp. failure$>$unexp. success \\
\eIIoutsenseabbr/ & unexp. failure$<$unexp. success \\
\eIIoutbrokenabbr/ & unexp. failure$>$unexp. success \\
\bottomrule
\end{tabular}

\egroup
\caption[Retrospective good vs. bad unexpected hypotheses]{Relative hypotheses regarding good vs. bad surprises.}
\label{tab:e2-good-vs-bad-unexpected-hypotheses}
\end{table}


Yet another predicted effect was a difference in perceived fairness between good and bad unexpected results, with hypotheses listed in \cref{tab:e2-good-vs-bad-unexpected-hypotheses}.
%
In both the unexpected failure and unexpected success case, \dunyazad/ effectively contradicts the normal effects of skill and item possession on the outcomes of actions to produce an effect that it labels as either \prq{unfair}{} or a \prq{miracle}{.}
%
In both cases, it could be argued that these outcomes are ``unfair,'' although perhaps in two possible senses of the word (``unjust'' vs. ``cheating'').
%
The prediction in this case is that participants will respond more strongly to negative surprises than to positive surprises, being more likely to label negative surprises as ``unfair'' or ``broken'' than positive surprises.


\begin{table}[!h]
\centering
\bgroup
\def\arraystretch{1.3}
\setlength{\tabcolsep}{0.6em}
\begin{tabular}{r c}
\toprule
Question & Hypothesis \\
\midrule
\eIIoutgoodabbr/ & unexp. failure$<$exp. failure \\
\eIIoutbadabbr/ & unexp. failure$>$exp. failure \\
\eIIouthappyabbr/ & unexp. failure$<$exp. failure \\
\eIIoutregretabbr/ & unexp. failure$>$exp. failure \\
\eIIoutgoodabbr/ & obv. failure [main]$<$exp. failure \\
\eIIoutbadabbr/ & obv. failure [main]$>$exp. failure \\
\eIIouthappyabbr/ & obv. failure [main]$<$exp. failure \\
\eIIoutregretabbr/ & obv. failure [main]$>$exp. failure \\
\bottomrule
\end{tabular}

\egroup
\caption[Retrospective expected vs. unexpected failure hypotheses]{Relative hypotheses regarding expected vs. unexpected failures.}
\label{tab:e2-expected-vs-unexpected-failure-hypotheses}
\end{table}

Besides the effect of valence given surprise, I also made predictions regarding the effect of surprise given valence, for both positive and negative outcomes.
%
\Cref{tab:e2-expected-vs-unexpected-failure-hypotheses,tab:e2-expected-vs-unexpected-success-hypotheses} list the relevant hypotheses, which are separate for failures and successes.
%
For failures, the expectation is that unexpected bad things will be perceived as worse all-around than expected bad things.
%
A secondary prediction is that participants will be more likely to want to change their choices when a bad result is unexpected than when it is hinted at beforehand.
%
Because there are two conditions that give rise to unexpected failures, there are a total of eight individual hypotheses that result instead of just four.

\begin{table}[!h]
\centering
\bgroup
\def\arraystretch{1.3}
\setlength{\tabcolsep}{0.6em}
\begin{tabular}{r c}
\toprule
Question & Hypothesis \\
\midrule
\eIIoutgoodabbr/ & exp. success$<$unexp. success \\
\eIIoutbadabbr/ & exp. success$>$unexp. success \\
\eIIouthappyabbr/ & exp. success$<$unexp. success \\
\eIIoutregretabbr/ & exp. success$>$unexp. success \\
\eIIoutgoodabbr/ & obv. success [main]$<$unexp. success \\
\eIIoutbadabbr/ & obv. success [main]$>$unexp. success \\
\eIIouthappyabbr/ & obv. success [main]$<$unexp. success \\
\eIIoutregretabbr/ & obv. success [main]$>$unexp. success \\
\bottomrule
\end{tabular}

\egroup
\caption[Retrospective expected vs. unexpected success hypotheses]{Relative hypotheses regarding expected vs. unexpected successes.}
\label{tab:e2-expected-vs-unexpected-success-hypotheses}
\end{table}

For successes, somewhat similar logic applies: a good surprise may seem sweeter than something which is expected beforehand.
%
This translates to hypotheses that unexpected successes will be perceived as better and will leave participants more satisfied with their choices than expected successes.


Note that for both of these proposed surprise-based effects, a fourth condition is technically relevant: the ``obvious success [alt]'' case consists of outcomes that are expected failures, and the ``obvious failure [alt]'' case consists of unexpected successes.
%
However, neither of these cases were expected to attract many participants, and expectations when choosing an option which presumably seems worse than the ``correct'' option are probably not the same as expectations when forced to choose one of three negative options, so these two cases were not included in these hypotheses.


In total, the proposed effects are backed by 34 relative hypotheses.
%
These hypotheses are exploratory: they aren't designed to verify that \dunyazad/ works as intended, but instead they represent guesses about how participants might react.
%
If they are confirmed, they have interesting implications for the choice poetics, and each proposed effect could be investigated further.


\subsubsection{Motivation Hypotheses}

The final category of hypotheses for the retrospective study was a set of hypotheses about the motivation questions.
%
These questions were not really tied to \dunyazad/'s performance, but rather designed to gather preliminary data on player motivations that could help guide the design of a study focused on modes of engagement.
%
The hypotheses are thus simple thresholds, and they don't have associated statistical tests.
%
Each of the motivation questions had at least one associated hypothesis:

\begin{itemize}
  \item The first motivation question was about what motives contributed to participants' decisions. The hypotheses for this question were that the first (\pr{speed}), fourth (\pr{avatar}), and sixth (\pr{power}) options would each be selected by at least 50\% of participants.
  \item The second motivation question concerned how participants decide that an option is ``good.'' The hypotheses were that the first (\pr{avatar}), fourth (\pr{power}), and fifth (\pr{progress}) options would each be selected by at least 50\% of participants.
  \item The third motivation question was like the second but it focused on ``bad'' outcomes. The hypotheses were that the first (\pr{avatar}), third (\pr{no.control}), fifth (\pr{power}), and sixth (\pr{progress}) options would each be selected by at least 50\% of participants.
  \item The final motivation question forced participants to choose a single answer and asked whether they believed their motives were consistent or malleable across different interactive experiences. The hypothesis for this question was that among participants who did not select the third option (\pr{no.exp}, indicating that they did not have much experience with interactive narratives), at least 70\% would select the second option (\pr{variable}) over the first (\pr{consistent}) and fourth (\pr{other}) options.
\end{itemize}

All of they hypotheses in the motivation section were motivated mostly by my own sense of how people play games, which is of course idiosyncratic.
%
The real value of the motivation questions is not in whether their related hypotheses are supported or not, but in the aggregate data that they supply.
%
The hypotheses are still an important point of reference, however, as they help avoid a completely post-hoc analysis of the data.

\subsection{Results}

As with the previous experiment, some submissions were rejected because they failed to answer screening questions correctly (the same two screening questions were used, one asking for age confirmation and the other an attention check within the body of the survey).
%
There were 10 rejected responses, and these were resubmitted to Amazon so that a new participant was found, so there a total of 280 responses including rejections.
%
The rejected responses were not used in the analysis.


Unfortunately, the system for ensuring that the same participant did not take the survey more than once was configured improperly at the beginning of data collection, and some participants submitted multiple responses (across different conditions).
%
When the issue was discovered it was corrected, and all responses from individuals who saw more than one choice were not used during data analysis (to preserve the condition that judgements were not based on cross-category comparisons).
%
A total of 65 responses from 14 individuals who submitted more than one survey were removed, reducing the total number of valid responses from 270 to 205.
%
Luckily, because workers who accepted multiple tasks generally did not do more than one from the same batch, the impact of this error was distributed fairly evenly.
%
After filtering out these responses and accounting for places where valid responses were missing an individual item, each condition still had between 32 and 36 valid responses, and each choice had between 8 and 14.
%
Thus while the statistical power of the study was reduced, the problem wasn't severe enough to call into question the validity of the results.


Unlike the first study, data was only filtered based on submission approval and this multiple-submission issue.
%
This meant that data from respondents who left some items blank was still included in the analysis.
%
Before doing the first study, I was worried that leaving a question blank might be a sign that someone was completing the survey haphazardly, and I wanted to filter out such responses.
%
After seeing the data from the first survey, including the distribution of response times, I was much less worried about this issue, and although leaving a question blank might be a sign that a participant was distracted, they might still be answering the other items honestly, especially if they answered the trick question correctly (as every single response that was accepted and wasn't a duplicate did).
%
The second study also included a mechanism that changed the color of each response to green to encourage participants to fill out every question.
%
In the end there were a total of only 13 missing responses out of the 3075 hypothesis-related Likert items seen by non-multiple-participants whose submissions were accepted.
%
The most-impacted single item was the statement ``[...] there are no options that seem bad [...],'' which was missing 5 of the 205 responses that it should have had.


\subsubsection{Response Times}

The median response time was 672 seconds (11:12) before filtering and 592 seconds (9:52) after filtering, which is in line with this survey being a little more than twice as long as the first survey.
%
The minimum accepted response time was 155 seconds (2:35) which did not seem short enough to filter out categorically, especially as a quick look at the answers given did not suggest random responses.
%
There were a total of eight accepted responses that were completed in less than 240 seconds (4:00).
%
The fact that the median decreased significantly after filtering likely has to do with the filtering of multiple responses from individual participants.
%
Normally if a worker on Mechanical Turk wants to do more than one task from a single category, they will accept a batch of tasks and then proceed to work through them, to ensure that other workers don't use up the remaining tasks while they work on one.
%
This means that their response times will be linearly increasing, because the response time is just the time between accepting and submitting a task.
%
This also means that multiple-submitters will account for more than their fair share of long-response-time submissions, and so filtering them out will tend to reduce the median response time.
%
Even workers who only submit a single task are likely doing other tasks on Mechanical Turk and may engage in the same accept-to-reserve behavior, which is why I did not consider filtering out responses that took ``too long.''
%
Without clever instrumentation of the survey page, the data only provides an upper bound on the time that it actually takes a worker to complete a task.

\begin{figure}[!p]
  \centering
  \includegraphics[width=\textwidth]{{fig/cropped-outcomes-report-01-opt.obvious}.pdf}
  \includegraphics[width=\textwidth]{{fig/cropped-outcomes-report-02-opt.balanced}.pdf}
  \includegraphics[width=\textwidth]{{fig/cropped-outcomes-report-03-opt.nobad}.pdf}
  \includegraphics[width=\textwidth]{{fig/cropped-outcomes-report-04-opt.nogood}.pdf}
  \includegraphics[width=\textwidth]{{fig/cropped-outcomes-report-05-opt.stakes}.pdf}
  \includegraphics[width=\textwidth]{fig/outcomes-legend.pdf}
  \caption[Retrospecitve option results]{Option-related results from the retrospective study. Cf. \cref{fig:e1-report}.}
  \label{fig:e2-option-report}
\end{figure}

\begin{figure}[!p]
  \centering
  \includegraphics[width=\textwidth]{{fig/cropped-outcomes-report-10-out.fair}.pdf}
  \includegraphics[width=\textwidth]{{fig/cropped-outcomes-report-11-out.unfair}.pdf}
  \includegraphics[width=\textwidth]{{fig/cropped-outcomes-report-12-out.sense}.pdf}
  \includegraphics[width=\textwidth]{{fig/cropped-outcomes-report-13-out.broken}.pdf}
  \includegraphics[width=\textwidth]{fig/outcomes-legend.pdf}
  \caption[Retrospective fair/unfair \& sense/nonsense results]{Fair/unfair and sense/nonsense results from the retrospective study. Setup is the same as \cref{fig:e1-report}.}
  \label{fig:e2-outcome-report-fair-sense}
\end{figure}

\begin{figure}[!p]
  \centering
  \includegraphics[width=\textwidth]{{fig/cropped-outcomes-report-06-out.good}.pdf}
  \includegraphics[width=\textwidth]{{fig/cropped-outcomes-report-07-out.bad}.pdf}
  \includegraphics[width=\textwidth]{{fig/cropped-outcomes-report-14-out.happy}.pdf}
  \includegraphics[width=\textwidth]{{fig/cropped-outcomes-report-15-out.regret}.pdf}
  \includegraphics[width=\textwidth]{fig/outcomes-legend.pdf}
  \caption[Retrospective good/bad \& satisifed/dissatisfied results]{Good/bad and satisfied/dissatisfied results from the retrospective study. Setup is the same as \cref{fig:e1-report}.}
  \label{fig:e2-outcome-report-valence-satisfaction}
\end{figure}

\begin{figure}[!t]
  \centering
  \includegraphics[width=\textwidth]{{fig/cropped-outcomes-report-08-out.expected}.pdf}
  \includegraphics[width=\textwidth]{{fig/cropped-outcomes-report-09-out.unexpected}.pdf}
  \includegraphics[width=\textwidth]{fig/outcomes-legend.pdf}
  \caption[Retrospective expected/unexpected results]{Expected/unexpected retrospective results. Setup as in \cref{fig:e1-report}.}
  \label{fig:e2-outcome-report-surprise}
\end{figure}

\subsubsection{Summary}

A summary of the results appears in \cref{fig:e2-option-report,fig:e2-outcome-report-fair-sense,fig:e2-outcome-report-valence-satisfaction,fig:e2-outcome-report-surprise}, which have the same format as \cref{fig:e1-report}.
%
\Cref{fig:e2-option-report} includes all of the option-related questions, and can be compared directly to results in \cref{fig:e1-report}, especially for the last four conditions, which correspond in pairs to the ``obvious'' and ``dilemma'' conditions of the first experiment, respectively.
%
In \cref{fig:e2-outcome-report-fair-sense,fig:e2-outcome-report-valence-satisfaction,fig:e2-outcome-report-surprise}, the questions are ordered such that complimentary question pairs are adjacent, and comparing the data visually there seems to be a good deal of symmetry between these pairs, as expected.
%
Overall, the data indicate that \dunyazad/ is better at creating positive expectations and results than negative ones: compare the incredibly one-sided results for the \prq{expected\_success}{} condition to the much more mixed results for \prq{expected\_failure}{,} for example.


As mentioned above, each of the rows in these figures represents between 32 and 36 responses, depending on the condition.
%
The sub-cases (e.g., \prq{obvious\_failure[main]}{} vs. \prq{obvious\_failure[alt]}{}) are not separated here, although they are analyzed separately for some of the hypotheses.
%
For the \prq{obvious\_success}{} condition, 25 of the 33 responses fell into the [main] sub-case, while the remaining 8 participants chose an option other than the one that \dunyazad/ expected would be most promising.
%
For the \prq{obvious\_failure}{} condition, the [main] sub-case contained 20 of the 35 responses, while the [alt] sub-case contained the remaining 15.
%
Note that each condition can be broken down into three individual choices, and there is a general expectation that these three choices will elicit generally similar responses, as they are constructed using the same set of constraints (although they are required to use three different setups).
%
Slight variation between choices within a condition should not disrupt the overall results, of course, but if there is significant divergence, it may be a sign that \dunyazad/'s internal reasoning about a choice doesn't agree with how participants experience that choice.


\subsubsection{Option Results}

\Cref{tab:e2-option-results} shows the results for the option-related hypotheses, with significant results listed in bold.
%
Note that in line with the results of the first study, none of the low-confidence hypotheses were supported by the data.
%
These hypotheses (marked with `\lc/') involved predictions which don't hold up if participants judge options relative to each other instead of in absolute terms, and the prospective study indicated strongly that this was the case for many participants.

\begin{table}[!t]
\centering
\bgroup
\def\arraystretch{1.3}
\setlength{\tabcolsep}{0.4em}
\begin{tabular}{r  c  c c c  c  c c c  c  c c c}
\toprule
\multirow{2}{5em}{\centering Question} &%
 & \multicolumn{3}{c}{\eIIexpectedsuccessabbr/} &%
 & \multicolumn{3}{c}{\eIIobvioussuccessabbr/} &%
 & \multicolumn{3}{c}{\eIIexpectedfailureabbr/} \\
&%
 & \multicolumn{3}{c}{\eIIunexpectedfailureabbr/} &%
 & \multicolumn{3}{c}{\eIIobviousfailureabbr/} &%
 & \multicolumn{3}{c}{\eIIunexpectedsuccessabbr/} \\
\midrule
\multirow{2}{5em}{\raggedleft \eIIoptobviousabbr/} &%
 & \tensig{D\lc/}{0.987} &%
 & \tesig{A}{$\bm{8.3\sqtimes 10^{-5}}$}{75\%} &%
 & \tensig{D\lc/}{0.467} \\
&%
 & \tensig{D\lc/}{0.128} &%
 & \tesig{A}{0.001}{70\%} &%
 & \tensig{D\lc/}{0.717} \\
\midrule
\multirow{2}{5em}{\raggedleft \eIIoptbalancedabbr/} &%
 & \tensig{A\lc/}{0.809} &%
 & \tesig{D}{0.003}{69\%} &%
 & \tensig{A\lc/}{0.485} \\
&%
 & \tensig{A\lc/}{0.767} &%
 & \tesig{D}{0.045}{61\%} &%
 & \tensig{A\lc/}{0.794} \\
\midrule
\multirow{2}{5em}{\raggedleft \eIIoptnobadabbr/} &%
 & \tensig{A\lc/}{0.807} &%
 & \tesig{D}{$\bm{4.5\sqtimes 10^{-4}}$}{72\%} &%
 & \tensig{D}{0.082} \\
&%
 & \tensig{A\lc/}{0.683} &%
 & \tesig{D}{0.013}{65\%} &%
 & \tesig{D}{0.011}{66\%} \\
\midrule
\multirow{2}{5em}{\raggedleft \eIIoptnogoodabbr/} &%
 & \tesig{D}{$\bm{1.1\sqtimes 10^{-5}}$}{78\%} &%
 & \tesig{D}{$\bm{4.1\sqtimes 10^{-5}}$}{76\%} &%
 & \tensig{A}{0.140} \\
&%
 & \tesig{D}{0.013}{65\%} &%
 & \tesig{D}{0.004}{68\%} &%
 & \tensig{A}{0.883} \\
\midrule
\multirow{2}{5em}{\raggedleft \eIIoptstakesabbr/} &%
 & \tensig{D}{0.279} &%
 & \tensig{D}{0.081} &%
 & \tensig{D}{0.310} \\
&%
 & \tensig{D}{0.257} &%
 & \tensig{D}{0.121} &%
 & \tesig{D}{0.003}{68\%} \\
\bottomrule
\end{tabular}

\egroup
\caption[Retrospective option results]{Option-related results in the retrospective experiment. Each row lists results for a single question; each column stacks results for two conditions (listed at the top) with identical \prq{option\_feel}{} constraints. Each entry indicates the hypothesis (`D' for `disagree' and `A' for `agree'), the $p$-value, and the common-language effect size for confirmed hypotheses (which are marked in bold instead of \nsighcolor/ where $p < 0.05$).} % TODO: Make this a one-liner to fit on the same page as the figure above...
  \label{tab:e2-option-results}
\end{table}



One immediate pattern in these results is the lack of support for the ``low stakes'' hypotheses.
%
In our initial study, the hypothesis that choices which \dunyazad/ labelled as high-stakes would appear that way to participants was clearly confirmed by the data, with a $p$-value of 0.0011 and an effect size of 70\% (a strong effect).
%
In this study, every single choice was required to be high-stakes, but only in a single case (the \prq{unexpected\_success}{} condition) was the relevant hypothesis confirmed.
%
One reason why this might be the case is that in this study, \dunyazad/ was not allowed to control either the setup or the stakes of the choices, whereas in the initial study \dunyazad/ was free to control both.
%
The high-stakes choices in this study therefore necessarily included choices with \prq{market}{} setups, where it is impossible for \dunyazad/ to construct a scenario that directly threatens the player's character.
%
A breakdown of responses to the stakes question across all conditions by setup (\cref{fig:e2-stakes-by-setup}) seems to confirm that the \prq{market}{} setup is a problem: it elicits an even split between agree and disagree responses while the other two setups lean towards disagreeing.

To test this theory, I did a followup analysis with three hypotheses--each individual setup as a condition was hypothesized to elicit disagreement with the low stakes statement.
%
The results of this analysis confirmed my suspicions.
%
The test for the \prq{market}{} hypothesis does not reject the null hypothesis ($p = 0.502$), while the tests for the \prq{threatened\_innocents}{} and \prq{monster\_attack}{} hypotheses are confirmed ($p = 0.041$; effect size 60\% and $p = 0.0093$; effect size 64\% respectively).
%
Essentially, when \dunyazad/ creates choices without constraints regarding their stakes, the high-stakes choices that it does create are in fact perceived to be high-stakes by players.
%
However, when \dunyazad/ is specifically asked to create high-stakes choices using certain fixed setups, such as the \prq{market}{} setup, it doesn't manage to create choices that are convincingly high-stakes.
%
Whether this means that the perception of stakes has more to do with the setup text than the actual content of the options and/or outcomes is an open question that could be tested.

\begin{figure}[!t]
  \centering
  \includegraphics[width=\textwidth]{{fig/cropped-extra-outcomes-report-by-setup-05-opt.stakes}.pdf}
\hspace*{1.5pt}\includegraphics[width=\textwidth]{fig/outcomes-legend.pdf}
  \caption[Retrospective stakes results by setup]{Stakes results across all conditions grouped by the setup used.}
  \label{fig:e2-stakes-by-setup}
\end{figure}


Besides the unconfirmed stakes hypotheses, there were two other places where high-confidence option-related hypothesis were not confirmed by the data.
%
The first is in the \prq{expected\_failure}{} case for the ``no bad'' statement.
%
Although the \prq{unexpected\_success}{} condition, which in theory generates identically-suggestive options, was confirmed to disagree with this statement, the data did not support this hypothesis for the \prq{expected\_failure}{} condition.
%
Of course, this condition is one susceptible to being undermined by relative value judgements: if all of the options seem bad, one might reason that it doesn't matter which is chosen, and therefore none of them are ``bad'' in the sense that choosing them would be a mistake.
%
It is telling that the obvious cases confirmed corresponding hypotheses: when a single positive option was present alongside negative options, participants strongly rejected the notion that there were ``no bad options'' present.

\begin{figure}[!b]
  \centering
  {\sffamily Expected Failure}
  \includegraphics[width=\textwidth]{{fig/cropped-detailed-report-expected_failure-02-opt.nobad}.pdf}
  {\sffamily Unexpected Success}
  \includegraphics[width=\textwidth]{{fig/cropped-detailed-report-unexpected_success-02-opt.nobad}.pdf}
  \includegraphics[width=\textwidth]{fig/detailed-outcomes-legend.pdf}
  \caption[Retrospective dilemma ``no bad options'' results]{Results for the ``no bad options'' statement in the \prq{expected\_failure}{} (top) and \prq{unexpected\_success}{} (bottom) conditions, grouped by choice. Each choice is identified by the numeric seed that was used to create it.}
  \label{fig:e2-extra-nobad}
\end{figure}


But what about the difference between the \prq{expected\_failure}{} and \prq{unexpected\_success}{} cases?
%
\Cref{fig:e2-extra-nobad} shows the results for each individual choice in both cases, and there isn't a clear culprit.
%
The choice with seed 58403 is the single choice among the six in that figure that elicited the most agreement with the statement, but it still tilts towards disagreement.
%
Given that both conditions include a substantial minority of ``agree'' and even ``strongly agree'' responses, there doesn't seem to be a clear underlying reason for the division in the results.
%
What is clear is that giving \dunyazad/ the ability to make relative rather than purely absolute value judgements should be a high priority.

\begin{figure}[!b]
  \centering
  {\sffamily Expected Failure}
  \includegraphics[width=\textwidth]{{fig/cropped-detailed-report-expected_failure-03-opt.nogood}.pdf}
  {\sffamily Unexpected Success}
  \includegraphics[width=\textwidth]{{fig/cropped-detailed-report-unexpected_success-03-opt.nogood}.pdf}
  \includegraphics[width=\textwidth]{fig/detailed-outcomes-legend.pdf}
  \caption[Retrospective dilemma ``no good options'' results]{Results for the ``no good options'' statement in the \prq{expected\_failure}{} (top) and \prq{unexpected\_success}{} (bottom) conditions, grouped by choice.}
  \label{fig:e2-extra-nogood}
\end{figure}


The final point of interest among the option-related results was the data's failure to support the hypotheses that the \prq{expected\_failure}{} and \prq{unexpected\_success}{} cases would be viewed as having ``no good options.''
%
Again, the dilemma case in the first experiment produced reactions that supported an equivalent hypothesis, so this result is somewhat surprising.
%
Here however the reasons for this discrepancy are clear.


\Cref{fig:e2-extra-nogood} shows the responses for these cases, and it is clear that the choices with seeds 87991, 8015, 47794, and 33152 are not yielding the expected results: participants are indicating that there is at least one good option at these choices.
%
\Cref{fig:e2-seed-87991} shows the choice with seed 87991, and the problem is immediately apparent: \dunyazad/ has used a \prq{travel\_onwards}{} action as a ``clearly bad result.''
%
This is a result of an overcorrection from the problems with \prq{travel\_onwards}{} that were discovered during the first study.
%
In that study, \prq{travel\_onwards}{} was a cure-all option which could get rid of any problem due to the way that it handled switching scenes.
%
For this second study, that bug was fixed, and instead, \prq{travel\_onwards}{} is marked as failing any goals relating to unresolved situations that it leaves behind. 
%
In this case, \dunyazad/ thinks that the  player will have an \prq{avoid\_accusations}{} goal with respect to the accused merchant, and simply leaving the merchant behind clearly fails this goal.

\begin{figure}[t]
\fseed{87991}
\caption[``Expected failure choice 87991]{The choice with seed 87991 minus the framing (which is mostly the same between choices). Boxed numbers before each outcome indicate the number of participants that chose that outcome. Note the popularity of the third option.}
\label{fig:e2-seed-87991}
\end{figure}


Of course, the underlying problem here is goal prioritization.
%
If given an good option to save the merchant, many players are eager to do so, but when such options are risky, they prefer to preserve their own safety.
%
With \dunyazad/'s two-level goal priority system, which was unchanged from the first experiment, it's not possible to represent this complexity.
%
Under the current system then, instead of \prq{travel\_onwards}{} being seen as an option with too few consequences, it is seen as an option with too many consequences.
%
The choice with seed 8015 also includes a \prq{travel\_onwards}{} option in a similar situation, and the inclusion of these options was likely the reason that players felt that the \prq{expected\_failure}{} cases actually \emph{did} have some good options.


\begin{figure}[h]
\fseed{47794}
\caption[``Unexpected success choice 47794]{The choice with seed 47794 minus its framing. Boxed numbers indicate the number of participants that selected each outcome.}
\label{fig:e2-seed-47794}
\end{figure}


Recall that the \prq{unexpected\_success}{} case exhibited the same problem, however.
%
The reason for this was different, however, because neither the choice with seed 47794 nor the choice with seed 33152 had \prq{travel\_onwards}{} options.
%
Instead, the reason that these choices were rated as having at least one good option is probably their outcomes.
%
Because participants saw an outcome before answering any questions, the responses to the option questions could have been affected by the outcomes seen.
%
\Cref{fig:e2-seed-47794} shows the choice with seed 47794, and looking at it reveals a likely explanation for the ``no good options'' response.


As shown in that figure, six of ten participants who saw that choice chose the first option, and were ``unexpectedly'' successful.
%
But remember that \dunyazad/'s notion of what is expected is written in terms of system-wide rules: ``normally'' someone missing the negotiation skill should \emph{always} fail to talk someone else down, so if such an action succeeds the result is unexpected.
%
This kind of expectation is something that players would learn if they interacted with the system over a period of time, but participants in this study only saw a single choice.
%
Faced with a positive outcome, it would be natural to conclude that perhaps the negotiation skill isn't important after all, and therefore that the option that was chosen was a good one.
%
This line of reasoning would not be available to players before seeing an outcome, but as stated before, the risk of such influences was intentionally accepted as the price of a more natural decision-making experience.


Because the choice with seed 33152 included a very similar \prq{talk\_down}{} option which was also quite popular, this outcome-dependent effect can explain the failed hypothesis in the \prq{unexpected\_success}{} condition.
%
Unfortunately, the impact of this effect and the \prq{travel\_onwards}{} issue affecting the choices with seeds 87991 and 8015 is not negligible.
%
When it comes to evaluating the outcome-related hypotheses, these choices didn't convey the intended prospecitve poetics, and this could impact their retrospective poetics as well.
%
Additionally, two other choices--the \prq{obvious\_success}{} choices with seeds 47371 and 20739--also included \prq{travel\_onwards}{} options.
%
These are shown in \cref{fig:e2-seeds-47371-20739}.
%
Luckily, the choice with seed 47371 happens to give participants an option to pursue the goal of saving the merchant that's more attractive than the \prq{travel\_onwards}{} option, so the same problem doesn't occur.
%
In fact, this is a good example of correct use of \prq{travel\_onwards}{} as an option that seems bad, which is reinforced by the fact that zero participants chose it.
%
However, based on the option popularities in choice 20739, that choice has the same problems as choices 47794 and 33152: the ``correct'' option involves some perceived risk or cost (in this case giving away an item to resolve the situation) and so simply ignoring the merchant's plight is seen as acceptable.


\begin{figure}[p]
\fseed{47371} \\
\fseed{20739}
\caption[``Obvious success choices 47371 and 20739]{The \prq{obvious\_success}{} choices 47371 and 20739 minus framing. Boxed numbers indicate the number of participants who saw each outcome.}
\label{fig:e2-seeds-47371-20739}
\end{figure}


Note that these options are in the \prq{obvious\_succeess}{} condition, and as such participants who saw them are divided into the \prq{[main]}{} and \prq{[alt]}{} sub-cases according to which option they chose.
%
For choice 47794, all 11 participants who chose option 2 are in the \prq{[main]}{} sub-case, while the other two participants are in the \prq{[alt]}{} sub-case.
%
For choice 20739, there are only two participants that the system thinks chose the ``correct'' option, and the majority of participants are put into the \prq{[alt]}{} sub-case.
%
This means that the meaning of the sub-cases is somewhat changed from its intent.
%
The \prq{[alt]}{} cases are supposed to consist of participants who intentionally chose a non-positive option despite the presence of a positive one, and as such they are assumed to be quite small, to the point that not many hypotheses were proposed which included \prq{[alt]}{} cases.
%
Because of \dunyazad/'s failure to predict the complexities of player goals, the \prq{[alt]}{} case consists largely of participants who thought they were choosing the best option available, but who disagreed with \dunyazad/'s evaluation of what that was.
%
Of course, although these effects diminish the number of participants in the \prq{[main]}{} cases, at least their meaning is not diluted very much: only a few participants from poorly-formed choices like 20739 wound up choosing the options that \dunyazad/ considered best, and so only a few participants in the \prq{[main]}{} cases weren't choosing what they thought was an obviously superior option.

\begin{figure}[b]
\fseed{99500}
\caption[``Unexpected success choice 99500]{The choice with seed 99500 minus framing. Boxed numbers indicate the number of participants that selected each outcome.}
\end{figure}

Before moving on to the outcome results, there is one more choice that stands out as a little bit odd:

% TODO: 99500


\subsubsection{Outcome Results}

By 

% TODO: Mention these three!!
%\fseed{95923} // "risky" vs. costly
%\fseed{46585} // unprovoked transformation seems bad
%\fseed{40550} // costly!

\subsection{Discussion}

\subsection{Conclusions}


% TODO: HERE

\begin{table}[p]
\centering
\bgroup
\def\arraystretch{1.3}
\setlength{\tabcolsep}{0.4em}
\begin{tabular}{r  c@{\hspace{2em}}  c c c  c@{\hspace{2em}} c c c}
\toprule
\multirow{2}{5em}{\centering Question} &%
 & \multicolumn{3}{c}{\eIIexpectedsuccessabbr/} &%
 & \multicolumn{3}{c}{\eIIunexpectedsuccessabbr/} \\
&%
 & \multicolumn{3}{c}{\eIIobvioussuccessmainabbr/} &%
 & \multicolumn{3}{c}{\eIIobviousfailurealtabbr/} \\
\midrule
\multirow{2}{6em}{\raggedleft \hangpara{1.3em}{1}\eIIoutfairabbr/} &%
 & \tesig{A}{$\bm{6.8\sqtimes 10^{-11}}$}{88\%} &%
 & \tesig{A}{$\bm{1.6\sqtimes 10^{-6}}$}{80\%} \\
&%
 & \tesig{A}{$\bm{1.7\sqtimes 10^{-8}}$}{87\%} &%
 & \tesig{A}{$\bm{1.1\sqtimes 10^{-5}}$}{85\%} \\
\midrule
\multirow{2}{6em}{\raggedleft \hangpara{1.3em}{1}\eIIoutunfairabbr/} &%
 & \tesig{D}{$\bm{2.7\sqtimes 10^{-10}}$}{87\%} &%
 & \tesig{D}{$\bm{3.5\sqtimes 10^{-5}}$}{76\%} \\
&%
 & \tesig{D}{$\bm{1.7\sqtimes 10^{-8}}$}{87\%} &%
 & \tesig{D}{$\bm{4.2\sqtimes 10^{-6}}$}{86\%} \\
\midrule
\multirow{2}{6em}{\raggedleft \hangpara{1.3em}{1}\eIIoutsenseabbr/} &%
 & \tesig{A}{$\bm{8.2\sqtimes 10^{-9}}$}{85\%} &%
 & \tesig{A}{$\bm{1.3\sqtimes 10^{-4}}$}{74\%} \\
&%
 & \tesig{A}{$\bm{4.7\sqtimes 10^{-8}}$}{85\%} &%
 & \tesig{A}{$\bm{4.2\sqtimes 10^{-6}}$}{86\%} \\
\midrule
\multirow{2}{6em}{\raggedleft \hangpara{1.3em}{1}\eIIoutbrokenabbr/} &%
 & \tesig{D}{$\bm{6.9\sqtimes 10^{-9}}$}{85\%} &%
 & \tesig{D}{$\bm{7.1\sqtimes 10^{-6}}$}{78\%} \\
&%
 & \tesig{D}{$\bm{1.7\sqtimes 10^{-6}}$}{82\%} &%
 & \tesig{D}{$\bm{1.1\sqtimes 10^{-5}}$}{85\%} \\
\midrule
\multirow{2}{6em}{\raggedleft \hangpara{1.3em}{1}\eIIoutgoodabbr/} &%
 & \tesig{A}{$\bm{6.8\sqtimes 10^{-11}}$}{88\%} &%
 & \tesig{A}{$\bm{1.5\sqtimes 10^{-6}}$}{79\%} \\
&%
 & \tesig{A}{$\bm{3.4\sqtimes 10^{-7}}$}{84\%} &%
 & \tesig{A}{$\bm{4.2\sqtimes 10^{-6}}$}{86\%} \\
\midrule
\multirow{2}{6em}{\raggedleft \hangpara{1.3em}{1}\eIIoutbadabbr/} &%
 & \tesig{D}{$\bm{6.7\sqtimes 10^{-13}}$}{90\%} &%
 & \tesig{D}{$\bm{9.2\sqtimes 10^{-7}}$}{80\%} \\
&%
 & \tesig{D}{$\bm{1.3\sqtimes 10^{-7}}$}{84\%} &%
 & \tesig{D}{$\bm{4.2\sqtimes 10^{-6}}$}{86\%} \\
\midrule
\multirow{2}{6em}{\raggedleft \hangpara{1.3em}{1}\eIIouthappyabbr/} &%
 & \tesig{A}{$\bm{1.7\sqtimes 10^{-10}}$}{88\%} &%
 & \tesig{A}{$\bm{1.4\sqtimes 10^{-7}}$}{82\%} \\
&%
 & \tesig{A}{$\bm{4.7\sqtimes 10^{-8}}$}{85\%} &%
 & \tesig{A}{$\bm{1.5\sqtimes 10^{-6}}$}{87\%} \\
\midrule
\multirow{2}{6em}{\raggedleft \hangpara{1.3em}{1}\eIIoutregretabbr/} &%
 & \tesig{D}{$\bm{2.2\sqtimes 10^{-10}}$}{88\%} &%
 & \tesig{D}{$\bm{4.8\sqtimes 10^{-7}}$}{81\%} \\
&%
 & \tesig{D}{$\bm{1.1\sqtimes 10^{-6}}$}{82\%} &%
 & \tesig{D}{$\bm{1.5\sqtimes 10^{-6}}$}{87\%} \\
\midrule
\multirow{2}{6em}{\raggedleft \hangpara{1.3em}{1}\eIIoutexpectedabbr/} &%
 & \tesig{A}{0.011}{66\%} &%
 & \tensig{D}{0.795} \\
&%
 & \tesig{A}{$\bm{3.0\sqtimes 10^{-4}}$}{75\%} &%
 & \tensig{D}{0.996} \\
\midrule
\multirow{2}{6em}{\raggedleft \hangpara{1.3em}{1}\eIIoutunexpectedabbr/} &%
 & \tesig{D}{$\bm{6.6\sqtimes 10^{-6}}$}{78\%} &%
 & \tensig{A}{0.896} \\
&%
 & \tesig{D}{$\bm{3.0\sqtimes 10^{-4}}$}{75\%} &%
 & \tensig{A}{0.997} \\
\bottomrule
\end{tabular}

\egroup
\caption[Retrospective positive outcome results]{TODO: HERE}
  \label{tab:e2-positive-outcome-results}
\end{table}

\begin{table}[!p]
\centering
\bgroup
\def\arraystretch{1.3}
\setlength{\tabcolsep}{0.4em}
\begin{tabular}{l  c  c c c  c  c c c}
\toprule
\multirow{2}{5em}{Question} &%
 & \multicolumn{3}{c}{Expected Failure} &%
 & \multicolumn{3}{c}{Unexp. Failure} \\
&%
 & \multicolumn{3}{c}{Obv. Success [alt]} &%
 & \multicolumn{3}{c}{Obv. Failure [main]} \\
\toprule
\midrule
\multirow{2}{8em}{\hangpara{1.3em}{1}\sIIoutfairabbr/} &%
 & \tesig{A}{0.001}{68\%} &%
 & \tensig{D}{0.343} \\
&%
 & \tensig{A}{0.274} &%
 & \tensig{D}{0.460} \\
\midrule
\multirow{2}{8em}{\hangpara{1.3em}{1}\sIIoutunfairabbr/} &%
 & \tesig{D}{0.017}{63\%} &%
 & \tensig{A}{0.634} \\
&%
 & \tesig{D}{0.024}{65\%} &%
 & \tensig{A}{0.419} \\
\midrule
\multirow{2}{8em}{\hangpara{1.3em}{1}\sIIoutsenseabbr/} &%
 & \tesig{A}{$\bm{1.2\sqtimes 10^{-4}}$}{72\%} &%
 & \tensig{D}{0.716} \\
&%
 & \tesig{A}{0.012}{67\%} &%
 & \tensig{D}{0.508} \\
\midrule
\multirow{2}{8em}{\hangpara{1.3em}{1}\sIIoutbrokenabbr/} &%
 & \tesig{D}{$\bm{4.2\sqtimes 10^{-4}}$}{70\%} &%
 & \tensig{A}{0.981} \\
&%
 & \tesig{D}{$\bm{6.2\sqtimes 10^{-4}}$}{74\%} &%
 & \tensig{A}{0.783} \\
\midrule
\multirow{2}{8em}{\hangpara{1.3em}{1}\sIIoutgoodabbr/} &%
 & \tensig{D}{0.129} &%
 & \tesig{D}{$\bm{2.6\sqtimes 10^{-5}}$}{73\%} \\
&%
 & \tensig{D}{0.519} &%
 & \tesig{D}{0.006}{67\%} \\
\midrule
\multirow{2}{8em}{\hangpara{1.3em}{1}\sIIoutbadabbr/} &%
 & \tensig{A}{0.762} &%
 & \tesig{A}{$\bm{1.7\sqtimes 10^{-4}}$}{71\%} \\
&%
 & \tensig{A}{0.919} &%
 & \tesig{A}{0.014}{65\%} \\
\midrule
\multirow{2}{8em}{\hangpara{1.3em}{1}\sIIouthappyabbr/} &%
 & \tenp  &%
 & \tesig{D}{0.024}{62\%} \\
&%
 & \tenp  &%
 & \tensig{D}{0.809} \\
\midrule
\multirow{2}{8em}{\hangpara{1.3em}{1}\sIIoutregretabbr/} &%
 & \tenp  &%
 & \tesig{A}{$\bm{3.7\sqtimes 10^{-4}}$}{70\%} \\
&%
 & \tenp  &%
 & \tensig{A}{0.897} \\
\midrule
\multirow{2}{8em}{\hangpara{1.3em}{1}\sIIoutexpectedabbr/} &%
 & \tesig{A}{0.031}{61\%} &%
 & \tesig{D}{$\bm{9.5\sqtimes 10^{-4}}$}{68\%} \\
&%
 & \tensig{A}{0.129} &%
 & \tensig{D}{0.168} \\
\midrule
\multirow{2}{8em}{\hangpara{1.3em}{1}\sIIoutunexpectedabbr/} &%
 & \tesig{D}{$\bm{2.9\sqtimes 10^{-4}}$}{70\%} &%
 & \tensig{A}{0.184} \\
&%
 & \tesig{D}{0.036}{64\%} &%
 & \tensig{A}{0.313} \\
\bottomrule
\end{tabular}

\egroup
\caption[Retrospective negative outcome results]{TODO: HERE}
  \label{tab:e2-negative-outcome-results}
\end{table}

\begin{table}[!p]
\centering
\bgroup
\def\arraystretch{1.3}
\setlength{\tabcolsep}{0.6em}
\begin{tabular}{r c c c}
\toprule
Question & Hypothesis & $p$-value & Effect \\
\midrule
\eIIoutfairabbr/ & \tensig{unexp. failure$>$obv. failure [main]}{0.640} \\
\eIIoutunfairabbr/ & \tensig{unexp. failure$<$obv. failure [main]}{0.292} \\
\eIIoutsenseabbr/ & \tensig{unexp. failure$>$obv. failure [main]}{0.339} \\
\eIIoutbrokenabbr/ & \tensig{unexp. failure$<$obv. failure [main]}{0.164} \\
\eIIoutgoodabbr/ & \tensig{unexp. failure$>$obv. failure [main]\lc/}{0.916} \\
\eIIoutbadabbr/ & \tensig{unexp. failure$<$obv. failure [main]\lc/}{0.902} \\
\eIIouthappyabbr/ & \tesig{unexp. failure$<$obv. failure [main]\lc/}{$\bm{7.7\sqtimes 10^{-4}}$}{75\%} \\
\eIIoutregretabbr/ & \tesig{unexp. failure$>$obv. failure [main]\lc/}{$\bm{3.7\sqtimes 10^{-5}}$}{80\%} \\
\eIIoutexpectedabbr/ & \tensig{unexp. failure$>$obv. failure [main]\lc/}{0.945} \\
\eIIoutunexpectedabbr/ & \tensig{unexp. failure$<$obv. failure [main]\lc/}{0.539} \\
\bottomrule
\end{tabular}

\egroup
\caption[Retrospective free vs. forced failure results]{TODO: HERE}
  \label{tab:e2-free-vs-forced-failure-results}
\end{table}

\begin{table}[!p]
\centering
\bgroup
\def\arraystretch{1.3}
\setlength{\tabcolsep}{0.6em}
\begin{tabular}{l c c c}
\toprule
Question & Hypothesis & $p$-value & Effect \\
\midrule
\sIIoutgoodabbr/ & \tensig{exp. success$<$obv. success [main]}{0.942} \\
\sIIoutbadabbr/ & \tensig{exp. success$>$obv. success [main]}{1.000} \\
\sIIouthappyabbr/ & \tensig{exp. success$<$obv. success [main]}{0.725} \\
\sIIoutregretabbr/ & \tensig{exp. success$>$obv. success [main]}{0.791} \\
\bottomrule
\end{tabular}

\egroup
\caption[Retrospective chosen vs. inevitable success results]{TODO: HERE}
  \label{tab:e2-chosen-vs-inevitable-success-results}
\end{table}

\begin{table}[!p]
\centering
\bgroup
\def\arraystretch{1.3}
\setlength{\tabcolsep}{0.6em}
\begin{tabular}{l c c c}
\toprule
Question & Hypothesis & $p$-value & Effect \\
\midrule
\sIIoutfairabbr/ & \tesig{unexp. failure$<$unexp. success}{$\bm{4.5\sqtimes 10^{-9}}$}{82\%} \\
\sIIoutunfairabbr/ & \tesig{unexp. failure$>$unexp. success}{$\bm{5.7\sqtimes 10^{-5}}$}{72\%} \\
\sIIoutsenseabbr/ & \tesig{unexp. failure$<$unexp. success}{$\bm{3.7\sqtimes 10^{-4}}$}{70\%} \\
\sIIoutbrokenabbr/ & \tesig{unexp. failure$>$unexp. success}{0.004}{65\%} \\
\bottomrule
\end{tabular}

\egroup
\caption[Retrospective good vs. bad unexpected results]{TODO: HERE}
  \label{tab:e2-good-vs-bad-unexpected-results}
\end{table}

\begin{table}[!p]
\centering
\bgroup
\def\arraystretch{1.3}
\setlength{\tabcolsep}{0.6em}
\begin{tabular}{l c c c}
\toprule
Question & Hypothesis & $p$-value & Effect \\
\midrule
\sIIoutgoodabbr/ & \tesig{unexp. failure$<$exp. failure}{$\bm{3.0\sqtimes 10^{-4}}$}{70\%} \\
\sIIoutbadabbr/ & \tesig{unexp. failure$>$exp. failure}{$\bm{3.3\sqtimes 10^{-5}}$}{74\%} \\
\sIIouthappyabbr/ & \tesig{unexp. failure$<$exp. failure}{0.006}{65\%} \\
\sIIoutregretabbr/ & \tesig{unexp. failure$>$exp. failure}{$\bm{2.6\sqtimes 10^{-5}}$}{74\%} \\
\sIIoutgoodabbr/ & \tesig{obv. failure [main]$<$exp. failure}{0.040}{62\%} \\
\sIIoutbadabbr/ & \tesig{obv. failure [main]$>$exp. failure}{0.004}{68\%} \\
\sIIouthappyabbr/ & \tensig{obv. failure [main]$<$exp. failure}{0.783} \\
\sIIoutregretabbr/ & \tensig{obv. failure [main]$>$exp. failure}{0.815} \\
\bottomrule
\end{tabular}

\egroup
\caption[Retrospective expected vs. unexpected failure results]{TODO: HERE}
  \label{tab:e2-expected-vs-unexpected-failure-results}
\end{table}

\begin{table}[!p]
\centering
\bgroup
\def\arraystretch{1.3}
\setlength{\tabcolsep}{0.6em}
\begin{tabular}{l c c c}
\toprule
Question & Hypothesis & $p$-value & Effect \\
\midrule
\sIIoutgoodabbr/ & \tensig{exp. success$<$unexp. success}{0.954} \\
\sIIoutbadabbr/ & \tensig{exp. success$>$unexp. success}{1.000} \\
\sIIouthappyabbr/ & \tensig{exp. success$<$unexp. success}{0.961} \\
\sIIoutregretabbr/ & \tensig{exp. success$>$unexp. success}{0.960} \\
\sIIoutgoodabbr/ & \tensig{obv. success [main]$<$unexp. success}{0.633} \\
\sIIoutbadabbr/ & \tensig{obv. success [main]$>$unexp. success}{0.747} \\
\sIIouthappyabbr/ & \tensig{obv. success [main]$<$unexp. success}{0.847} \\
\sIIoutregretabbr/ & \tensig{obv. success [main]$>$unexp. success}{0.750} \\
\bottomrule
\end{tabular}

\egroup
\caption[Retrospective expected vs. unexpected success results]{TODO: HERE}
  \label{tab:e2-expected-vs-unexpected-success-results}
\end{table}


\begin{figure}[!p]
  % TODO: Consistent colors between graphs in this table!
  \begin{tabular}{>{\hspace*{-4pt}}c@{} c@{}}
\sffamily Motives for Deciding & \sffamily Consistency of Motives \\[0.5em]
\includegraphics[width=0.495\textwidth]{{fig/motives-motives}.pdf}&%
\includegraphics[width=0.495\textwidth]{{fig/motives-consistency}.pdf} \\[1em]
\sffamily Reasons for Positive Evaluations & \sffamily Reasons for Negative Evaluations \\[0.5em]
\includegraphics[width=0.495\textwidth]{{fig/motives-judge.good}.pdf} &
\includegraphics[width=0.495\textwidth]{{fig/motives-judge.bad}.pdf} \\
  \end{tabular}
  \caption[Retrospective study motive results]{Histograms of responses to the four motive questions.}
  \label{fig:e2-motive-results}
\end{figure}
