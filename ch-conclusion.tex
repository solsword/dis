\chapter{Conclusion}

\label{ch:conclusion}

Throughout the previous three chapters which describe the theory of choice poetics, the design of \dunyazad/, and results from experiments that test its capabilities, there have been mentions of ``future work;'' of unexplored paths.
%
The theory developed in \cref{ch:choice-poetics} is incomplete and shallow; it doesn't directly address some of the most interesting poetic effects of choices, such a regret.
%
The system description on \cref{ch:dunyazad} admits that \dunyazad/ is focused on single choices for now, and doesn't have strong mechanisms for guiding plot arcs.
%
The experimental results detailed in \cref{ch:option-results,ch:outcome-results} are mixed, and highlight several areas that need work.
%
Taken together, this perhaps gives the impression that \dunyazad/ as a project is woefully incomplete.


However, recall the purpose of \dunyazad/ as a project: learn new things about choice poetics using a hybrid research method that incorporates artificial intelligence into a humanistic agenda, while producing a novel generative system centered on choices.
%
Measured against these goals, the theoretical and technical results that have been demonstrated so far represent an important first step in a new direction.
%
The ideas about choice poetics described in \cref{ch:choice-poetics}, for example, are not ideas that would have resulted from a close reading of multiple choice-based narratives, and this is also the case for the caveats explored in \cref{ch:option-results,ch:outcome-results}.
%
Furthermore, \dunyazad/ as a generative system does things that no other existing system attempts, giving users a uniquely detailed level of control over the choices that it generates: it is a system that attempts to understand and provide direct control over the impressions that its choices make, and it is largely successful at this.
%
Finally, the experimental results from \cref{ch:option-results,ch:outcome-results} are something that would not have been possible without \dunyazad/: based on \dunyazad/'s transparent operationalization of choice poetics, they are able to establish a direct link between assessments that result from particular ways of reasoning about choices and audience reactions to those choices.
%
Knowledge of where \dunyazad/'s reasoning breaks down is in fact just as interesting as knowledge of where it succeeds, and is the main benefit of the use of AI in this project: without a technical implementation of the theory of choice poetics, these hidden assumptions would have gone unnoticed.


Viewed in this light, I'm happy with the results of the project so far.
%
It has broken new theoretical and technical ground, backed up by rigorous experimental results.
%
The following sections summarize the key results of the \dunyazad/ project.

\section{A Theory of Choice Poetics}

\Cref{ch:choice-poetics} describes both a broad foundation for understanding the poetics of choices and a detailed procedure for analyzing choices as they related to player goals.
%
It provides language for talking about explicit, discrete choices in terms of \emph{framing}, \emph{options}, and \emph{outcomes}, and it introduces the notion of modes of engagement, which help conceptualize players' sometimes complex motivations.
%
When discussing poetic effects, it goes beyond well-established aspects of player experience like agency and touches on effects like autonomy, responsibility, and regret.
%
Finally, \cref{ch:choice-poetics} describes in detail a technique for analyzing choices based on player goals; this is the main theoretical contribution of this chapter.


By mechanically breaking down a player's experience of a choice into formal components such as `player goals' and `outcome evaluations,' goal-based choice analysis seeks to expose hidden connections and account for a range of possible player motives.
%
While a na\"ive evaluation of a choice might be able to come up with the most common reaction and predict which option most players would choose, the detailed dissection involved in goal-based choice analysis helps understand how alternative motives might color some player's perceptions and exposes the perceived trade-offs between options.
%
For authors of branching narratives, goal-based choice analysis should be a useful tool for viewing a choice from multiple player perspectives, and it may be helpful when ``debugging'' a choice that playtesting has identified as provoking unexpected reactions.
%
For critics, goal-based choice analysis ought to help go from a gut reaction to an explicit description of why a particular choice is effective or ineffective for a particular purpose.


The theory developed in \cref{ch:choice-poetics} is based on existing theories of traditional and interactive poetics along with craft advice, as described in \cref{ch:related-work}.
%
But what sets it apart from similar theories is that it was developed in tandem with a generative system.
%
It is intentionally more mechanical to allow for easy operationalization, and this also makes it easy to incorporate feedback from experimental results.


\section{A Choice-Point Generator}

The main technical result of the \dunyazad/ project is a program that generates narrative choices.
%
While any interactive narrative project necessarily gives the player some kind of choices, until now there have only been a few projects that focused on those choices as a first-class design problem: most projects generate (or manage) some other structure which implicitly contains choices without ever focusing on the total set of options available to the player at each moment.
%
As the results in \cref{ch:option-results,ch:outcome-results} demonstrate, however, this total set of options can profoundly impact player reactions, and even options which are never chosen by the player color their perception of the outcomes that they do see.
%
While a few other projects such as \citep{Barber2007a, Yu2013} have made explicit choices and their structures the center of their attention, these projects have not attempted to generate a variety of choices in a principled manner.
%
Relative to existing work, then, \dunyazad/ stands out as a project that aims to be able to generate a range of different choice structures from first principles.


\dunyazad/'s reliance on an operationalization of choice poetics is key here.
%
The same results in terms of measurable audience impact could likely have been achieved using a variety of approaches.
%
Certainly human authors could have been employed to write choices given similar criteria (e.g., ``make it have an obvious best option'').
%
What is the advantage of using a computer for this task (or of using answer-set programming as opposed to some other technique)?
%
The answer lies in the other half of the hybrid research approach: all of these alternative techniques, including paying human authors, could have achieved equivalent audience impacts, but none of them offer the same level of insight into the mechanisms behind that impact.


Because \dunyazad/ uses answer set programming, both its successes and failures can be reliably understood in terms of the logical rules of implication that it operates over, and these rules directly correspond to tenets of the choice poetic theory that it was developed in dialogue with.
%
In other words, \dunyazad/ can establish a sufficiency relationship between the statements of choice poetic theory and the effects that they purport to describe.
%
By generating choices which audiences perceive as, say, ``obvious,'' \emph{using only the rules of choice poetics to guide it}, \dunyazad/ is a procedural proof that those rules are \emph{sufficient} to give rise to ``obviousness,'' at least in the story contexts that \dunyazad/ operates in.
%
Likewise, when a choice is labelled ``obvious'' but humans don't perceive it as such, \dunyazad/ has demonstrated that its rules are insufficient as an explanation of what ``obviousness'' is, and often, the human perceptions that do result can explain \emph{why} this is the case in terms, again, of the theoretical rules of choice poetics.
%
Few other approaches to generating choices, computational or otherwise, have the same potential to generate experimental results which can directly inform the underlying theory.
%
\dunyazad/ is thus not only an example of a successful system that explores a novel generative space, it is also a unique tool for the study of choice poetics, and a demonstration of a new research approach that employs artificial intelligence as a tool for the development of a humanistic theory.


\section{Experimental Results}

The experimental results presented in \cref{ch:option-results,ch:outcome-results} are the linchpin of the hybrid research approach that drives \dunyazad/.
%
These results establish, via \dunyazad/'s rules, a direct link between actual human reactions to choices and the theory of choice poetics which attempts to predict those reactions.
%
\Cref{tab:prospective-impressions-redux,tab:retrospective-impressions-redux} at the very end of \cref{ch:outcome-results} show the results of this feedback: caveats and amendments to the analysis method described in \cref{ch:choice-poetics} that resulted from an analysis of experimental data.


With the participation of hundreds of paid subjects contacted via Amazon Mechanical Turk, the options and outcomes that \dunyazad/ generates were evaluated in two experiments, focused on immediate reactions such as obviousness or the valence of outcomes.
%
\dunyazad/ generated choices for these experiments by following the implications of simple constraints, such as ``make this choice obvious'' within the body of choice poetics theory that it is programmed with.
%
Because of this, unexpected results could be traced back to the rules that allowed them.
%
Critically, \dunyazad/'s inhuman exploration of the possibilities allowed by its rules discovered edge cases that are hidden by human assumptions, such as the inability of straightforward goal analysis to account for certain kinds of moral reasoning.
%
These edge cases would probably not have been discovered by a human attempt to work within the same constraints, because a human instructed to ``create an obvious choice'' would unconsciously rules out structures inconsistent with moral reasoning without being told to do so.


While the experiments did reveal that \dunyazad/ needs more work to consistently produce the desired low-level poetics, they also confirmed that it was largely successful at generating specific types of choices.
%
The flaws that the results exposed in \dunyazad/'s reasoning have also served to directly inform the theory of choice poetics from \cref{ch:choice-poetics}.
%
For example, \dunyazad/'s difficulty in reasoning about moral conflicts of interest (between e.g., self-preservation and helping an innocent victim in need) shows that when humans apply the goal-based choice analysis method presented in \cref{sec:goal-based-choice-analysis} they should be careful to exercise their capacity for moral reasoning.
%
This result, and the others like it presented in \cref{ch:option-results,ch:outcome-results}, are the real fruits of the hybrid research method employed in \dunyazad/.
%
By using \dunyazad/'s inhuman reasoning to double-check the human reasoning behind the theory of choice poetics, that theory can be improved and developed more thoroughly than by using human reasoning alone.


\section{A Hybrid Research Process}

The hybrid research approach described here is not just applicable to choice poetics.
%
I hope that the success of \dunyazad/ as a project can inspire other researchers to find value in the application of artificial intelligence as a tool for research in the humanities and social sciences.
%
Beyond looking for patterns in large amounts of data or running intricate simulations designed to model real-world phenomenon, computers can use symbolic reasoning to explore logical theories along inhuman lines of inquiry.
%
As frustrating as this can be at times, the laborious process of explicitly encoding human assumptions about how things work tends to reveal hidden biases and assumptions that were taken-for-granted before a computer without them attempted to put a theory to use.
%
By putting a computer's output in front of other humans for judgement, a feedback loop can be formed, allowing the computer's ``mistakes'' to expose the underlying assumptions of a theory's author, thus informing the theory and the next batch of generated results.


\section{Future Work}

As a generative system, despite not being able to produce entire gamebooks at a command, \dunyazad/ can produce interesting choices, and it has a big enough generative space to occasionally surprise authors without disappointing them (although some of its surprises, like those of most generative systems, are disappointing).
%
Perhaps more importantly, the choices that it generates are complex and consistent enough to reveal edge-cases of the theory that drives it, and not just question its implementation of that theory.
%
Given these successes, there are multiple directions that the project could go in.


First, the flaws revealed by the experimental data should be addressed.
%
This involves implementing relative value judgements as well as moral reasoning (perhaps along the lines of the defeasible reasoning approach presented by Joseph Blass and Ian Horswill in \citep{Blass2015}).
%
\dunyazad/ will also need to implement a more complicated approach to goal priorities, and this will have to be backed up by an understanding of how players approach conflicting goals: the theory of choice poetics will need to become more specific on this point.
%
As with the functionality that \dunyazad/ already has, once the theory and system have evolved, they should be tested in an experimental study, and further revised based on the results.


Besides developing \dunyazad/'s core reasoning capabilities, it should also be extended to reason about larger plot structures.
%
Currently, \dunyazad/ can chain together choices to play out a single scene and transition from scene to scene by moving the action to a new location, but it lacks high-level reasoning about plot structure and certain elements like recurring characters from scene-to-scene that would enable it to generate longer stories.
%
Implementing these things could enable the study of more complex poetic effects like regret, although of course these effects also provide challenges in terms of experimental design.
%
Having \dunyazad/ generate full stories would also open up new applications: at that point, it could be studied as a full interactive narrative experience in its own right.


Finally, the theory of choice poetics should be put to use to gauge its effectiveness in analyzing human-constructed narrative choices.
%
Not only should it enable more detailed and explicit analysis of works like Choose-Your-Own-Adventure books and visual novels, but the act of applying it will result in a different kind of feedback that can drive the development of the theory.
%
This in turn could expose new avenues for the growth of \dunyazad/ as a generative system, and in turn.
%
Ultimately, both the system and the theory should benefit from the theory's application.


All of this work fits into a broader research agenda of developing generative AI systems that can both push the boundaries of what is possible with interactive media and simultaneously become sites of study to understand how it works.
%
As digital media continue to impact our society, from Facebook's filtering algorithms to the online virtual environments that have become important spaces for socialization and growth for teens and young adults, I hope to see more projects like \dunyazad/ that use computers not just to create these environments but to help us make sense of them and understand how they impact the humans that use them.
