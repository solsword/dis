\chapter{English Generation}
\label{ap:nlg}%

Once \dunyazad/'s iterative story generation has finished filling in the timepoints of a story with predicates that represent states, events, and choices, all of that information needs to be made playable.
%
\dunyazad/ uses a template-based natural-language generation system to turn its predicate statements representing story actions into text (and choice interface elements).
%
The obvious format is hypertext, but some practical problems and even some real generative work are entailed by the transformation from predicates to text.
%
\dunyazad/ doesn't yet do anything groundbreaking in terms of discourse, but the experiments that it has been used in do depend on the text generation capabilities described here.


\dunyazad/ uses simple grammar-based text generation with small pre-defined grammars for each setup, action and potential (special world states that motivate actions).
%
Beyond this it knows how to describe the initial state of the world, and has pre-defined introduction and epilogue grammars.
%
In the expansion of these grammars to create text, it can substitute variables but also knows some simple conjugation rules to enforce subject-verb agreement and it does automatic pronominalization to help the text flow together as a whole.

\section{Grammar Expansion}

The basic unit of discourse content in \dunyazad/ is a grammar expansion, which looks like this:
\begin{quote}
\ttfamily
:action/attack/option \\
N\#?aggressor/they V\#attack/prs/?aggressor N\#?target/them
\end{quote}

\noindent \ldots and might be rendered into this:

\begin{quote}
``you attack the bandits''
\end{quote}

\noindent \ldots or this:

\begin{quote}
``the merchant attacks you''
\end{quote}

The first line, which begins with a \exchar{:} character, indicates a substitution key (a non-terminal in the grammar), and another grammar rule could include an expansion using this key, which would look like this:

\begin{quote}
  \ttfamily
  [[action/attack/option]]
\end{quote}

Although not required, path-like key names like this example help organize the grammar and can be matched over using some wildcards (see below).
%
After a key definition, one or more expansions for that key are given on the following lines, and when an expansion is made, one of those will be selected at random.
%
Besides the key of the desired expansion, a grammar expansion can optionally contain flags and variable assignments, like this:

\begin{quote}
  \ttfamily
  [[S|misc/you\_ask\_for@statement=[[action/?\_action/option]]]]
\end{quote}

Flags are capital letters given at the beginning of the expansion followed by a \exchar{|} character.
%
The only valid flag is currently `S' for `sentence' which after expansion capitalizes the first letter of the result and adds a period to the end.
%
Variable assignments are given by an \exchar{@} symbol followed by \prq{<var>=<value>}{.}
%
Any number of variable assignments may be given and, as shown, both variables
and grammar expansions can be nested in each other.
%
In this example, the key \prq{misc/you\_ask\_for}{} will be looked up and a random expansion for it will be chosen.
%
Within text generated by that expansion and its children (if any), the variable \prq{statement}{} will take on the literal value \prq{[[action/?\_action/option]]}{.}
%
If the expansion text includes a reference to that variable, of course, when the variable is substituted a further grammar expansion will be present, but the exact key to be looked up will depend on the value of the variable \prq{\_action}{} at the time of the substitution of the \prq{statement}{} variable.
%
(More precisely, when the string \prq{?statement} is processed as a variable within text that results from this example grammar expansion, the result will include the string \prq{?\_action} which will be subsequently processed as a variable before any further grammar expansion.)


\section{Variable Expansion}

As already indicated, variable substitutions happen concurrently with grammar substitutions.
%
A variable substitution is indicated by a question mark followed by a variable name, these must be valid Python identifiers (generally lowercase strings that may include underscores).
%
Different variables are supplied by \dunyazad/ in different contexts, and grammar expansions may also manually assign variables as shown above.


The key in the first example above is \prq{action/\abr/attack/\abr/option}{,} which is used when \dunyazad/ needs to display option text (the text that the player will click on to trigger the action that follows the standard ``What do you do?'' prompt) for an \prq{attack}{} action.
%
In this context, \dunyazad/ runs the grammar with all of the arguments of the current option's action bound as variables, so because the \prq{attack}{} action has \prq{aggressor}{} and \prq{target}{} arguments, those variables names can be used.
%
Other default variables like \prq{\_action}{} which identifies the current action are supplied, in the second example above this is used to switch between grammar keys in an expansion.
%
Variables for actors and items expand to the unique identifier for the actor/item in question, so \prq{?target}{} might expand to \prq{bandits\_17}{} during text generation.


Variables and grammar expansions may both contain each other because their expansion is interleaved.
%
When expanding a grammar element, the first step is to scan for all variable substitutions and replace them with their values.
%
This step uses a while loop, so variables whose values contain variable expansions will keep expanding until there are no expansions left (setting a variable's value to an expansion for itself would cause an infinite loop).
%
Once there are (apparently) no more variables to substitute, the text is parsed into a list of static text elements separated by grammar expansions.
%
In text order, the grammar expansions are then recursively expanded one-by-one (making for a depth-first expansion policy, although the grammar is context-free, so this matters little).
%
The first step in each of these recursive calls, of course, is the exhaustive variable substitution mentioned above.


The net result is that whenever a variable is substituted which contains a grammar expansion, that expansion will be considered afterwards, and when a grammar expansion contains a variable, it will be substituted before any sub-expansions occur.
%
The exhaustive variable substitution rule means that if a variable contains a grammar expansion with a variable key (as in the example above), the key variable will be substituted with its value before the grammar expansion is even parsed.
%
Nested grammar expansions, on the other hand, are not allowed (in part simply to reduce parser complexity); this is one categorical difference between grammar expansions and variables.
%
Another is that all grammar key-expansion relationships are assigned at compile time (technically, when the English generation system starts up), while variable assignments can change at runtime.


\section{Tags}

The extra syntax that surrounds the variable expansions in the \prq{action/\abr/attack/\abr/option}{} example above is \dunyazad/'s approach to managing subject/verb agreement and pronominalization.
%
These are called `tags,' and they indicate a third, more dynamic type of substitution.
%
Whereas variable substitution and grammar expansion happen concurrently, tag expansions happen only after the other two types are exhausted, and are scoped in terms of linear text rather than the tree of grammar expansions.
%
Tags allow \dunyazad/ to look up all of the properties of an entity (an actor or item) and use this information to get the text right.
%
Their two main purposes are dynamically substituting noun references which may be pronouns, and conjugating verbs while ensuring subject-verb agreement.


In the first example above, notice how ``attack'' changes to ``attacks'' to agree with the singular verb ``the merchant'' in the second example result.
%
Because the verb tag (\prq{V\#}{}) contains a reference to the subject of the sentence (\prq{?aggressor}{}), it knows whether the noun it must agree with is singular or plural.
%
Of course, forcing the author to manually specify which noun each verb agrees with is a significant burden, but integrating automatic analysis that could determine this correctly most of the time proved beyond the scope of the project (there are existing solutions to this problem).
%
Besides noun agreement, verb tags are also used for conjugation, and thus specify a tense as well as an agreeing entity (\prq{prs}{} in this case to indicate present-tense).


Noun tags (\prq{N\#}{}) meanwhile include both an entity reference and a third person plural pronoun.
%
These pronouns are used as a shorthand to identify the case and position of noun use, because the third-person plural pronouns are unique for each case/position (`they,' `them,' `their,' `theirs,' `themselves').
%
The rest of the information necessary to figure out an appropriate pronoun (person, number, and gender) is gleaned from the entity reference.
%
All of this information can be used to refer to an entity like a normal English speaker would instead of always using its full proper name.


Because \dunyazad/ often needs to assemble paragraphs from sentences that are written independently from one another, it's impossible for the author to specify which noun instances should be full references or pronouns at the discourse level.
%
As an example of this, consider a paragraph that uses the same nouns multiple times with just a single form of address:

\begin{quote}
  As you travel onwards, you come across the bandits threatening the merchant. You defeat the bandits, and the bandits are injured. The bandits are no longer threatening the merchant.
\end{quote}

The same paragraph using proper pronominalization and introduction flows much better, even if the sentences are still a bit choppy:

\begin{quote}
  As you travel onwards, you come across some bandits threatening a merchant. You defeat the bandits, and they are injured. They are no longer threatening the merchant.
\end{quote}

Especially if each sentence is authored in a different file and the final sentence comes from a grammar key that is used in multiple different contexts, the author can't determine when writing the underlying grammar fragments which nouns should be indefinite or pronominalized.
%
The solution to this problem is to let \dunyazad/ handle introduction and pronominalization and just have the author indicate what pronoun would be used in each situation if one is necessary.
%
\dunyazad/ detects the introduction of new entities, inserting ``a merchant'' instead of ``the merchant'' the first time a specific merchant is encountered, and it uses a table of recently-used nouns to insert pronouns when they are unambiguous.


The system for pronominalization is fairly simple: it keeps track of the last noun used which would be a referent for three different pronoun `slots:' `he/she,' `it,' and `they.'
%
When a noun is used, if it matches the latest referent for its slot, a pronoun is used instead; otherwise, the information in that slot is marked as confounded.
%
After an adjustable number of nouns pass without renewal, old slots are cleared so that new information can be stored.
%
Only when a slot is empty is new information stored upon encountering a noun that fits it.
%
These rules are an attempt to avoid confusing situations where, for example, the word `it' can potentially refer to two different objects.
%
To complement this system, the slot values are forcibly cleared at certain points (such as between setup text and action text) to avoid using too many pronouns (note the occurrence of `the bandits' in the pronominalized text above instead of `them,' which would sound a bit weird).



Besides verb and noun tags, the third and final type of tag is the directive tag (\prq{D\#}{}), which is used for the \prq{timeshift}{} directive.
%
Tense specifications included in verb tags make it possible in simple cases to automatically alter the tense of a statement, as seen in one expansion for the key \prq{misc/you\_ask\_for}{}:

\begin{quote}
\ttfamily
N\#you/they V\#ask/prs/you \\
\ind D\#timeshift/infinitive@@?statement@@D\#timeshift/pop@@
\end{quote}

Timeshift is the only valid directive, and it takes one argument: a tense to shift into, or the special value `pop,' which returns to the previous tense (tense shifts are maintained in a stack, with the top shift overriding all below it).
%
In this example, it is used to take option text that would be given as the action of a secondary protagonist and turn it into a request from the main character for the secondary character to perform that action (in order to maintain second-person focalization).
%
Recall that the \prq{statement}{} variable will have a value which looks like \prq{[[actions/attack/option]]}{} because of the forced assignment in the \prq{misc/you\_ask\_for}{} expansion shown above (assuming that \prq{\_action}{} has the value \prq{attack}{}).
%
In this case, which is chosen by the Python code when an option represents the action of a protagonist other than the main character, the \prq{aggressor}{} variable will hold the identifier of the acting character.
%
If the acting character is named Peter, the innermost grammar substitution would therefore normally produce text along the lines of:

\begin{quote}
  Peter attacks the bandits.
\end{quote}

However, that text has been placed between the two \prq{timeshift}{} directives shown above, and so it will be shifted from the present tense (as specified in the original verb tag) into the infinitive tense, resulting in:

\begin{quote}
  Peter to attack the bandits.
\end{quote}

Of course, the outer noun and verb tags will accompany it, so the end result will be:

\begin{quote}
  You ask Peter to attack the bandits.
\end{quote}

The advantage of the \prq{timeshift}{} directive is the ability to represent ``asking-for'' a task and individual task descriptions separately, as opposed to having to write every single task with both a ``direct'' and an ``asked-for'' version.
%
Although this doesn't happen in \dunyazad/, it could also be used to reframe narratives, for example, by having a character tell another character about events that happened to them in the past tense (the perspective change involved would also be automatic because of automatic pronominalization). 


\section{Conditional Expansions}

There are three more mechanisms that \dunyazad/'s English generator makes use of in special cases.
%
The first are conditional expansions: by placing a line which starts with \exchar{\{} and ends with \exchar{\}} where an expansion would normally go, all following lines defining expansions for the current key are made conditional.
%
Conditional expansions are only considered valid expansions for a key when the current story facts contain a set of predicates which taken together produces a match for each predicate listed on the condition line (separated by semicolons).
%
So for example, if the following grammar rules are defined:

\begin{quote}
  \ttfamily
  :color \\
  \{favorite\_color(X);warm(X)\} \\
  a warm ?\_X \\
  \{favorite\_color(X);cool(X)\} \\
  a cool ?\_X
\end{quote}

\noindent \ldots and the predicates representing the current story were these:

\begin{quote}
  \ttfamily
  color(red). \\
  color(blue). \\
  color(yellow). \\
  warm(red). \\
  warm(yellow). \\
  cool(blue). \\
  favorite\_color(blue).
\end{quote}

\noindent \ldots then expansion of the grammar \prq{[[color]]}{} would result in the text ``a cool blue.''
%
If there are multiple predicate sets that can match a condition, \dunyazad/ considers each possible match as a separate expansion possibility.
%
Internally, \dunyazad/ parses each precondition and looks for matches in the fact set, allowing uppercase names to take on any value (even a compound predicate).
%
It proceeds to bind any uppercase names to the values required by the match(es) found and, for each match, recursively attempts to match the rest of the preconditions subject to that binding.
%
The end result works a bit like variables in ASP constraints, and allows fairly complex preconditions, although negation has not been implemented, and only conjunction is allowed.
%
An added bonus illustrated in this example is that any matched precondition variables appear as substitutable grammar variables during expansion of a conditional match (with an \exchar{\_} prepended to their names).


To allow for meaningful preconditions, a handful of special variable names starting with \exchar{\_} are provided which if used inside a precondition will only trigger a match when a specific value is found:
%
\begin{itemize}
  \item \prq{\_Now}{} matches the current timepoint (whichever timepoint the text currently being expanded is associated with).
  \item \prq{\_Opt}{} matches the current option. In some contexts (like setups) there is no current option, and this won't match anything.
  \item \prq{\_Act}{} matches the name of the current action. Only valid when \prq{\_Opt}{} is.
\end{itemize}
%
These variables allow conditions to be expressed over things like the outcomes of the current action (which as predicates include information about the current timepoint and option).
%
Note that preconditions may not contain normal variable substitutions, grammar expansions, or tags, because they are not evaluated in the same way that grammars are.


The main use of conditional expansions is to describe actions differently depending on their outcomes.
%
Although \dunyazad/ does describe the consequences of actions independently, perfectly generic action descriptions leave a lot to be desired, and being able to write different text for an action like ``attack'' when you know the end result will be victory or defeat is useful.
%
On a more practical note, giving the grammar system the ability to interface directly with the ASP results cuts out the middleman so that you don't have to route a variable from predicates through Python every time you want the text to respond more closely to the story situation.


\section{Wildcard Keys}

Another trick that \dunyazad/ has up its sleeve is wildcard keys.
%
The grammar keys are organized into a directory-like structure with key elements separated by \exchar{/} characters.
%
When an entire path element consists of one of the \exchar{?} and \exchar{*} wildcard characters they take on a special meaning.


The \exchar{*} wildcard is simpler: written in a grammar expansion, it means that keys must match up to that point, but afterwards may contain any path elements, which will be ignored for matching purposes.
%
\exchar{*} wildcards may only occur as the final element of a key, and may only occur in grammar expansions, not in key definitions.
%
As an example of the \exchar{*} wildcard, the expansion \prq{[[foo/bar/*]]}{} matches key \prq{foo/bar/baz}{,} and will thus consider expansions listed for \prq{foo/bar/baz}{} along with any other expansions that match it when it is being expanded.


\exchar{?} matching is a bit more complicated, because \exchar{?} is allowed to occur both in expansions and in key definitions.
%
In expansions, the \exchar{?} wildcard matches any single path element, just like the \exchar{*} wildcard but a bit more limited.
%
\prq{[[foo/bar/?]]}{} will match \prq{foo/bar/baz}{} but not \prq{foo/bar/baz/zzx}{.}
%
When \exchar{?} is included in a key \emph{definition}, however, it not only matches as a single wildcard but also binds the special variable \prq{\_}{} to the value that it matched.
%
This allows ``passing-in'' a single argument via the \prq{\_}{} variable without having to make a variable assignment in your expansion statement.
%
In particular, this streamlines the case where one grammar expansion wants to modify a set of other expansions.
%
For example, consider the following pair of definitions:

\begin{quote}
\ttfamily \raggedright
:misc/is\_unskilled/? \\
N\#?who/they V\#be/prs/?who [[misc/unskilled/?\_]] \vspace{0.8\baselineskip} \\

:misc/unskilled/wilderness\_lore \\
ignorant about plants and animals \\ \relax
[[misc/bad\_at]] surviving in the wild \\
incapable of surviving in the wilderness%
\end{quote}

There are individual \prq{misc/unskilled/<skill>}{} definitions for each skill, to help boost variety, although it would be easy enough to define a single generic expansion which worked for every skill.
%
But now we want to go from ``ignorant about plants and animals,'' to ``the merchant is ignorant about plants and animals.''
%
By defining our expansion as \prq{misc/is\_unskilled/?}{} and then using the \exchar{\_} variable, we can effectively provide a \prq{misc/is\_unskilled/<skill>}{} definition for each \prq{misc/unskilled/<skill>}{} definition we had before.
%
Of course, it could also be done using a variable (and in this case the variable \prq{who}{} is required to specify the actor) but the wildcard key version is a bit less clunky and helps keep things mentally organized by folding the variable back into our directory tree structure where it's easier to remember.

\section{Special Directives}

Once all the grammar expansions and variable substitutions have been exhausted and the final pass has been made by substituting tags, it turns out that there are still a few loose ends to sweep up.
%
One problem is that grammar expansions don't know how they'll be used, and often start with a tag in any case, so proper capitalization is difficult.
%
The `S' flag for grammar substitutions helps a lot, but sometimes the author just needs to specify that a particular letter (which will vary because of some form of substitution) should be uppercase.
%
This is handled by the special \prq{@CAP@}{} directive, which removes itself and capitalizes the letter that immediately follows it.


Similarly, for formatting reasons normal text gets run together, but in some cases (like for the prologue), the author needs to be able to insert a paragraph break.
%
The \prq{@PAR@}{} directive does exactly this, and it even behaves appropriately depending on the output format (html vs. txt, for example).
%
The final special directive is simply \prq{@@}{,} which actually appeared in an example above.
%
This is used where a break is needed for parsing reasons, but no whitespace is desired in the output (likely because there's already enough whitespace around).
%
Final processing simply removes the \prq{@@}{} directive.


\section{English Generation Results}

Once special directives have been removed, the text is ready to be seen by humans.
%
Along the way, Python code wraps grammar results up into larger structures depending on the desired output format.
%
\dunyazad/ currently supports plain HTML (non-interactive output for debugging purposes), HTML-in-csv (used to define Amazon Mechanical Turk tasks for experiments), ChoiceScript (the first platform it worked with) and Twine output formats, with some additional options to control whether an entire story or just a single node is output.
%
\Cref{fig:dunyazad-text-example} shows an example of \dunyazad/'s text output copy-pasted from the plain HTML output with some manual editing for formatting.
%
Note that some of the pronominalization is imperfect: the odd option texts result when it tries to refer to items using their owner as context and the owner reference becomes a pronoun, even though later in the same sentence it uses a definite reference.
%
The text overall is easier to read than a version that used determinate noun references everywhere, however, and it has enough variation not to seem blatantly robotic.

\afterpage{
\begin{figure}[!h]
\quotebox{
    \quoteshape
You are about to set out on an epic journey. You are are heading towards the distant country of Jichish, hoping to earn fame and fortune.

You have some perfume and a book of herbal lore, and you have skill: literacy, you have skill: wilderness lore, you have skill: musician, and you have skill: sorcery.

Eager to be on your way, you set off on the road towards Jichish.

A merchant shows up and offers to trade an ancient grimoire.

The merchant is selling an oboe and he is selling his ancient grimoire.

\begin{enumerate}
  \item You steal his ancient grimoire from the merchant (You are missing skill: thievery. He is missing skill: thievery). \\
    $\rightarrow$ You steal his ancient grimoire from the merchant, and he doesn't even notice.
  \item You offer to trade him your book of herbal lore for the merchant's oboe (no relevant skills). \\
    $\rightarrow$ He agrees to trade his oboe for your book of herbal lore.
  \item You steal his oboe from the merchant (You are missing skill: thievery. He is missing skill: thievery). \\
    $\rightarrow$ You steal his oboe from the merchant, and he doesn't even notice.
\vspace*{-1em}
\end{enumerate}
}
\caption[Example output]{An example of \dunyazad/'s text output. In an interactive output format{\footnotemark} the outcome text would be hidden until an option was chosen. Note the awkward pronominalization in the option text.
}
\label{fig:dunyazad-text-example}
\end{figure}
\footnotetext{At the time of this writing, an interactive example could be found online at \url{https://www.cs.hmc.edu/~pmawhorter/projects/dunyazad_example.html}}
}


Through automatic subject-verb agreement, noun introduction, and pronominalization, \dunyazad/ is able to render a wide variety of possible situations into text using only a few very generic text templates.
%
Although there are plenty of discourse-focused systems that do much more complex things, \dunyazad/'s simple grammar-based generator enables it to do what it needs to do without a prohibitive authoring burden.
%
The size of its generative space relies on the fact that individual state and action predicates have descriptions which can be combined freely to describe complex situations.
%
Of course, the quality of these descriptions is limited, but they are good enough to communicate clearly.
%
The grammar-based system employed by \dunyazad/ also allows for the possibility of incremental refinement, as additional variety can be added by adding grammar expansions while conditional expansions allow an author to add nuanced phrases which only show up in specific situations.


