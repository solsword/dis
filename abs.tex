Building on lessons learned from a rational reconstruction of Scott Turner's \minstrel/, \dunyazad/ is a novel system which generates narrative choices.
%
\dunyazad/ is one product of a methodology which uses artificial intelligence to guide the development of a theory, the other product in this case being an analytical framework for understanding the poetics of narrative choices.
%
Two experiments involving human subjects have largely confirmed \dunyazad/'s capabilities, and have helped advance choice poetics.


The research questions addressed involve both computer science and narrative theory, namely: ``How can a computer automatically generate choices that achieve specific poetic effects?'' and ``What poetic effects can choices accomplish within a narrative and how can those be recognized by examining said choices?''
%
As one might expect, the answers provided here are somewhat limited in scope.
%
\dunyazad/ is in fact a program that automatically generates choices with specific qualities, but its repertoire is certainly not ``any poetic effect,'' and as with all such systems, some of the credit for its achievements rests with me, the author, so whether it truly constructs its choices automatically is subject to debate.
%
My theory of choice poetics on the other hand helps explain what choices can accomplish in narratives and how to recognize their effects, but it is focused only on explicit, discrete choices, and it only identifies some possible effects.
%
Despite their limitations, both the system and theory that I developed break new ground, and hopefully they can pave the way for further research into both choice poetics and choice-generating AI systems.


This document begins by outlining my earlier work with Brandon Tearse on \skald/, a rational reconstruction of \minstrel/, and then describes \dunyazad/, which was inspired by my work on \skald/.
%
It also describes choice poetics, the theory that informs and is informed by \dunyazad/'s choice-generation system, and presents the results of two experiments using human subjects that largely confirm \dunyazad/'s ability to produce choices with specific qualities.
%
Finally, it includes guidelines for pursuing the hybrid AI-driven research approach I used to jointly develop choice poetics and \dunyazad/, and highlights some of the unique benefits of this research methodology.
